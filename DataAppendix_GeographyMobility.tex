\documentclass[a4paper, 12pt]{article}
\usepackage[top=2cm, bottom=2cm, left=2.5cm, right=2.5cm]{geometry}
\usepackage[utf8]{inputenc}
\usepackage[colorlinks,colorlinks=false,citecolor=blue,urlcolor=blue,bookmarks=false, hypertexnames=true]{hyperref} 
\usepackage{amsmath, amsfonts, amssymb}
\usepackage{float}
\usepackage{graphicx}
\usepackage{adjustbox}
\usepackage{indentfirst}
\usepackage{booktabs}
\usepackage{tabu}
\usepackage{scalefnt}
\usepackage{pdflscape}
%\usepackage{draftwatermark}
\usepackage{rotating}
\usepackage{caption}
\usepackage{subcaption}
\usepackage{adjustbox}
\usepackage{natbib} [options]
\usepackage{fixmath}
\usepackage[utf8]{inputenc}
\usepackage{xparse}
\usepackage{varwidth}
\usepackage{breqn}
\usepackage{mathtools}
\usepackage{adjustbox}
\usepackage{pdflscape,array,booktabs}
\usepackage{caption}
\usepackage[singlelinecheck=false]{caption}
\usepackage{lscape}
\usepackage[titletoc]{appendix}
\usepackage[skip=1pt]{caption}
\usepackage[nottoc,numbib]{tocbibind}
\usepackage{titling,lipsum}
\usepackage{afterpage}


\makeatletter
\renewcommand*\l@section{\@dottedtocline{1}{1em}{2.5em}}
\makeatother
 







\title{The geography of intergenerational mobility: \\ \small  Evidence of educational persistence and the ``Great Gatsby Curve" in Brazil.} 

%\thanks{Acknowledgments: A previous version of this paper has been presented at the conferences Equal Chances: Equality of }



%\author{\vspace{-0.5em} Tharcisio Leone \\ \vspace{-0.5em} \small GIGA German Institute of Global and Area Studies \\ \small $\&$ Free University of Berlin}

%\date{\today\footnote{This version has been created exclusively for the paper submission process by XX. Please do not quote or circulate.}} 
%\SetWatermarkScale{5}

%for the ``Inclusive Societies" Brown Bag Seminar at GIGA
%the $17^{th}$ Nordic conference on development economics 2018 and in
%for the colloquium titled ``Latin American Economics" at FU-Berlin

\begin{document}
\begin{titlingpage}
\maketitle

\begin{abstract}
This Appendix supplements the working paper ``The geography of intergenerational mobility: Evidence of educational persistence and the Great Gatsby Curve in Brazil". 

\end{abstract}

\end{titlingpage}

\newpage

\tableofcontents
\newpage


\setcounter{table}{0}
\renewcommand{\thetable}{A\arabic{table}}
\setcounter{figure}{0}
\renewcommand{\thefigure}{A\arabic{figure}}


\section{Structure of the Brazilian Educational System}
\label{appendix:EducationSystem}


The current Brazilian educational system is anchored in the 1988 Constitution, which recognises education as a right for the population and an obligation of the government. 

The same legislation distributes the responsibility for education between all three administrative levels of the federation: the federal, state, and municipal governments. Thus, the municipalities are responsible for providing and regulating pre-school education, while the states are involved with the same tasks for the primary and secondary education. The federal government plays only a secondary role in this context, providing financial and technical assistance to the states and municipalities in order to promote equality of opportunity and minimum quality standards.

The main responsibility of the federal government lies in providing education in its institutions – the vast majority of them related to tertiary education – and in regulating the private sector, which is free to operate within all three educational levels.\footnote{Own calculations on the basis of the National Household Sample Survey (PNAD) from 2014 indicated that the share of students enrolled in private institutions in Brazil reached respectively 14\% in primary, 13\% in secondary and 75\% in tertiary education.} After 1996, which saw the publication of the Law of Directives and Bases of National Education (Lei de Diretrizes e Bases da Educação) or LDB, the central government also became responsible for defining a common national basis for curriculum in primary and secondary education, which needs to be used by states and municipalities as the basis for the development of their own curriculums.

Since the 1934 Constitution, there has been compulsory education in Brazil. However, in the beginning only children aged between 7 and 10 years were obliged to undertake full-time education. Over the years the obligatory period of schooling has grown steadily, so that in 1971 compulsory education ended at the age of 14, and in 2010 at 17.\footnote{See \citet{wjuniski2013education} for a detailed description of the changes over time in the legal framework for the educational system in Brazil.} The next table provides an overview of the Brazilian educational system and the changes made to it over the last six decades.




% Table generated by Excel2LaTeX from sheet 'Tabelle7'
\begin{table}[H]
  \centering
%  \captionsetup{justification=centering,margin=1cm}
\caption{\textbf{Structure of Brazilian Educational System.}}
%\vspace*{-2mm}
  \begin{adjustbox}{width=\textwidth,totalheight=\textheight,keepaspectratio}
    \begin{tabular}{ccccc}
    \toprule
    \textbf{Year} & \textbf{Level} & \textbf{Duration (in years)} & \textbf{Age group} & \textbf{Compulsory} \\
    \midrule
          & \multicolumn{1}{l}{Pré-escola (Pre-school)} & 3     & 4 to 6 & No \\
          & \multicolumn{1}{l}{Escola primária (Primary school)} & 4     & 7 to 10 & Yes \\
    \multicolumn{1}{l}{Until 1971} & \multicolumn{1}{l}{Ginásio (Lower high school)} & 4     & 11 to 14 & No \\
          & \multicolumn{1}{l}{Colégio (High school)} & 3     & 15 to 17 & No \\
          & \multicolumn{1}{l}{Ensino superior (College)} & variable   & $\geq18$  & No \\
    \midrule
          & \multicolumn{1}{l}{Pré-escola (Pre-school)} & 3     & 4 to 6 & No \\
    \multicolumn{1}{l}{1971 to 1995} & \multicolumn{1}{l}{1º grau (1st Degree)} & 8     & 7 to 14 & Yes \\
          & \multicolumn{1}{l}{2º grau (2nd Degree)} & 3     & 15 to 17 & No \\
          & \multicolumn{1}{l}{Ensino superior (College)} & variable   & $\geq17$  & No \\
    \midrule
          & \multicolumn{1}{l}{Educação infantil (Early childhood education)} & 7     & 0 to 6 & No \\
    \multicolumn{1}{l}{1996 to 2009} & \multicolumn{1}{l}{Ensino fundamental (Primary education)} & 8     & 7 to 14 & Yes \\
          & \multicolumn{1}{l}{Ensino médio (Secondary education)} & 3     & 15 to 17 & No \\
          & \multicolumn{1}{l}{Ensino superior (College)} & variable   & $\geq17$  & No \\
    \midrule
    & \multicolumn{1}{l}{Educação infantil (Early childhood education)} & 4     & 0 to 3 & No \\
    & \multicolumn{1}{l}{Ensino pré-fundamental (Pre-primary education)} & 2     & 4 to 5 & Yes \\
    \multicolumn{1}{l}{Since 2010} & \multicolumn{1}{l}{Ensino fundamental (Primary education)} & 9     & 6 to 14 & Yes \\
          & \multicolumn{1}{l}{Ensino médio (Secondary education)} & 3     & 15 to 17 & Yes \\
          & \multicolumn{1}{l}{Ensino superior (College)} & variable   & $\geq17$  & No \\
    \midrule
    \multicolumn{5}{l}{\small{Source: Law 5,540 of 28/11/1968, Law 5,692 of 11/08/1971, Law 9,394 of 20/12/1996 and Constitutional Amendment 59 of 11/11/2009.}} \\
   \end{tabular}%
  \label{tab:EducationSystem}%
  \end{adjustbox}	
\end{table}%


%Currently, education is compulsory in Brazil for all individuals between 6 and 14 years of age.

Currently, the Brazilian educational system is composed of five distinct levels: early childhood, pre-primary, primary, secondary and tertiary education. Individuals aged between 4 and 17 years are obliged to attend school. Children under four may attend the optional early childhood education. Attendance at the pre-primary educational level, usually at the age of 4, is the first phase of compulsory education. This is followed by the primary educational level, which comprises nine years of schooling. The third level of the educational system in Brazil is known as secondary education and lasts for a period of three years. Students who complete this level have the right to attend vocational training, or to start pursuing higher education qualifications: a bachelor’s degree, for example, usually takes four years. Individuals who hold a university degree are eligible to undertake graduate studies, which consist of a master’s degree followed, potentially, by a doctoral degree.

The currently requirement that children complete $14$ years of compulsory education in Brazil was stipulated by constitutional amendment 59 of 11 November 2009, which created an obligatory 2+9+3 pattern in the education system. This was an increase from the previous system (valid until 2009), where students were required to remain in school only for nine years.\footnote{Although the increasing of the compulsory education level via the constitutional amendment had already been established in September 2009, the states and municipalities had until 2016 to achieve its full implementation.}



%https://www.nafsa.org/uploadedFiles/Chez_NAFSA/Resource_Library_Assets/Networks/ACE/EDU%20Systems%20Brazil.pdf



%\newpage
\section{Codification of years of schooling}
\label{appendix:CodificationSchooling}


Based on the PNAD sample, this paper used two main variables related to education for the investigation of intergenerational mobility: the number of completed years of education (years of schooling) and the (highest) educational level achieved. The PNAD already provides both variables for the children’s generation, but for the parents the information on years of schooling is missing.
 
Given this limitation, I calculated the parents’ years of schooling according to their educational levels. Table \ref{tab:Codification} presents the matching procedure used for the codification.\footnote{See table \ref{tab:EducationSystem} for an overview of the different educational levels used in the codification.}


\begin{table}[H]
\captionsetup{justification=centering,margin=1cm}
\caption{\textbf{Codification of Parents’ Years of Schooling.}}
\vspace*{-5mm}
   \label{tab:Codification}%
 \tiny{\[
    Years \quad of \\ \quad Education =\left\{
                \begin{array}{ll}
                  00 \qquad if \qquad $Only pre-primary education$\\
                  00 \qquad if \qquad $Went to school, but never completed $1^{st}$ grade$\\
                  02 \qquad if \qquad $Completed $1^{st}$ grade but didn’t complete all grades up to $4^{th}$ grade (before 1971)$\\
                  02 \qquad if \qquad $Uncompleted literacy classes (young people and adults)$\\
                  03 \qquad if \qquad $Attended literacy classes (young people and adults), but do not know if they were completed$\\
                  03 \qquad if \qquad $Attended primary school, but do not know if all grades up to $4^{th}$ grade were completed (before 1971)$\\
                  04 \qquad if \qquad $Completed up to $4^{th}$ grade$\\
                  05 \qquad if \qquad $Completed literacy classes (young people and adults)$\\
                  05 \qquad if \qquad $Completed $1^{st}$ grade but didn’t complete all grades up to $8^{th}$ grade (after 1971)$\\
                  06 \qquad if \qquad $Completed $5^{th}$ grade but didn’t complete all grades up to $8^{th}$ grade (before 1971)$\\
                  07 \qquad if \qquad $Attended $1^{st}$ degree, but do not know if all grades up to $8^{th}$ grade were completed (after 1971)$\\
                  07 \qquad if \qquad $Attended lower high school, but do not know if all grades up to $8^{th}$ grade were completed (before 1971)$\\
                  08 \qquad if \qquad $Completed up to $8^{th}$ grade$\\
                  09 \qquad if \qquad $Completed $9^{th}$ grade but didn’t complete all grades up to $11^{th}$ grade $\\
                  10 \qquad if \qquad $Attended $2^{nd}$ degree, but do not know if all grades up to $11^{th}$ grade were completed (after 1971)$\\
                  11 \qquad if \qquad $Completed up to $11^{th}$ grade$\\
                  13 \qquad if \qquad $Completed $1^{st}$ year in college/university, but didn’t graduate$\\
                  14 \qquad if \qquad $Attended college/university, but do not know if graduated$\\
                  15 \qquad if \qquad $Graduated college/university$\\
                  16 \qquad if \qquad $Incomplete master or doctorate$\\
                  17 \qquad if \qquad $Attended master’s or doctoral studies, but do not know if they were completed$\\
                  19 \qquad if \qquad $Completed master’s or doctorate$\\
                \end{array}
              \right.
  \]}
\end{table}


It is important to note that the information concerning the parents’ educational level is based on the self-declaration of their children – i.e. the individuals who were interviewed by PNAD – and refers to educational attainment of parents when the children were 15 years old.\footnote{Because the investigation of intergenerational mobility in this paper is based on children born between 1940 and 1989, the reform of the education system through constitutional amendment 59 of 11 November 2009 had no consequences for the codification of parents’ education.} Thereby, three variables from PNAD have been used for the codification of parents’ years of schooling: (a) the highest level of education attended, (b) whether the first year (grade) of this attended level was completed, and (c) whether the attended level was also completed. 


For the first variable, 10 different educational levels were permitted: kindergarten, literacy classes for six-year-olds, literacy classes for young people and adults, primary school, lower high school, high school, $1^{st}$ degree/primary education, $2^{nd}$ degree/secondary education, college, and master or doctorate. For the second and third variables only three answers were possible: yes, no, or unknown.



%\newpage
\section{Data Harmonisation}
\label{appendix:Harmonisation}


The empirical investigations in this paper are based on the Brazilian National Household Sample Survey (PNAD). This nationally representative survey has been conducted annually since 1981 by the Brazilian Institute of Geography and Statistics (IBGE) and gathers information about household composition, educational attainment, labour market status, income, and a set of demographic variables (age, gender, location, race, etc.).\footnote{Until 2003, the rural areas of Rondônia, Acre, Amazonas, Roraima, Pará e Amapá were not part of the PNAD. These six states compose Brazil's Northern region and their population living in rural areas constitutes around 3 percent of the total Brazilian population.}

In principle, it is possible to observe a relative consistency between the different sets of PNAD microdata over time. However, through the years the PNAD has undergone some restructuring in methodological terms, and for this reason some variables are not available for all the years and/or may not have been collected in the same way. 

For this paper, the particularly relevant change was the reformulation of the definition of labour activities that occurred in 1992. The new formulation aimed to integrate some subsamples of the population involved in economic activities that were, previously not included in the occupied population. Particularly noteworthy was the establishment of three additional categories of workers: those involved in production for self-consumption, construction for their own personal use, and paid domestic work. For this reason, it was necessary to harmonize all the PNADs' microdata to make the information about income inequality used in the investigations of the ``Great Gatsby Curve" and the decision to drop out of school compatible.

For the standardization process I took the survey from 1981 as the initial base and made the subsequent PNADs compatible with it. This required that only those variables which already existed in the 1981s` sample be maintained for the investigation.\footnote{This standardization process was made using the $``datazoom-pnad"$ package developed by the Department of Economics at Pontifical Catholic University of Rio de Janeiro (PUC-Rio), which aimed at compiling all the variables over the last four decades that could be obtained and organized in a conceptually consistent way.}

For the measures of Gini coefficient and ratio 75/10, I followed the theoretical approach of \citet{hoffmann2006queda} and calculated the income inequality based on (positive) monthly personal income for the economically active population aged 15 or over. In the integrated data there are 12 variables related to income that are common to all the samples. For the investigation, I used the variable (personal) monthly income from all sources, which is derived from the sum of all job income, retirement, pension, rent, allowances and other sources. Subsequently, the variable related to income was deflated to the year 2012 with help of an income deflation based on the National Consumer Price Index (INPC/IBGE). Not least, I omitted the observations with income equalling zero to exclude those individuals performing unpaid work (care work, voluntary work, etc.) from the analysis.

In this paper, the economically active population consists of those individuals who were either employed or actively seeking employment in the PNAD reference week. Finally, because the state of Tocantins was created only in year 1989, I aggregated its data with Goiás for those years in which the separation had already occurred.\footnote{In the 1988 Brazilian Constitution, the state of Tocantins was officially created from the northern two-fifths of Goiás and admitted as a new state.}


The investigation of mobility conducted in this study is based on an intertemporal choice about (more) educational attainment, that occurred $14$ years after the birth of the children. Therefore, the first PNAD sample (1981) was used to investigate the educational choices of the individuals born in 1967, and the PNAD from 2003 for the investigation of children born in 1989. Because there are no nationally representative databases for the period prior to 1981 that could be harmonized in a reliable way with the PNADs generated after 1981, this paper limited the estimations in sections \ref{Correlating} and \ref{Dropout} to the individuals born from the year 1965 onwards.\footnote{For the individuals born in 1965 and 1966, I used the inequality level from 1981 as a proxy. In the years 1991, 1994 and 2000 the PNAD was not carried out. For that purpose, I used the inequality levels for the respective following years (1992, 1995 and 2001) by the investigation of individuals born respectively in 1977, 1980 and 1986.}





\newpage
\section{A Model of the Intergenerational Transmission of Income Inequality}
\label{appendix:Model-Solon}


%\section{Appendix: Counting words without repetitions}


The model of \citet{Solon2004}, which is based on the theoretical approach of \citet{Becker&Tomes1979, Becker&Tomes1986}, has been used in the economic literature as the starting point to understand the correlation between income inequality and intergenerational mobility.\footnote{The description of the theoretical model of \citet{Solon2004} in this appendix refers to the simplified version of the model presented by \citet{solon2014theoretical}.}

In this model, the family $i$ is composed of a parent from generation $t-1$ and a child from generation $t$ and it is assumed as an intergenerational decision maker, that should allocate the lifetime earnings gained from parent $y_{i,t-1}$ into only two goods: the parent’s own consumption $C_{i,t-1}$ and investment in the child’s human capital $I_{i,t-1}$, subject to the budget constraint:\footnote{For purposes of simplification, the model of \citet{Solon2004} presented in this section does not take into account variables such as taxation, government investment in children, and the borrowing and bequeathing of financial assets. See \citet{Becker&Tomes1986} for a more complete version of the model.}


\begin{equation} \label{eq:Budget}
y_{i,t-1}=C_{i,t-1}+I_{i,t-1}
\end{equation}

The investment $I_{i,t-1}$ presents diminishing marginal returns on education and will form the child’s stock of human capital $h_{it}$ used in the future to produce economic value in the labour market


\begin{equation} \label{eq: Technology}
h_{it}=\theta \ \log \ I_{i,t-1}+e_{it} 
\end{equation}


The diminishing marginal returns on education in equation \eqref{eq: Technology} refer to the fact that (more) investments produce even a positive marginal product for human capital stock $\theta > 0$, but of a lesser and less additional value, due to the semi-log form of the function. The error term $e_{it}$ accounts for the variation in the child’s human capital endowment that cannot be explained by the investment of parents, referring mainly to genetic endowment and personality traits which are also transmitted in the family environment and play an important role in human capital accumulation. 

The independent human capital endowment $e_{it}$ follows a first-order autoregressive process as:


\begin{equation} \label{eq: Human}
e_{it}=\delta + \lambda \: e_{i,t-1}+\upsilon_{i,t}
\end{equation}


in which $\upsilon_{i,t}$ represents a white-noise error term and the parameter $\lambda$ is a heritability coefficient with $\lambda \in [0,1]$


\citet{Solon2004} assumed in his model that the lifetime income of child $y_{it}$ can be regarded as a semi-log earnings function, where $p$ represents the earnings return on human capital.

\begin{equation} \label{eq: Lifetime}
log \ y_{it}=\mu + p \ h_{it} 
\end{equation}


Substituting equation \eqref{eq: Technology} into equation \eqref{eq: Lifetime}, we have: 

\begin{equation} \label{eq: Subs1}
\log \ y_{it}=\mu + p \ \theta \ log \ I_{i,t-1} + p \ e_{it}
\end{equation}

From equation \eqref{eq: Subs1} the value $p \ \theta$ can be interpreted as the elasticity of the child's income in relation to the human capital investment $I_{i,t-1}$, representing in this way the earnings return on human capital investment, in the following briefly termed as $\gamma$. 
 
\begin{equation} \label{eq: Substituting}
\log \ y_{it}=\mu + \gamma \ log \ I_{i,t-1} + pe_{it}
\end{equation}


To make the optimal decisions concerning the investment in the child's human capital, the family considers a two-good world, in which the parents' lifetime income $y_{i,t-1}$ needs to be allocated between their own consumption $C_{i,t-1}$ and investment $I_{i,t-1}$ in the child's human capital. The family wishes to maximise the utility, denoted $U_i(C_{i,t-1},I_{i,t-1})$, subject to the budget constraint in \eqref{eq: Cobb-Douglas}:


\begin{equation} \label{eq: Cobb-Douglas} 
U_i=(1-\alpha) \log \ C_{i,t-1}+ \alpha \log \ y_{it}
\end{equation}


with $\alpha \in [0,1]$ indicating the degree of altruism of parents for child's income $y_{it}$ in relation to their own consumption $C_{i,t-1}$. Plugging \eqref{eq:Budget} and \eqref{eq: Substituting} into equation \eqref{eq: Cobb-Douglas}:

\begin{equation} \label{eq: rewritten0} 
U_i = (1 - \alpha) \log \ \underbrace{(y_{i,t-1}-I_{i,t-1})}_{\text{C_{i,t-1}}} + \alpha \underbrace{(\mu + \gamma \log \ I_{i,t-1} + pe_{it})}_{y_{it}}
\end{equation}

And rewriting it:

\begin{equation} \label{eq: rewritten} 
U_i = (1 - \alpha) \log \ (y_{i,t-1}-I_{i,t-1})+ \alpha \mu + \alpha \gamma \log \ I_{i,t-1} + \alpha pe_{it}
\end{equation}

In order to solve the problem, the main condition for the maximisation of the utility function is that 

\begin{equation} \label{eq: First-order} 
\frac{\partial U_i}{\partial I_{i,t-1}} = \frac{-(1-\alpha)}{y_{i,t-1}-I_{i,t-1}} + \frac{\alpha \gamma}{I_{i,t-1}}=0
\end{equation}

Solving for the optimal choice of $I_{i,t-1}$, we can rewrite the first-order condition as 


\begin{equation} \label{eq: Optimal-choice} 
I_{i,t-1} = \left \{ \frac{\alpha \gamma}{1 - \alpha (1 - \gamma)} \right \} y_{i,t-1}
\end{equation}

Note that the investment in the child's human capital $I_{i,t-1}$ increases by increasing parents’ income $y_{i,t-1}$, altruism $\alpha$, and earnings return on human capital investment $\gamma$. From these results, we can deduce the two most important conclusions from the model of \citet{Solon2004}: Firstly, parents with higher income have a higher financial capacity to invest in the human capital of their children, and secondly, they also have a greater incentive to make this investment if the return on investment in human capital increases over time.




\vspace{0.5cm}

\numberwithin{equation}{section}
\setcounter{equation}{0}

%\section*{\centering Appendix B \\ \normalsize A Stylized Model of the Decision to Dropout Education System}

%{\color{red}(The model still has to be derivated.)}




\newpage
\section{A Stylised Model for Dropout of Education System}
\label{appendix:kearneyLevine}


\citet{kearney2014income} presented the theoretical model used in this paper to explain the causal relationship between higher income inequality and the higher probability that children from socially disadvantaged families will drop out of school.

Let us assume that the child (student) tends to maximise an intra-generational utility function between utility in the current $(t)$ and future period $(t+1)$. If the student drops out of the education system in $t$, he or she will achieve the current-period utility $u^d$ and the present discounted sum of future period $V^d$. Otherwise, the student has $u^e$ and $V^e$ from the decision to remain enrolled in the school. The generalisation of the individual decision to drop out can be written as: 



\begin{equation} \label{eq:Condition1} 
u^d + E(V^d)>u^e + E(V^e)
\end{equation}

Given the positive returns on education, we assume that dropping out of school has a negative effect on the utility in period $t+1$, due to the reduction in the level of future consumption, such that $E(V^e)>E(V^d)$. 

The decision to drop out of education will be never optimal so long as $u^d \leq u^e$. However, if $u^d > u^e$ in the case that the student's participation in the school system is associated with substantial utility costs, such as psychic costs, then dropping out of school can be considered an alternative. 

Suppose that the child's utility in the future can achieve $U^{high}$ or $U^{low}$ – i.e. a high or low value respectively – and $U^{low}$ represents the utility level in the case that the child drops out the school. If the student remains enrolled in the education system, he or she will have the probability $p \in [0,1]$ of attaining the high-utility position. Assuming $V^{low}$ as the deterministic present discounted value of the utility, we can rewrite equation \eqref{eq:Condition1} as:


\begin{equation} \label{eq:Condition2} 
u^d + V^{low} > u^e + pV^{high} + (1-p)V^{low}
\end{equation}

By rearranging the terms in equation \eqref{eq:Condition2}, the condition for remaining in school yields:


\begin{equation} \label{eq:Condition3} 
\left[pV^{high} + (1-p)V^{low} \right] - V^{low} > u^d - u^e  
\end{equation}

Thus, the student will continue studying as long as the likelihood of attaining a high utility in the future is greater than the current loss of utility caused by school attendance and his/her consequent sacrifice of leisure. Given the uncertainty associated with the future, the child cannot determine $p$ with the best possible accuracy in the period $t$, working in this way with its individual subjective perception of success $q$. 

Let's assume $q$ as a function of $p$ and $x$, such that $q=q(p,x)$ in which $x$ represents external factors affecting the individual's perception of returns on schooling. \citet{kearney2014income} have pointed out that these external factors can be influenced strongly by the lived experience during childhood and adolescence. Children who grow up in poverty have restricted contact to highly qualified individuals and may assume that a college degree is an objective very far from their reality, leading to an underestimation of the probability $p$. As a result, at the same level of $p$, students from different socio-economic backgrounds (SES) will present different individual perceptions of $q$.

This means that, income inequality will affect the perceived returns on education $q$ in two ways: Firstly, it affects $x$ given that the higher the inequality, the higher the perception of social exclusion for poor children. Secondly, higher income inequality will increase the current return on investment in schooling, leading to a rise in the individual perception of return $p$. Then the condition for the student to continue studying follows:


\begin{equation} \label{eq:Condition4} 
\left[qV^{high} + (1-q)V^{low} \right] > V^{low} + (u^d - u^e) 
\end{equation}


From equation \eqref{eq:Condition4} it becomes evident that the chance of remaining enrolled rises with increasing $q$. Therefore, the student will invest more time in schooling if he or she notes that this investment will increase the chance of achieving $V^{high}$. However, if children are right in assuming that independently of their educational attainment they will never leave the situation of social exclusion, i.e. if $q$ is very low, this increases the incentive to drop out of school.

Solving the equation \eqref{eq:Condition4} for $q$, we can define the reservation subjective probability $q^r$ required students’ continuation of schooling.


\begin{equation} \label{eq:Condition5} 
q \geq q^r = \left \{ \frac{u^d - u^e}{V^{high} - V^{low}} \right \}
\end{equation}

The derivative from \eqref{eq:Condition5} to the socio-economic backgrounds (SES) represented an increasing function at point $q$, indicating that the higher the SES, the greater the perception of success as a consequence of educational attainment, such that:


\begin{equation} \label{eq:Condition6}
\frac{\partial q}{\partial (SES)} > 0
\end{equation}

\citet{kearney2014income} propose that the perception of success $q$ can also be described as a function of SES and income inequality in the society. For children from socially weaker families, the increase in the gap between the bottom and middle of the income distribution might lead to a reduction of the subjective perception $p$.


\begin{equation} \label{eq:Condition7}
\frac{\partial q}{\partial (SocIneq)} < 0
\end{equation}

In practice, it means that the farther away poor children’s experiences are from the experiences of the middle class, the greater their perception of “social exclusion", strengthening in this way the individual view that “it is not for people like me".







\newpage
\section{Figures}
\label{appendix:Figures}

\begin{figure}[H]
\centering
    \includegraphics[width=0.8\textwidth]{Figure/Evolution-Mean}
    \label{fig:Evolution-Mean} 
\begin{minipage}{0.8\textwidth} % choose width suitably
{\scriptsize
Notes: Children's education for boys and girls. Estimations of parent's education based on educational attainment of the most educated parent.\\ Source: PNAD-2014, own estimates.\par}
\end{minipage}
\captionsetup{justification=centering,margin=2cm}
\caption{\textbf{Development of Average Schooling, per State}}
\end{figure}



\begin{figure}[H]
\centering
   \includegraphics[width=0.8\textwidth]{Figure/Evolution-SD}
    \label{fig:Evolution-SD}
\begin{minipage}{0.8\textwidth} % choose width suitably
{\scriptsize
Notes: Children's education for boys and girls. Estimations of parent's education based on educational attainment of the most educated parent.\\ Source: PNAD-2014, own estimates.\par}
\end{minipage}
\captionsetup{justification=centering,margin=1cm}
\caption{\textbf{Development of Inequality in Schooling, per State}}
\end{figure}





\begin{figure}[H]
\centering
\includegraphics[scale=1]{Figure/Mean-Schooling}
\label{fig:Mean-Schooling}
\begin{minipage}{0.88\textwidth} % choose width suitably
{\scriptsize
Note: Estimations for boys and girls. \\
Source: PNAD-2014, own estimates.\par}
\end{minipage}
\captionsetup{justification=centering,margin=2cm}
\caption{\textbf{Average Years of Schooling}}
\end{figure}

\vspace{20pt}


\begin{figure}[htb]
\centering
\includegraphics[scale=1]{Figure/Composition-Level}
\label{fig:Composition-Level}
\begin{minipage}{0.88\textwidth} % choose width suitably
{\scriptsize
Note: Estimations for boys and girls. \\
Source: PNAD-2014, own estimates.\par}
\end{minipage}
\captionsetup{justification=centering,margin=2cm}
\caption{\textbf{Levels of Education, by Regions and States.}}
\end{figure}




\begin{figure}[htb]
\centering
      \includegraphics[width=0.9\textwidth]{Figure/OutcomeShare}
     \label{fig:OutcomeShare}
     \begin{minipage}{0.9\textwidth} % choose width suitably
{\scriptsize
Note: Estimations based on self-declared per capita domiciliary income. \\
Source: PNAD-2014, own estimates.\par}
\end{minipage}
\captionsetup{justification=centering,margin=2cm}
\caption{\textbf{Income inequality across states.}}
\end{figure}   


\vspace{60pt}

\begin{figure}[H]
   \centering
   \includegraphics[width=0.75\textwidth]{Figure/Ratio-7510-v4719}
   \label{fig:Ratio5010a}
   \begin{minipage}{0.77\textwidth} % choose width suitably
{\scriptsize
Notes: Ratio 75/10 represents the relation between the income earned by individuals at the 75th percentile compared to the earnings of individuals at the 10th percentile. Estimations based on total income for the economically active population with earnings greater than zero and aged 15 or over. \\
Source: PNAD-2014, own estimates.\par}
\end{minipage}
   \captionsetup{justification=centering,margin=2cm}
   \caption{\textbf{Ratio 75/10 of income distribution}}
 \end{figure}





\newgeometry{,vmargin=3cm,hmargin=5cm}
\begin{sidewaystable}

\section{Tables}






%\newgeometry{top=1cm,right=-8cm}

 
% Table generated by Excel2LaTeX from sheet 'DescriptiveSta'
\begin{table}[H]\clearpage 
%\captionsetup{justification=centering,margin=1cm}
\caption{\textbf{Weighted Descriptive Statistics (PNAD-2014).}}
%\vspace*{-2mm}
  \centering
   \label{tab:descriptive}
   \begin{adjustbox}{max height={20cm},max width={25cm}}
    \begin{tabular}{lllllllllllllllllllllll}
   \toprule
          &       &       & \multicolumn{1}{c}{(1)} & \multicolumn{1}{c}{(2)} & \multicolumn{1}{c}{(3)} &       & \multicolumn{1}{c}{(4)} & \multicolumn{1}{c}{(5)} & \multicolumn{1}{c}{(6)} &       & \multicolumn{1}{c}{(7)} & \multicolumn{1}{c}{(8)} & \multicolumn{1}{c}{(9)} &       & \multicolumn{1}{c}{(10)} & \multicolumn{1}{c}{(11)} & \multicolumn{1}{c}{(12)} & \multicolumn{1}{c}{(13)} &       & \multicolumn{1}{c}{(14)} & \multicolumn{1}{c}{(15)} & \multicolumn{1}{c}{(16)} \\
    \multicolumn{2}{c}{\textbf{State}} &       & \multicolumn{3}{c}{\textbf{Population}} &       & \multicolumn{3}{c}{\textbf{Income distribution (R\$)}} &       & \multicolumn{3}{c}{\textbf{Net enrolment ratio (age)}} &       & \multicolumn{4}{c}{\textbf{Average years of schooling }} &       & \multicolumn{3}{c}{\textbf{Standard deviation}} \\
\cmidrule{1-2}\cmidrule{4-6}\cmidrule{8-10}\cmidrule{12-14}\cmidrule{16-19}\cmidrule{21-23}    \multicolumn{1}{c}{\textbf{Name}} & \multicolumn{1}{c}{\textbf{Abbrev.}} &       & \multicolumn{1}{c}{\textbf{Total}} & \multicolumn{1}{c}{\textbf{Average age}} & \multicolumn{1}{c}{\textbf{Ratio in rural}} &       & \multicolumn{1}{c}{\textbf{Bottom}} & \multicolumn{1}{c}{\textbf{Middle}} & \multicolumn{1}{c}{\textbf{Top}} &       & \multicolumn{1}{c}{\textbf{7 - 14}} & \multicolumn{1}{c}{\textbf{15 - 17}} & \multicolumn{1}{c}{\textbf{16 - 24}} &       & \multicolumn{1}{c}{\textbf{Obs.}} & \multicolumn{1}{c}{\textbf{Children}} & \multicolumn{1}{c}{\textbf{Fathers}} & \multicolumn{1}{c}{\textbf{Mothers}} &       & \multicolumn{1}{c}{\textbf{Children}} & \multicolumn{1}{c}{\textbf{Fathers}} & \multicolumn{1}{c}{\textbf{Mothers}} \\
    \midrule
    Rondônia & \multicolumn{1}{p{5.39em}}{RO} &       & \multicolumn{1}{c}{1,748,531} & \multicolumn{1}{c}{30.65} & \multicolumn{1}{c}{0.2369} &       & \multicolumn{1}{c}{204} & \multicolumn{1}{c}{593} & \multicolumn{1}{c}{1,666} &       & \multicolumn{1}{c}{0.9921} & \multicolumn{1}{c}{0.7948} & \multicolumn{1}{c}{0.2972} &       & \multicolumn{1}{c}{541} & \multicolumn{1}{c}{8.4842} & \multicolumn{1}{c}{2.9488} & \multicolumn{1}{c}{3.2734} &       & \multicolumn{1}{c}{4.5993} & \multicolumn{1}{c}{3.5316} & \multicolumn{1}{c}{3.9711} \\
    Acre  & \multicolumn{1}{p{5.39em}}{AC} &       & \multicolumn{1}{c}{790,101} & \multicolumn{1}{c}{27.27} & \multicolumn{1}{c}{0.2591} &       & \multicolumn{1}{c}{128} & \multicolumn{1}{c}{400} & \multicolumn{1}{c}{1,500} &       & \multicolumn{1}{c}{0.9653} & \multicolumn{1}{c}{0.7616} & \multicolumn{1}{c}{0.3191} &       & \multicolumn{1}{c}{415} & \multicolumn{1}{c}{7.3954} & \multicolumn{1}{c}{2.9512} & \multicolumn{1}{c}{3.7874} &       & \multicolumn{1}{c}{5.3104} & \multicolumn{1}{c}{4.2085} & \multicolumn{1}{c}{4.7970} \\
    Amazonas & \multicolumn{1}{p{5.39em}}{AM} &       & \multicolumn{1}{c}{3,873,743} & \multicolumn{1}{c}{28.49} & \multicolumn{1}{c}{0.1634} &       & \multicolumn{1}{c}{150} & \multicolumn{1}{c}{438} & \multicolumn{1}{c}{1,500} &       & \multicolumn{1}{c}{0.9776} & \multicolumn{1}{c}{0.8294} & \multicolumn{1}{c}{0.3482} &       & \multicolumn{1}{c}{1,045} & \multicolumn{1}{c}{8.2446} & \multicolumn{1}{c}{3.8986} & \multicolumn{1}{c}{4.1803} &       & \multicolumn{1}{c}{4.7067} & \multicolumn{1}{c}{4.2232} & \multicolumn{1}{c}{4.2799} \\
    Roraima & \multicolumn{1}{p{5.39em}}{RR} &       & \multicolumn{1}{c}{496,936} & \multicolumn{1}{c}{28.37} & \multicolumn{1}{c}{0.1680} &       & \multicolumn{1}{c}{191} & \multicolumn{1}{c}{530} & \multicolumn{1}{c}{1,862} &       & \multicolumn{1}{c}{0.9865} & \multicolumn{1}{c}{0.7576} & \multicolumn{1}{c}{0.3238} &       & \multicolumn{1}{c}{160} & \multicolumn{1}{c}{10.0415} & \multicolumn{1}{c}{4.1797} & \multicolumn{1}{c}{4.5257} &       & \multicolumn{1}{c}{4.2826} & \multicolumn{1}{c}{4.2936} & \multicolumn{1}{c}{4.3761} \\
    Pará  & \multicolumn{1}{p{5.39em}}{PA} &       & \multicolumn{1}{c}{8,073,924} & \multicolumn{1}{c}{29.82} & \multicolumn{1}{c}{0.2992} &       & \multicolumn{1}{c}{133} & \multicolumn{1}{c}{399} & \multicolumn{1}{c}{1,185} &       & \multicolumn{1}{c}{0.9824} & \multicolumn{1}{c}{0.8454} & \multicolumn{1}{c}{0.3268} &       & \multicolumn{1}{c}{2,019} & \multicolumn{1}{c}{7.3653} & \multicolumn{1}{c}{3.3370} & \multicolumn{1}{c}{3.6478} &       & \multicolumn{1}{c}{4.4826} & \multicolumn{1}{c}{3.7415} & \multicolumn{1}{c}{4.0299} \\
    Amapá & \multicolumn{1}{p{5.39em}}{AP} &       & \multicolumn{1}{c}{750,912} & \multicolumn{1}{c}{27.46} & \multicolumn{1}{c}{0.1039} &       & \multicolumn{1}{c}{200} & \multicolumn{1}{c}{499} & \multicolumn{1}{c}{1,860} &       & \multicolumn{1}{c}{0.9917} & \multicolumn{1}{c}{0.8541} & \multicolumn{1}{c}{0.3197} &       & \multicolumn{1}{c}{217} & \multicolumn{1}{c}{9.3247} & \multicolumn{1}{c}{4.5997} & \multicolumn{1}{c}{4.2800} &       & \multicolumn{1}{c}{4.8566} & \multicolumn{1}{c}{4.6337} & \multicolumn{1}{c}{4.4242} \\
    Tocantins & \multicolumn{1}{p{5.39em}}{TO} &       & \multicolumn{1}{c}{1,496,880} & \multicolumn{1}{c}{31.59} & \multicolumn{1}{c}{0.2148} &       & \multicolumn{1}{c}{164} & \multicolumn{1}{c}{500} & \multicolumn{1}{c}{1,674} &       & \multicolumn{1}{c}{0.991} & \multicolumn{1}{c}{0.8205} & \multicolumn{1}{c}{0.3233} &       & \multicolumn{1}{c}{545} & \multicolumn{1}{c}{7.6599} & \multicolumn{1}{c}{2.3349} & \multicolumn{1}{c}{2.9175} &       & \multicolumn{1}{c}{4.7665} & \multicolumn{1}{c}{3.3784} & \multicolumn{1}{c}{3.9490} \\
    \textbf{North} &       &       & \multicolumn{1}{c}{\textbf{17,231,027}} & \multicolumn{1}{c}{\textbf{29.50}} & \multicolumn{1}{c}{\textbf{0.2408}} &       & \multicolumn{1}{c}{\textbf{146}} & \multicolumn{1}{c}{\textbf{437}} & \multicolumn{1}{c}{\textbf{1,433}} &       & \multicolumn{1}{c}{\textbf{0.9826}} & \multicolumn{1}{c}{\textbf{0.829}} & \multicolumn{1}{c}{\textbf{0.3277}} &       & \multicolumn{1}{c}{\textbf{4,942}} & \multicolumn{1}{c}{\textbf{7.8259}} & \multicolumn{1}{c}{\textbf{3.3986}} & \multicolumn{1}{c}{\textbf{3.7343}} &       & \multicolumn{1}{c}{\textbf{4.6621}} & \multicolumn{1}{c}{\textbf{3.9186}} & \multicolumn{1}{c}{\textbf{4.1553}} \\
    Maranhão & \multicolumn{1}{p{5.39em}}{MA} &       & \multicolumn{1}{c}{6,850,884} & \multicolumn{1}{c}{30.14} & \multicolumn{1}{c}{0.4083} &       & \multicolumn{1}{c}{89} & \multicolumn{1}{c}{333} & \multicolumn{1}{c}{1,015} &       & \multicolumn{1}{c}{0.9825} & \multicolumn{1}{c}{0.8508} & \multicolumn{1}{c}{0.2747} &       & \multicolumn{1}{c}{864} & \multicolumn{1}{c}{6.2833} & \multicolumn{1}{c}{1.9498} & \multicolumn{1}{c}{2.6083} &       & \multicolumn{1}{c}{4.9003} & \multicolumn{1}{c}{3.3379} & \multicolumn{1}{c}{3.8792} \\
    Piauí & \multicolumn{1}{p{5.39em}}{PI} &       & \multicolumn{1}{c}{3,194,718} & \multicolumn{1}{c}{32.26} & \multicolumn{1}{c}{0.3247} &       & \multicolumn{1}{c}{117} & \multicolumn{1}{c}{400} & \multicolumn{1}{c}{1,114} &       & \multicolumn{1}{c}{0.9879} & \multicolumn{1}{c}{0.8552} & \multicolumn{1}{c}{0.3149} &       & \multicolumn{1}{c}{628} & \multicolumn{1}{c}{6.0786} & \multicolumn{1}{c}{1.7586} & \multicolumn{1}{c}{2.2864} &       & \multicolumn{1}{c}{5.1269} & \multicolumn{1}{c}{3.2388} & \multicolumn{1}{c}{3.8753} \\
    Ceará & \multicolumn{1}{p{5.39em}}{CE} &       & \multicolumn{1}{c}{8,842,791} & \multicolumn{1}{c}{33.38} & \multicolumn{1}{c}{0.2648} &       & \multicolumn{1}{c}{120} & \multicolumn{1}{c}{400} & \multicolumn{1}{c}{1,134} &       & \multicolumn{1}{c}{0.9824} & \multicolumn{1}{c}{0.8346} & \multicolumn{1}{c}{0.2742} &       & \multicolumn{1}{c}{2,018} & \multicolumn{1}{c}{6.7593} & \multicolumn{1}{c}{2.3395} & \multicolumn{1}{c}{2.8007} &       & \multicolumn{1}{c}{4.9012} & \multicolumn{1}{c}{3.7040} & \multicolumn{1}{c}{3.9939} \\
    Rio Grande do Norte & \multicolumn{1}{p{5.39em}}{RN} &       & \multicolumn{1}{c}{3,408,510} & \multicolumn{1}{c}{32.90} & \multicolumn{1}{c}{0.2352} &       & \multicolumn{1}{c}{150} & \multicolumn{1}{c}{434} & \multicolumn{1}{c}{1,314} &       & \multicolumn{1}{c}{0.9936} & \multicolumn{1}{c}{0.8235} & \multicolumn{1}{c}{0.2861} &       & \multicolumn{1}{c}{560} & \multicolumn{1}{c}{6.9774} & \multicolumn{1}{c}{2.3294} & \multicolumn{1}{c}{3.1026} &       & \multicolumn{1}{c}{4.8696} & \multicolumn{1}{c}{3.5646} & \multicolumn{1}{c}{4.0347} \\
    Paraíba & \multicolumn{1}{p{5.39em}}{PB} &       & \multicolumn{1}{c}{3,943,885} & \multicolumn{1}{c}{33.08} & \multicolumn{1}{c}{0.1839} &       & \multicolumn{1}{c}{145} & \multicolumn{1}{c}{436} & \multicolumn{1}{c}{1,400} &       & \multicolumn{1}{c}{0.9752} & \multicolumn{1}{c}{0.7962} & \multicolumn{1}{c}{0.2943} &       & \multicolumn{1}{c}{672} & \multicolumn{1}{c}{6.7942} & \multicolumn{1}{c}{2.7729} & \multicolumn{1}{c}{3.0116} &       & \multicolumn{1}{c}{5.1071} & \multicolumn{1}{c}{4.1112} & \multicolumn{1}{c}{4.1259} \\
    Pernambuco & \multicolumn{1}{p{5.39em}}{PE} &       & \multicolumn{1}{c}{9,277,727} & \multicolumn{1}{c}{33.50} & \multicolumn{1}{c}{0.1894} &       & \multicolumn{1}{c}{140} & \multicolumn{1}{c}{437} & \multicolumn{1}{c}{1,308} &       & \multicolumn{1}{c}{0.9831} & \multicolumn{1}{c}{0.8162} & \multicolumn{1}{c}{0.2627} &       & \multicolumn{1}{c}{2,541} & \multicolumn{1}{c}{7.4292} & \multicolumn{1}{c}{3.4088} & \multicolumn{1}{c}{3.5120} &       & \multicolumn{1}{c}{4.9493} & \multicolumn{1}{c}{4.3138} & \multicolumn{1}{c}{4.3208} \\
    Alagoas & \multicolumn{1}{p{5.39em}}{AL} &       & \multicolumn{1}{c}{3,321,730} & \multicolumn{1}{c}{30.91} & \multicolumn{1}{c}{0.2833} &       & \multicolumn{1}{c}{95} & \multicolumn{1}{c}{348} & \multicolumn{1}{c}{1,005} &       & \multicolumn{1}{c}{0.9706} & \multicolumn{1}{c}{0.7744} & \multicolumn{1}{c}{0.2814} &       & \multicolumn{1}{c}{563} & \multicolumn{1}{c}{6.3830} & \multicolumn{1}{c}{2.9068} & \multicolumn{1}{c}{3.0360} &       & \multicolumn{1}{c}{5.1566} & \multicolumn{1}{c}{4.1516} & \multicolumn{1}{c}{4.2944} \\
    Sergipe & \multicolumn{1}{p{5.39em}}{SE} &       & \multicolumn{1}{c}{2,219,574} & \multicolumn{1}{c}{32.09} & \multicolumn{1}{c}{0.2812} &       & \multicolumn{1}{c}{156} & \multicolumn{1}{c}{431} & \multicolumn{1}{c}{1,200} &       & \multicolumn{1}{c}{0.977} & \multicolumn{1}{c}{0.8352} & \multicolumn{1}{c}{0.3242} &       & \multicolumn{1}{c}{652} & \multicolumn{1}{c}{6.5579} & \multicolumn{1}{c}{1.9785} & \multicolumn{1}{c}{2.4805} &       & \multicolumn{1}{c}{4.8874} & \multicolumn{1}{c}{3.4009} & \multicolumn{1}{c}{3.5932} \\
    Bahia & \multicolumn{1}{p{5.39em}}{BA} &       & \multicolumn{1}{c}{15,126,371} & \multicolumn{1}{c}{33.11} & \multicolumn{1}{c}{0.2488} &       & \multicolumn{1}{c}{139} & \multicolumn{1}{c}{431} & \multicolumn{1}{c}{1,400} &       & \multicolumn{1}{c}{0.985} & \multicolumn{1}{c}{0.8461} & \multicolumn{1}{c}{0.3171} &       & \multicolumn{1}{c}{3,359} & \multicolumn{1}{c}{7.1842} & \multicolumn{1}{c}{2.8426} & \multicolumn{1}{c}{3.0643} &       & \multicolumn{1}{c}{4.9479} & \multicolumn{1}{c}{3.9217} & \multicolumn{1}{c}{4.1399} \\
    \textbf{North-east} &       &       & \multicolumn{1}{c}{\textbf{56,186,190}} & \multicolumn{1}{c}{\textbf{32.62}} & \multicolumn{1}{c}{\textbf{0.2632}} &       & \multicolumn{1}{c}{\textbf{126}} & \multicolumn{1}{c}{\textbf{403}} & \multicolumn{1}{c}{\textbf{1,250}} &       & \multicolumn{1}{c}{\textbf{0.9827}} & \multicolumn{1}{c}{\textbf{0.8315}} & \multicolumn{1}{c}{\textbf{0.2904}} &       & \multicolumn{1}{c}{\textbf{11,857}} & \multicolumn{1}{c}{\textbf{6.8927}} & \multicolumn{1}{c}{\textbf{2.6340}} & \multicolumn{1}{c}{\textbf{2.9858}} &       & \multicolumn{1}{c}{\textbf{4.9778}} & \multicolumn{1}{c}{\textbf{3.8801}} & \multicolumn{1}{c}{\textbf{4.1025}} \\
    Minas Gerais & \multicolumn{1}{p{5.39em}}{MG} &       & \multicolumn{1}{c}{20,734,097} & \multicolumn{1}{c}{34.77} & \multicolumn{1}{c}{0.1544} &       & \multicolumn{1}{c}{236} & \multicolumn{1}{c}{706} & \multicolumn{1}{c}{1,933} &       & \multicolumn{1}{c}{0.9876} & \multicolumn{1}{c}{0.8674} & \multicolumn{1}{c}{0.2785} &       & \multicolumn{1}{c}{4,111} & \multicolumn{1}{c}{7.8235} & \multicolumn{1}{c}{3.4013} & \multicolumn{1}{c}{3.4944} &       & \multicolumn{1}{c}{4.7173} & \multicolumn{1}{c}{3.7202} & \multicolumn{1}{c}{3.8092} \\
    Espírito Santo & \multicolumn{1}{p{5.39em}}{ES} &       & \multicolumn{1}{c}{3,885,049} & \multicolumn{1}{c}{34.25} & \multicolumn{1}{c}{0.1553} &       & \multicolumn{1}{c}{225} & \multicolumn{1}{c}{700} & \multicolumn{1}{c}{2,066} &       & \multicolumn{1}{c}{0.9728} & \multicolumn{1}{c}{0.8133} & \multicolumn{1}{c}{0.324} &       & \multicolumn{1}{c}{772} & \multicolumn{1}{c}{8.3036} & \multicolumn{1}{c}{3.3526} & \multicolumn{1}{c}{3.2428} &       & \multicolumn{1}{c}{4.5309} & \multicolumn{1}{c}{3.6945} & \multicolumn{1}{c}{3.6406} \\
    Rio de Janeiro & \multicolumn{1}{p{5.39em}}{RJ} &       & \multicolumn{1}{c}{16,461,173} & \multicolumn{1}{c}{36.80} & \multicolumn{1}{c}{0.0268} &       & \multicolumn{1}{c}{266} & \multicolumn{1}{c}{750} & \multicolumn{1}{c}{2,566} &       & \multicolumn{1}{c}{0.9899} & \multicolumn{1}{c}{0.8736} & \multicolumn{1}{c}{0.3319} &       & \multicolumn{1}{c}{3,085} & \multicolumn{1}{c}{9.5006} & \multicolumn{1}{c}{6.0272} & \multicolumn{1}{c}{5.4156} &       & \multicolumn{1}{c}{4.3063} & \multicolumn{1}{c}{4.7137} & \multicolumn{1}{c}{4.2991} \\
    São Paulo & \multicolumn{1}{p{5.39em}}{SP} &       & \multicolumn{1}{c}{44,035,304} & \multicolumn{1}{c}{35.29} & \multicolumn{1}{c}{0.0344} &       & \multicolumn{1}{c}{326} & \multicolumn{1}{c}{860} & \multicolumn{1}{c}{2,650} &       & \multicolumn{1}{c}{0.9938} & \multicolumn{1}{c}{0.8645} & \multicolumn{1}{c}{0.2907} &       & \multicolumn{1}{c}{4,339} & \multicolumn{1}{c}{9.9718} & \multicolumn{1}{c}{5.1388} & \multicolumn{1}{c}{4.7980} &       & \multicolumn{1}{c}{4.3038} & \multicolumn{1}{c}{4.4431} & \multicolumn{1}{c}{4.3142} \\
    \textbf{South-east} &       &       & \multicolumn{1}{c}{\textbf{85,115,623}} & \multicolumn{1}{c}{\textbf{35.41}} & \multicolumn{1}{c}{\textbf{0.0676}} &       & \multicolumn{1}{c}{\textbf{277}} & \multicolumn{1}{c}{\textbf{766}} & \multicolumn{1}{c}{\textbf{2,405}} &       & \multicolumn{1}{c}{\textbf{0.9905}} & \multicolumn{1}{c}{\textbf{0.8645}} & \multicolumn{1}{c}{\textbf{0.2968}} &       & \multicolumn{1}{c}{\textbf{12,307}} & \multicolumn{1}{c}{\textbf{9.2358}} & \multicolumn{1}{c}{\textbf{4.7551}} & \multicolumn{1}{c}{\textbf{4.4993}} &       & \multicolumn{1}{c}{\textbf{4.5212}} & \multicolumn{1}{c}{\textbf{4.3904}} & \multicolumn{1}{c}{\textbf{4.2187}} \\
    Paraná & \multicolumn{1}{p{5.39em}}{PR} &       & \multicolumn{1}{c}{11,081,692} & \multicolumn{1}{c}{34.52} & \multicolumn{1}{c}{0.1252} &       & \multicolumn{1}{c}{315} & \multicolumn{1}{c}{817} & \multicolumn{1}{c}{2,325} &       & \multicolumn{1}{c}{0.9881} & \multicolumn{1}{c}{0.8231} & \multicolumn{1}{c}{0.2923} &       & \multicolumn{1}{c}{2,301} & \multicolumn{1}{c}{8.3716} & \multicolumn{1}{c}{3.8072} & \multicolumn{1}{c}{3.6635} &       & \multicolumn{1}{c}{4.6968} & \multicolumn{1}{c}{3.8634} & \multicolumn{1}{c}{3.9054} \\
    Santa Catarina & \multicolumn{1}{p{5.39em}}{SC} &       & \multicolumn{1}{c}{6,727,148} & \multicolumn{1}{c}{35.59} & \multicolumn{1}{c}{0.1589} &       & \multicolumn{1}{c}{390} & \multicolumn{1}{c}{1000} & \multicolumn{1}{c}{2,500} &       & \multicolumn{1}{c}{0.9914} & \multicolumn{1}{c}{0.8207} & \multicolumn{1}{c}{0.2948} &       & \multicolumn{1}{c}{1,117} & \multicolumn{1}{c}{8.7443} & \multicolumn{1}{c}{4.3552} & \multicolumn{1}{c}{4.1688} &       & \multicolumn{1}{c}{4.6119} & \multicolumn{1}{c}{3.7287} & \multicolumn{1}{c}{3.6662} \\
    Rio Grande do Sul & \multicolumn{1}{p{5.39em}}{RS} &       & \multicolumn{1}{c}{11,207,274} & \multicolumn{1}{c}{36.90} & \multicolumn{1}{c}{0.1501} &       & \multicolumn{1}{c}{300} & \multicolumn{1}{c}{850} & \multicolumn{1}{c}{2,500} &       & \multicolumn{1}{c}{0.9869} & \multicolumn{1}{c}{0.837} & \multicolumn{1}{c}{0.3084} &       & \multicolumn{1}{c}{3,702} & \multicolumn{1}{c}{8.4965} & \multicolumn{1}{c}{4.0837} & \multicolumn{1}{c}{3.9188} &       & \multicolumn{1}{c}{4.4833} & \multicolumn{1}{c}{4.0926} & \multicolumn{1}{c}{3.9434} \\
    \textbf{South} &       &       & \multicolumn{1}{c}{\textbf{29,016,114}} & \multicolumn{1}{c}{\textbf{35.69}} & \multicolumn{1}{c}{\textbf{0.1426}} &       & \multicolumn{1}{c}{\textbf{320}} & \multicolumn{1}{c}{\textbf{870}} & \multicolumn{1}{c}{\textbf{2,444}} &       & \multicolumn{1}{c}{\textbf{0.9884}} & \multicolumn{1}{c}{\textbf{0.8278}} & \multicolumn{1}{c}{\textbf{0.2989}} &       & \multicolumn{1}{c}{\textbf{7,120}} & \multicolumn{1}{c}{\textbf{8.5022}} & \multicolumn{1}{c}{\textbf{4.0360}} & \multicolumn{1}{c}{\textbf{3.8726}} &       & \multicolumn{1}{c}{\textbf{4.5912}} & \multicolumn{1}{c}{\textbf{3.9347}} & \multicolumn{1}{c}{\textbf{3.8773}} \\
    Mato Grosso do Sul & \multicolumn{1}{p{5.39em}}{MS} &       & \multicolumn{1}{c}{2,619,657} & \multicolumn{1}{c}{33.19} & \multicolumn{1}{c}{0.1078} &       & \multicolumn{1}{c}{300} & \multicolumn{1}{c}{766} & \multicolumn{1}{c}{2,232} &       & \multicolumn{1}{c}{0.9849} & \multicolumn{1}{c}{0.7832} & \multicolumn{1}{c}{0.2918} &       & \multicolumn{1}{c}{615} & \multicolumn{1}{c}{8.5436} & \multicolumn{1}{c}{4.1849} & \multicolumn{1}{c}{4.0337} &       & \multicolumn{1}{c}{4.7883} & \multicolumn{1}{c}{4.5221} & \multicolumn{1}{c}{4.5129} \\
    Mato Grosso & \multicolumn{1}{p{5.39em}}{MT} &       & \multicolumn{1}{c}{3,224,357} & \multicolumn{1}{c}{32.12} & \multicolumn{1}{c}{0.1720} &       & \multicolumn{1}{c}{285} & \multicolumn{1}{c}{733} & \multicolumn{1}{c}{2,000} &       & \multicolumn{1}{c}{0.9911} & \multicolumn{1}{c}{0.7919} & \multicolumn{1}{c}{0.284} &       & \multicolumn{1}{c}{627} & \multicolumn{1}{c}{8.7704} & \multicolumn{1}{c}{4.2925} & \multicolumn{1}{c}{4.7669} &       & \multicolumn{1}{c}{4.8419} & \multicolumn{1}{c}{4.6268} & \multicolumn{1}{c}{4.7849} \\
    Goiás & \multicolumn{1}{p{5.39em}}{GO} &       & \multicolumn{1}{c}{6,523,222} & \multicolumn{1}{c}{33.35} & \multicolumn{1}{c}{0.0776} &       & \multicolumn{1}{c}{268} & \multicolumn{1}{c}{724} & \multicolumn{1}{c}{1,912} &       & \multicolumn{1}{c}{0.9923} & \multicolumn{1}{c}{0.8126} & \multicolumn{1}{c}{0.3153} &       & \multicolumn{1}{c}{1,361} & \multicolumn{1}{c}{8.0773} & \multicolumn{1}{c}{3.0697} & \multicolumn{1}{c}{3.4046} &       & \multicolumn{1}{c}{4.7121} & \multicolumn{1}{c}{3.7236} & \multicolumn{1}{c}{4.0061} \\
    Distrito Federal & \multicolumn{1}{p{5.39em}}{DF} &       & \multicolumn{1}{c}{2,852,372} & \multicolumn{1}{c}{32.69} & \multicolumn{1}{c}{0.0442} &       & \multicolumn{1}{c}{301} & \multicolumn{1}{c}{1000} & \multicolumn{1}{c}{5,000} &       & \multicolumn{1}{c}{0.9931} & \multicolumn{1}{c}{0.8955} & \multicolumn{1}{c}{0.414} &       & \multicolumn{1}{c}{408} & \multicolumn{1}{c}{11.7179} & \multicolumn{1}{c}{7.1630} & \multicolumn{1}{c}{7.2125} &       & \multicolumn{1}{c}{3.8968} & \multicolumn{1}{c}{5.2292} & \multicolumn{1}{c}{5.3176} \\
    \textbf{West Central} &       &       & \multicolumn{1}{c}{\textbf{15,219,608}} & \multicolumn{1}{c}{\textbf{32.94}} & \multicolumn{1}{c}{\textbf{0.0965}} &       & \multicolumn{1}{c}{\textbf{285}} & \multicolumn{1}{c}{\textbf{750}} & \multicolumn{1}{c}{\textbf{2,500}} &       & \multicolumn{1}{c}{\textbf{0.9909}} & \multicolumn{1}{c}{\textbf{0.8193}} & \multicolumn{1}{c}{\textbf{0.3237}} &       & \multicolumn{1}{c}{\textbf{3,011}} & \multicolumn{1}{c}{\textbf{8.7258}} & \multicolumn{1}{c}{\textbf{4.0041}} & \multicolumn{1}{c}{\textbf{4.2698}} &       & \multicolumn{1}{c}{\textbf{4.7976}} & \multicolumn{1}{c}{\textbf{4.4461}} & \multicolumn{1}{c}{\textbf{4.6084}} \\
    \textbf{Brazil} &       &       & \multicolumn{1}{c}{\textbf{202,768,562}} & \multicolumn{1}{c}{\textbf{33.55}} & \multicolumn{1}{c}{\textbf{0.1494}} &       & \multicolumn{1}{c}{\textbf{200}} & \multicolumn{1}{c}{\textbf{662}} & \multicolumn{1}{c}{\textbf{2,000}} &       & \multicolumn{1}{c}{\textbf{0.987}} & \multicolumn{1}{c}{\textbf{0.8427}} & \multicolumn{1}{c}{\textbf{0.3002}} &       & \multicolumn{1}{c}{\textbf{39,237}} & \multicolumn{1}{c}{\textbf{8.3237}} & \multicolumn{1}{c}{\textbf{3.9598}} & \multicolumn{1}{c}{\textbf{3.9380}} &       & \multicolumn{1}{c}{\textbf{4.7947}} & \multicolumn{1}{c}{\textbf{4.2511}} & \multicolumn{1}{c}{\textbf{4.1976}} \\
    \midrule
   \multicolumn{23}{p{105.83em}}{Notes: Column 1 refers to the IBGE estimation based on the PNAD-2014 data. Columns 2 to 9 are the author’s own estimates based on all the observations from PNAD-2014. The values from columns 10 to 16 have been determined on the basis of the PNAD-2014 mobility supplement. The income distribution is based on monthly per capita domiciliary income. Bottom, middle, and top represent, the poorest $10\%$, the middle $50\%$ and the richest $10\%$, respectively, of the income distribution.} \\
    \end{tabular}%
    \end{adjustbox}
\end{table}

\end{sidewaystable}
\restoregeometry


\newpage
%\section{Reference}
\bibliography{ReferenceGeo}
\bibliographystyle{apa}

\end{document}
