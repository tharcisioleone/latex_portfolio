\documentclass{beamer} 
\usetheme{Frankfurt}
%\usecolortheme{beaver} % or try albatross, beaver, crane, ...
\usepackage{hyperref} 
\usepackage{amsmath, amsfonts, amssymb}
\usepackage{amsmath, amsfonts, amssymb}
\usepackage{float}
\usepackage{xcolor}
%\usepackage{graphicx}
\usepackage{adjustbox}
\usepackage{indentfirst}
\usepackage{booktabs}
\usepackage{tabu}
\usepackage{scalefnt}
\usepackage{pdflscape}
\usepackage{draftwatermark}
\usepackage{rotating}
\usepackage{caption}
\usepackage{subcaption}
%\usepackage{natbib} [options]
\usepackage{fixmath}
\usepackage[utf8]{inputenc}
\usepackage{xparse}
\usepackage{varwidth}
\usepackage{breqn}
\usepackage{mathtools}
\usepackage{pdflscape,array,booktabs}
\usepackage{wrapfig}
\usepackage{attachfile}
\usepackage{booktabs,multirow}
%\usepackage{color}
\usepackage{appendixnumberbeamer}
%\usepackage[printwatermark]{xwatermark}
\usepackage{tikz}
\usepackage{lipsum}
\usepackage{appendixnumberbeamer}
\usepackage{mathtools}
\usepackage{verbatim}
\usepackage{dcolumn}
\usepackage{booktabs,multirow}
\usepackage{wasysym}
\usepackage{marvosym}
\setbeamercolor{subsection in toc}{fg=blue!80}
\usepackage{appendixnumberbeamer}
\usepackage[T1]{fontenc}
\usepackage{lmodern}
\usepackage{stmaryrd}
\usepackage{listings}
%\usepackage{bm}

\makeatletter
\newcommand{\setnextsection}[1]{%
  \setcounter{section}{\numexpr#1-1\relax}%
  \beamer@tocsectionnumber=\numexpr#1-1\relax\space}
\makeatother

\defbeamertemplate{subsection in toc}{bullets}{%
  \leavevmode
  \parbox[t]{1em}{\textbullet\hfill}%
  \parbox[t]{\dimexpr\textwidth-1em\relax}{\inserttocsubsection}\par}
\defbeamertemplate{section in toc}{sections numbered roman}{%
  \leavevmode%
  \MakeUppercase{\romannumeral\inserttocsectionnumber}.\ %
  \inserttocsection\par}
  
  \definecolor{codegreen}{rgb}{0,0.6,0}
\definecolor{codegray}{rgb}{0.5,0.5,0.5}
\definecolor{codepurple}{rgb}{0.58,0,0.82}
\definecolor{backcolour}{rgb}{0.95,0.95,0.92}
 
\lstdefinestyle{mystyle}{
    backgroundcolor=\color{backcolour},   
    commentstyle=\color{codegreen},
    %keywordstyle=\color{magenta},
    numberstyle=\tiny\color{codegray},
    %stringstyle=\color{codepurple},
    basicstyle=\ttfamily\footnotesize,
    breakatwhitespace=false,         
    breaklines=true,                 
    captionpos=b,                    
    keepspaces=true,                 
    numbers=left,                    
    numbersep=5pt,                  
    showspaces=false,                
    showstringspaces=false,
    showtabs=false,                  
    tabsize=2
}
 
\lstset{style=mystyle}









\setbeamersize{text margin left=5mm,text margin right=5mm} 


\usepackage{tikz}
\usetikzlibrary{shapes}
\newcommand*{\circled}[2][red]{
  \tikz[baseline=(char.base)]{
              \node[shape=ellipse,inner sep=2pt,
                draw=#1,
             ] (char) {#2};}
}
\newcommand{\markcells}[3][green]{
  \tikz[baseline=(char.mid)]{\node[shape=ellipse,overlay,draw,#1]{\phantom{\rule{#2}{#3}}};}%
}

\newcommand{\diff}{\mathop{}\!\mathrm{d}}
\newcommand{\Rho}{\mathrm{P}}
\newcommand{\Chi}{\mathrm{X}}




\setbeamertemplate{button}{\tikz
  \node[
  inner xsep=5pt,
  draw=structure!20,
  fill=structure!10,
  rounded corners=4pt]  {\scriptsize\insertbuttontext};}


\long\def\comment#1{}
\comment{

\newsavebox\mybox
\savebox\mybox{\tikz[color=red,opacity=0.5]\node{Do not circulate};}
\newwatermark*[
  allpages,
  angle=45,
  scale=4.5,
  xpos=-20,
  ypos=15
]{\usebox\mybox}

// Finalising ignore
}

%\captionsetup[table]{position=top,skip=7pt} 
%\usepackage[singlelinecheck=false]{caption}
%\usefonttheme{serif}  % or try serif, structurebold, ...
%\setbeamertemplate{navigation symbols}{}
%\setbeamertemplate{caption}[numbered]
%\usetikzlibrary{arrows,shapes}



\titlegraphic{ \vspace{-3.2cm}\includegraphics[width=5.7cm]{Figure/giga}\hspace*{0.5cm}~%
   \includegraphics[width=5.7cm]{Figure/logo}
}


\title[Does productivity bonus pay off?]{Does productivity bonus pay off?}
\subtitle{\footnotesize The effects of teacher incentive pay on student achievement in Brazilian schools}



\author [Tharcisio Leone]{ \\ \textbf {Tharcisio Leone} \\ \footnotesize PhD Candidate in Economics}

 
\institute[GIGA/FU-Berlin] 
{ \\ Dissertation Defense \\ Freie Universität Berlin
    } 

%German Institute of Global and Area Studies \\ Free University of Berlin

\date[\\ April 21, 2021]{\footnotesize April 21, 2021}





\AtBeginSection[]
{
  \begin{frame}<beamer>
    \frametitle{Outline}
    \tableofcontents[currentsection,currentsubsection]
  \end{frame}
}






\begin{document}

\begin{frame}
  \titlepage
\end{frame}



\begin{frame} {Tharcisio Leone (Ph.D Candidate)}
\begin{block}{\centering Academic Career and Education}
\begin{itemize}
\item \footnotesize Research Fellow / Doctoral Student \hspace{1,8cm} GIGA Hamburg
\item \footnotesize Ph.D in Economics \hspace{4.1cm} Free University of Berlin
\item \footnotesize M.A. in Public Economics \hspace{3,15cm} Free University of Berlin
\item \footnotesize B.A. in Economics and Management\hspace{1.79cm} University of Hanover
\item \footnotesize B.A. in International Trade	\hspace{3cm} UNIVEM (Brazil)
\end{itemize}
\end{block}


\begin{block}{\centering Working Papers}
\begin{enumerate}
\item \footnotesize The gender gap in intergenerational education mobility in Brazil.
\item \footnotesize The Geography of Intergenerational Mobility: Evidence of educational persistence and the ``Great Gatsby Curve" in Brazil.
\item \footnotesize Intergenerational Mobility in Education: Estimates of the Worldwide Variation.
\item \footnotesize \textcolor{red}{Does productivity bonus pay off? The effects of teacher incentive pay on student achievement in Brazilian schools.}
\end{enumerate}
\end{block}
\end{frame}



\long\def\comment#1{}
\comment{
\begin{frame}{Research Design}
\vspace{1cm}
\centering \textbf{Figure 1: Visual Diagram for the Mixed Method Research}
\vspace{-10pt}
\begin{figure}[htb]
\centering
\includegraphics[width=0.9\textwidth,height=1\textheight]{Figure/Diagram}
\end{figure}
     \vspace{-5.64cm}  \hspace{17pt}
\begin{minipage}{1\textwidth} 
{{\fontsize{4}{4}\selectfont  
\Tiny Source: Own creation based on Creswell and Clark (2011).\par}}
\end{minipage} 
\end{frame}
// Finalising ignore
}


\begin{frame}{Abstract}
\begin{block}{\centering Does productivity bonus pay off?}
\begin{itemize}
\item \footnotesize Quasi-Experimental Design.
\\ \tiny \textcolor{red}{$\Longrightarrow$} Ex-post impact evaluation.
\\ \tiny \textcolor{red}{$\Longrightarrow$} Difference in difference approach.
\item \footnotesize Empirical Dynamic Models
\\ \tiny\textcolor{red}{$\Longrightarrow$} Longitudinal data of academic achievements.
\item \footnotesize Teacher Bonus has lead to an increase in the performance by Mathematics.
\\ \tiny\textcolor{red}{$\Longrightarrow$} No significant impact was found by Portuguese (reading).
\end{itemize}
\end{block}
\end{frame}

 

% Uncomment these lines for an automatically generated outline.
\begin{frame}[label=Outline]{Outline}
\vspace{-37pt} \flushright \hyperlink{Mirror}{\beamerbutton{\textcolor{red}{Appendix}}}
\vspace{15pt}
\begin{flushleft}
  \tableofcontents
\end{flushleft}
\end{frame}



\section{Motivation}

\begin{frame}{Motivation}
\begin{figure}[htb]
\vspace{-8pt}
\centering
\includegraphics[scale=0.38]{Figure/PISA}
\end{figure}
     \vspace{-13pt}  \hspace{15pt}
\begin{minipage}{1\textwidth} 
{{\fontsize{4}{4}\selectfont  
Source: Organization for Economic Cooperation and Development (OECD), 2018-2019.\par}}
\end{minipage} 
\end{frame}



\begin{frame}{Motivation}

\begin{block}{\centering Teacher Bonus Programs}
 \begin{itemize}
\item \scriptsize Productivity bonus as a mechanism to improve the teacher engagement in the schools.
\item \scriptsize Empirical evidence for the impact of teacher bonus on student performance is mixed.
\\ \tiny\textcolor{red}{$\Longrightarrow$} \textbf{Positive impact}: Loyalka et al. (2019) for China, Britton and Propper (2016) for England, Dee and Wyckoff (2015) and Imberman and Lovenheim (2015) for the United States, Duflo et al. (2012) and Muralidharan and Sundararaman (2011) for India, and Lavy (2009) for Israel.
\\ \tiny\textcolor{red}{$\Longrightarrow$} \textbf{No impact}: Barrera-Osorio and Raju (2017) for Pakistan, Behrman et al. (2015) for Mexico, Yuan et al. (2013) and Springer et al. (2012) for the United States, and Glewwe et al. (2010) for Kenya.
\item \scriptsize No empirical evidence for Brazil.
\item \scriptsize Currently, almost half of the 27 Brazilian states have any teacher bonus program.
\end{itemize}
\end{block}

\begin{figure}[htb]
%\vspace{10pt}
\centering
\includegraphics[scale=0.4]{Figure/TimeTB}
\end{figure}
     \vspace{-5.3cm}  \hspace{8pt}
\begin{minipage}{1\textwidth} 
{{\fontsize{4}{4}\selectfont  
Source: Own elaboration based on Scorzafave et al. (2016).\par}}
\end{minipage} 
\end{frame}



\begin{frame} {Motivation}
\begin{block}{\centering Research Question}
  		\begin{itemize}
			\item Has the teacher bonus program, implemented in the Brazilian state of São Paulo, led to an improvement of student performance?
			\end{itemize}
\end{block}


\vspace{1cm}

\begin{block}{\centering Contributions to the Literature}
\begin{enumerate}
\item \footnotesize Estimating the impacts of teacher bonus on the student performance.
\\ \tiny\textcolor{red}{$\Longrightarrow$} For Brazil, no empirical evidence has still been published in international peer-reviewed journals.
\\ \tiny\textcolor{red}{$\Longrightarrow$} Application of a quasi-experimental design.
\hfill \break 
\item \footnotesize Empirical evidence of the education production function using dynamic models.
\\ \tiny\textcolor{red}{$\Longrightarrow$} Use of longitudinal data of academic achievements.
\\ \tiny\textcolor{red}{$\Longrightarrow$} Controlling for individual ability of students.	
\hfill \break 
\item \footnotesize Integration of lagged achievements into the research of teacher bonus program.
\\ \tiny\textcolor{red}{$\Longrightarrow$} Going beyond the (obvious) fixed effect models.
\\ \tiny\textcolor{red}{$\Longrightarrow$} Use of instrument variables in order to counterbalance the biased effects of the autocorrelation in error terms.	
\end{enumerate}
\end{block}
\end{frame}




\section{Institutional Setting}

\begin{frame}[label=Main1]{Institutional Setting}
\begin{block}{\centering \Large Brazilian Educational System}
\begin{itemize}
\item \normalsize Legal Framework
\\ \tiny\textcolor{red}{$\Longrightarrow$} Education is a right for all and an obligation of the government.
\\ \tiny\textcolor{red}{$\Longrightarrow$} Existence of public and private schools in all levels of education.
\\ \tiny\textcolor{red}{$\Longrightarrow$} Public schools can be municipal, state or federal one.
\\ \tiny\textcolor{red}{$\Longrightarrow$} Principle of independence in administrative law.

\item \normalsize Division of Competences
\\ \tiny\textcolor{red}{$\Longrightarrow$} States and municipalities are primarily responsible for the functioning of public schools.
%\\ \tiny\textcolor{red}{$\Longrightarrow$} Federal government: (a) Legislation of guidelines and bases for national education; (b) Manage of federal institutions; (c) Provision of technical and financial aid to ensure a minimum of quality standard.
\\ \tiny\textcolor{red}{$\Longrightarrow$} Municipal Responsibilities: Early childhood education, pre-primary and primary schooling.
\\ \tiny\textcolor{red}{$\Longrightarrow$} State Responsibilities: Secondary education.
\\ \tiny\textcolor{red}{$\Longrightarrow$} But states need to ensure universality of primary education (when municipalities are not able to guarantee that).
\\ \tiny\textcolor{red}{$\Longrightarrow$} Common existence of municipal, state and private schools in the same city.
\end{itemize}
\end{block}


\begin{block}{\centering \Large Intervention}
\vspace{-18pt} \flushright \hyperlink{IDESP}{\beamerbutton{\textcolor{red}{IDESP}}} \vspace{-18pt} \hspace{5pt} \hyperlink{ICM}{\beamerbutton{\textcolor{red}{ICM}}}
\vspace{15pt}
\begin{itemize}
\item \normalsize Implementation of Teacher Bonus Program
\\ \tiny\textcolor{red}{$\Longrightarrow$} Locality: State of São Paulo.
\\ \tiny\textcolor{red}{$\Longrightarrow$} Grantees: All employees of the São Paulo Sate Secretary of Education (SEE/SP).
\\ \tiny\textcolor{red}{$\Longrightarrow$} Scope: School-based Bonus.
\\ \tiny\textcolor{red}{$\Longrightarrow$} Goal: Improve the educational performance of students enrolled in state schools.
\\ \tiny\textcolor{red}{$\Longrightarrow$} Accounting based on the achievements of the Education Development Index of the State of São Paulo (IDESP), which are annual targets of quality improvement calculated individually for each school and educational level.
\\ \tiny\textcolor{red}{$\Longrightarrow$} IDESP is an synthetic indicator based on the indicator of performance and average passing rate. 
\\ \tiny\textcolor{red}{$\Longrightarrow$} Payment: Proportional to achievement of IDESP (see ICM). Once a year. Maximum of $20\%$ of the annual salary.
\end{itemize}
\end{block}
\end{frame}



\begin{frame}{Institutional Setting}
\vspace{0.3cm}
\begin{block}{\centering \Large Time-Frame of Bonus Implementation}
\begin{itemize}
\item \normalsize School Quality Program 
\\ \tiny\textcolor{red}{$\Longrightarrow$} Teacher bonus is part of the School Quality Program, introduced by the Law Nº $1,017$ of 15 October 2007.
\\ \tiny\textcolor{red}{$\Longrightarrow$} It has set long-term goals for the improvement of the quality of schools, which must be reached by 2030.
\\ \tiny\textcolor{red}{$\Longrightarrow$} It created also short-term targets (IDESP) to monitor annually the development of the education system.
\\ \tiny\textcolor{red}{$\Longrightarrow$} The IDESP for the current year will be established based on the IDESP reached in the previous year.
\\ \tiny\textcolor{red}{$\Longrightarrow$} 2008 was the first evaluated year (referred to the relative improvement compared to 2007). 
\end{itemize}
\end{block}

\begin{figure}[htb]
%\vspace{-1.2cm}
\centering
\includegraphics[width=1.02\textwidth,height=0.8\textheight]{Figure/Bonus_Implementation}
\end{figure}
     \vspace{-3.67cm}  \hspace{2pt}
\begin{minipage}{1\textwidth} 
{{\fontsize{4}{4}\selectfont  
Notes: The academic year in Brazil aligns with the calendar year, lasting from February to November. The \href{http://tiny.cc/4opsaz}{Law $N^{o}$ 1,017} of 15 October 2007 has introduced the teacher incentive bonus in the state of São Paulo, and the \href{http://tiny.cc/ympsaz}{Decree $N^{o}$ 52,719} of 14 February 2008 and the \href{http://tiny.cc/0wpsaz}{Law $N^{o}$ 1,078} of 17 December 2008 have supplemented it. \\
Source: Own elaboration based on the Official Gazette of the State of São Paulo.\par}}
\end{minipage} 
\end{frame}



\section{Data}


\begin{frame}[label=Main2]{Data}
\begin{block}{\centering \Large GERES Database}
\vspace{-18pt} \flushright \hyperlink{GERESSample}{\beamerbutton{\textcolor{red}{Sample}}}
\begin{itemize}
\item \scriptsize Longitudinal Study
of Quality and Equity in Brazilian Elementary Education.
\item \scriptsize First educational panel data concluded with success in Brazil.
\item \scriptsize Belo Horizonte, Campinas, Campo Grande, Rio de Janeiro and Salvador.
\item \scriptsize Following $21,529$ students from $309$ schools in the first four years of primary education.
\item \scriptsize Portuguese and mathematics standardized proficiency tests.
\\ \tiny\textcolor{red}{$\Longrightarrow$} Questionnaires to socio-economic status of the family (education, income, professional occupation).
\\ \tiny\textcolor{red}{$\Longrightarrow$} Teacher and school principal questionnaires (professional qualifications, working time, teaching methods, personal behaviour, and expectation to student performance).
\\ \tiny\textcolor{red}{$\Longrightarrow$} School questionnaire (infrastructure, selection of staffs, integration in community, and professional culture).
\end{itemize}
\end{block} 


\begin{figure}[htb]
\vspace{-0.5cm}
\centering
\includegraphics[width=1.02\textwidth,height=0.8\textheight]{Figure/GERES}
\end{figure}
     \vspace{-4.72cm}  \hspace{-1pt}
\begin{minipage}{1\textwidth} 
{{\fontsize{4}{4}\selectfont  
Notes: The academic year in Brazil aligns with the calendar year, lasting from February to November. \\
Source: Own elaboration based on the GERES database.\par}}
\end{minipage} 
\end{frame}



\begin{frame} {Figure 2: Student Performance in Mathematics}
%\vspace{-0.5cm}
\begin{figure}[H]
\vspace{5pt}
\centering
%\captionsetup{justification=centering,margin=1.5cm}
%\caption{Student Performance in Mathematics}
\vspace{-8pt}
      \includegraphics[width=0.85\textwidth]{Figure/MathScores-AllCampinas}
       \vspace{-1pt}
     \label{fig:Math-RegStud}
     \begin{minipage}{0.86\textwidth} % choose width suitably
{{\fontsize{4}{4}\selectfont 
Notes: Data are limited to the city of Campinas. The proficiency (test scores) by GERES was estimated using the Item Response Theory. The reported values refer to the average test scores calculated by type of schools. The Law $N^{o}$ 1,017 of 15 October 2007 has introduced the teacher incentive system in the state of São Paulo, and its first year of implementation was 2008. The academic year in Brazil aligns with the calendar year, lasting from February to November. The dashed red lines illustrate the school academic years.  \\
Source: GERES database (2005-2008), own estimates.\par}}
\end{minipage}
\end{figure} 
\end{frame}



%\long\def\comment#1{}
%\comment{
\begin{frame} {Figure 3: Student Performance in Portuguese}
%\vspace{-0.5cm}
\begin{figure}[H]
\vspace{5pt}
\centering
%\captionsetup{justification=centering,margin=1.5cm}
%\caption{Student Performance in Mathematics}
\vspace{-8pt}
      \includegraphics[width=0.85\textwidth]{Figure/PorScores-AllCampinas}
       \vspace{-1pt}
     \label{fig:Math-RegStud}
     \begin{minipage}{0.86\textwidth} % choose width suitably
{{\fontsize{4}{4}\selectfont 
Notes: Data are limited to the city of Campinas. The proficiency (test scores) by GERES was estimated using the Item Response Theory. The reported values refer to the average test scores calculated by type of schools. The Law $N^{o}$ 1,017 of 15 October 2007 has introduced the teacher incentive system in the state of São Paulo, and its first year of implementation was 2008. The academic year in Brazil aligns with the calendar year, lasting from February to November. The dashed red lines illustrate the school academic years.  \\
Source: GERES database (2005-2008), own estimates.\par}}
\end{minipage}
\end{figure} 
\end{frame} 
%// Finalising ignore
%}



\begin{frame}{Table A1: Descriptive Statistics}
\begin{adjustbox}{max width=\textwidth} 
\begin{tabular}{l*{3}{cc}}
\toprule
                    &\multicolumn{2}{c}{\textbf{\emph{State}}}&\multicolumn{2}{c}{\textbf{\emph{Municipal}}}&\multicolumn{2}{c}{\textbf{\emph{Private}}}\\
                    &\multicolumn{1}{c}{{Mean}}&\multicolumn{1}{l}{{Std.Dev.}}&\multicolumn{1}{c}{{Mean}}&\multicolumn{1}{l}{{Std.Dev.}}&\multicolumn{1}{c}{{Mean}}&\multicolumn{1}{l}{{Std.Dev.}} \vspace{-3pt}\\
\midrule
\textbf{\emph{Student}}&            &            &            &            &            &           \vspace{-3pt} \\
Male                &      0.5145&        0.50&      0.5464&        0.50&      0.5436&        0.50\\
Enjoy Learning      &      0.6731&        0.47&      0.6947&        0.46&      0.3769&        0.48\\
Make Homework       &      0.6593&        0.47&      0.6895&        0.46&      0.7338&        0.44\\
\emph{Household Income}&            &            &            &            &            &            \\
Very Low            &      0.0866&        0.28&      0.0934&        0.29&      0.0020&        0.04\\
Low                 &      0.3311&        0.47&      0.3394&        0.47&      0.0275&        0.16\\
Medium              &      0.3428&        0.47&      0.3552&        0.48&      0.1343&        0.34\\
High                &      0.2060&        0.40&      0.1904&        0.39&      0.4098&        0.49\\
Very High           &      0.0335&        0.18&      0.0216&        0.15&      0.4265&        0.49\\
\emph{Education Mother}&            &            &            &            &            &            \\
Less then 4 years   &      0.1353&        0.34&      0.1867&        0.39&      0.0033&        0.06\\
4 years             &      0.2956&        0.46&      0.3335&        0.47&      0.0321&        0.18\\
8 years             &      0.2515&        0.43&      0.2373&        0.43&      0.0819&        0.27\\
Secondary           &      0.2691&        0.44&      0.2222&        0.42&      0.4137&        0.49\\
Tertiary            &      0.0485&        0.21&      0.0204&        0.14&      0.4690&        0.50 \vspace{-4pt}\\
\bottomrule
\vspace{-18pt} \\
\multicolumn{7}{l}{\tiny Note: Continued on next slide.}\\
\end{tabular}
\end{adjustbox}
\end{frame} 



\begin{frame}{Table A1: Descriptive Statistics}
\vspace{-3pt} 
\begin{adjustbox}{max width=\textwidth}
\vspace{-10pt} 
\begin{tabular}{l*{3}{cc}}
\toprule
                    &\multicolumn{2}{c}{\textbf{\emph{State}}}&\multicolumn{2}{c}{\textbf{\emph{Municipal}}}&\multicolumn{2}{c}{\textbf{\emph{Private}}}\\
                    &\multicolumn{1}{c}{{Mean}}&\multicolumn{1}{l}{{Std.Dev.}}&\multicolumn{1}{c}{{Mean}}&\multicolumn{1}{l}{{Std.Dev.}}&\multicolumn{1}{c}{{Mean}}&\multicolumn{1}{l}{{Std.Dev.}} \vspace{-5pt} \\
 \midrule
\textbf{\emph{Teacher Characteristics}}&            &            &            &            &            &           \vspace{-3pt} \\ 
Male                &      0.0183&        0.13&      0.0276&        0.16&      0.0064&        0.08\\
\emph{Education Level}&            &            &            &            &            &         \vspace{-3pt}   \\
Less than secondary &      0.0001&        0.01&      0.0067&        0.08&      0.0000&        0.00\\
Secondary           &      0.0039&        0.06&      0.0002&        0.01&      0.0000&        0.00\\
Vocational          &      0.1819&        0.39&      0.1238&        0.33&      0.0451&        0.21\\
Tertiary            &      0.8077&        0.39&      0.8632&        0.34&      0.9132&        0.28\\
Master              &      0.0064&        0.08&      0.0061&        0.08&      0.0379&        0.19\\
Doctorate           &      0.0000&        0.00&      0.0000&        0.00&      0.0039&        0.06\\
\textbf{\emph{School Characteristics}}&            &            &            &            &            &          \vspace{-3pt}  \\
Library             &      0.8913&        0.31&      0.7322&        0.44&      0.9640&        0.19\\
Computer lab        &      0.2262&        0.42&      0.8540&        0.35&      0.8894&        0.31\\
Science lab         &      0.0337&        0.18&      0.0024&        0.05&      0.7755&        0.42\\
Sports court        &      0.8915&        0.31&      0.7658&        0.42&      0.9135&        0.28\\
Art room            &      0.1398&        0.35&      0.0455&        0.21&      0.4520&        0.50\\
Violence against students&      0.3447&        0.48&      0.3750&        0.48&      0.0955&        0.29\\
Violence against employees &      0.2370&        0.43&      0.2693&        0.44&      0.0450&        0.21 \\
Depredation         &      0.4158&        0.49&      0.5218&        0.50&      0.1271&        0.33\\
Drug use            &      0.3930&        0.49&      0.5701&        0.50&      0.0791&        0.27 \vspace{-4pt} \\
\bottomrule
\vspace{-18pt} \\
\multicolumn{7}{l}{\tiny Notes: Data are limited to the city of Campinas. Panel structure for the five sample waves. For the sake of conciseness, some variables were omitted.}\vspace{-8pt} \\
\multicolumn{7}{l}{\tiny Source: GERES database (2005-2008).}\\
\end{tabular}
\end{adjustbox}
\end{frame} 



\section{Study Design}

\begin{frame} {Study Design}
\vspace{-0.22cm}
\begin{block}{\centering \Large Quasi-Experimental Design}
\begin{itemize}
\item \scriptsize Ex-post impact evaluation.
\item \scriptsize Limitation to Campinas.
\item \scriptsize Normalization of test scores to an unit variance with mean 0 and standard deviation 1.
\item \scriptsize Difference in difference approach (DiD).
\\ \tiny\textcolor{red}{$\Longrightarrow$} Treatment group: $2,158$ students enrolled in $20$ state schools.
\\ \tiny\textcolor{red}{$\Longrightarrow$} Control group: $1,919$ students enrolled in $20$ municipal schools.
\\ \tiny\textcolor{red}{$\Longrightarrow$} Post-treatment are the test scores achieved in November 2008 and pre-treatment the scores from 2005 to 2007.
\end{itemize}
\end{block} 


\begin{figure}[htb]
\vspace{-0.7cm}
\centering
\includegraphics[width=1.02\textwidth,height=0.7\textheight]{Figure/Cronograma3-GERES}
\end{figure}
     \vspace{-2.26cm}  \hspace{1pt}
\begin{minipage}{1\textwidth} 
{{\fontsize{4}{4}\selectfont  
Notes: The academic year in Brazil aligns with the calendar year, lasting from February to November. The \href{http://tiny.cc/4opsaz}{Law $N^{o}$ 1,017} of 15 October 2007 has introduced the teacher incentive bonus in the state of São Paulo, and the \href{http://tiny.cc/ympsaz}{Decree $N^{o}$ 52,719} of 14 February 2008 and the \href{http://tiny.cc/0wpsaz}{Law $N^{o}$ 1,078} of 17 December 2008 have supplemented it. In April 2009, the state has paid for the first time the productivity bonus, which was referred to the previous year's targets. \\
Source: Own elaboration based on GERES database and information from the Official Gazette of the State of São Paulo.\par}}
\end{minipage} 
\end{frame}



\section{Estimation Strategy}
\setbeamertemplate{itemize item}[triangle]

\begin{frame}{Estimation Strategy}

\begin{itemize} 
{\footnotesize\item Assuming the educational production function:}
\vspace{-5pt}

\begin{equation} \label{eq:ProdFunction}
\mathbf{Y}_{ijst} = \:f\:(\mathbf{B}_{ijst}, \mathbf{S}_{ijst}, \mathbf{C}_{ijst}, \mathbf{A}_{i})
\end{equation}

{\scriptsize where:}
\begin{itemize}
{\tiny\item $\mathbf{Y}_{ijst}$ is the test score for $i$, $j$, $s$, $t$ denoting student, school, subject, and time respectively.}
{\tiny\item $\mathbf{B}_i$, $\mathbf{S}_i$, and $\mathbf{C}_i$ denotes, respectively, student, school and teacher characteristics.}
\vspace{-5pt}
{\tiny\item $A_{i}$ is the initial ability of student $i$.}
\end{itemize}


\vspace{5pt}
{\footnotesize\item The DiD to estimate the intent-to-treat effect is given as:}
\vspace{-15pt}

\begin{equation} \label{eq:DiD-Treat}
\resizebox{.85\hsize}{!}{$\mathbf{Y}_{ijst} = \alpha + \underbrace{\phi Treated_{ijst} + \gamma Post_{t} + \delta (Treated \times Post)_{ijst}}_{\text{DiD}}+ \beta\mathbf{X}_{ijst}^{\prime} + \varphi A_{i} + \epsilon_{ijst}$} 
\end{equation}


{\scriptsize with: \hspace{50pt} $\epsilon_{ijst} = \mu^i + \mu^j + \mu^t +  \nu_{ijst}$ \hspace{30pt} and \hspace{20pt} $\nu_{ijst} \sim IID(0, \sigma^2_{\epsilon})$}




\vspace{5pt}
{\scriptsize where:}
\begin{itemize}
{\tiny\item $\mathbf{X}_{ijst}^{\prime}$ is a vector of control variables for student, teacher and school characteristics.}
{\tiny\item $Treated_{ijst}$ is a dummy variable indicating the treatment group (1 if state schools).} 
{\tiny\item $Post_{t}$ is a time dummy equal to one if the incentive pay is implemented (1 if year is 2008).}
{\tiny\item $\mu^{i}$, $\mu^{j}$ and $\mu^{t}$ are individual, school and time fixed effects respectively.}
\vspace{-3pt} 
{\tiny\item Therefore, $\delta$ is the parameter of interest.}
\end{itemize}
\end{itemize}
\end{frame}



\begin{frame}{Estimation Strategy}

\begin{itemize} 
{\footnotesize\item Educational process is cumulative over time. \\ \vspace{-7pt} \tiny\textcolor{red}{$\Longrightarrow$} But the importance of the previous educational inputs depreciates at a constant rate $\theta$.}
\vspace{-15pt}

\begin{equation} \label{eq:LinearCumulative}
\resizebox{.85\hsize}{!}{$\mathbf{Y}_{ijst} = \sum_{t=0}^T \alpha (1 - \theta)^{T-t} + DiD + \sum_{t=0}^T \beta \mathbf{X}_{ijst}^{\prime} (1 - \theta)^{T-t} + \sum_{t=0}^T \varphi A_{i} (1 - \theta)^{T-t} + \sum_{t=0}^T \epsilon_{ijst} (1 - \theta)^{T-t}$} 
\end{equation} 


\vspace{-1pt}
{\scriptsize Major challenges to estimate \eqref{eq:LinearCumulative}:}
\begin{itemize}
{\tiny\item Integration of all educational inputs for all times before $t$.}
\vspace{-5pt}
{\tiny\item Quantification of initial ability of students.} 
\end{itemize}


\vspace{8pt}
{\footnotesize\item Decomposing \eqref{eq:LinearCumulative} into current and previous educational inputs:}

\vspace{-20pt}
\begin{multline} \label{eq:LinearDecomp}
\mathbf{Y}_{ijst} = DiD + \underbrace{\alpha + \beta\mathbf{X}_{ijst}^{\prime} + \varphi A_{i} + \epsilon_{ijst}}_{\text{current}} \\
\resizebox{.85\hsize}{!}{$+ \underbrace{\sum_{t=0}^{T-1} \alpha (1 - \theta)^{T-t} + \sum_{t=0}^{T-1} \beta \mathbf{X}_{ijst}^{\prime} (1 - \theta)^{T-t} + \sum_{t=0}^{T-1} \varphi A_{i} (1 - \theta)^{T-t}  + \sum_{t=0}^{T-1} \epsilon_{ijst} (1 - \theta)^{T-t}}_{\text{previous}}$}
\end{multline}



%\vspace{5pt}
{\footnotesize\item Use of value-added strategy, i.e. $\mathbf{Y}_{ijs,t-1}$.}
\end{itemize}
\end{frame}



\begin{frame}[label=Main6]{Estimation Strategy}
\vspace{-42pt} \flushright \hyperlink{Bibliography1}{\beamerbutton{\textcolor{red}{Papers}}}
\vspace{15pt}
\begin{itemize} 
{\footnotesize\item Consequently, the lagged value-added model is given as:}
%\vspace{-10pt}

\begin{equation} \label{eq:value-added}
\mathbf{Y}_{ijst} = \alpha + DiD + \beta \mathbf{X}_{ijst}^{\prime} + \pi (1 - \theta) \mathbf{Y}_{ijs,t-1} + \epsilon_{ijst} 
\end{equation}

{\scriptsize Problem by estimating \eqref{eq:value-added}: \vspace{-5pt}
\\ \tiny\textcolor{red}{$\Longrightarrow$} Nickell (1981) bias: $Corr(\mathbf{X}_{ijs,t-1}, \nu_{ijst}) > 0$ and $Corr(\mathbf{Y}_{ijst}, \mathbf{Y}_{ijs,t-1}) > 0$.}


\vspace{15pt}
{\footnotesize\item Anderson and Hsiao (1982) difference and levels estimator (2SLS).
\\ \tiny\textcolor{red}{$\Longrightarrow$} Introduction of instrument variables (IV) to estimate consistent coefficients of $\pi$.
\\ \tiny\textcolor{red}{$\Longrightarrow$} IV should be correlated with $\mathbf{Y}_{ijs,t-1}$, but orthogonal to $\nu_{ijst}$. \vspace{-7pt}
\\ \tiny\textcolor{red}{$\Longrightarrow$} Use of first difference: $\Delta \mathbf{Y}_{ijst} = \mathbf{Y}_{ijs,t} - \mathbf{Y}_{ijs,t-1}$}

\vspace{15pt}
{\footnotesize\item Integrating first difference model and 2SLS, \eqref{eq:value-added} can be estimated by:}

\begin{equation} \label{eq:2SLS}
\mathbf{Y}_{ijst} = \alpha + DiD + \beta \mathbf{X}_{ijst}^{\prime} + \pi (1 - \theta) \widetilde{\mathbf{Y}}_{ijs,t-1} + \epsilon_{ijst} 
\end{equation}

{\scriptsize with: \hspace{80pt} $\widetilde{\mathbf{Y}}_{ijs,t-1} = \eta + Z_1(\Delta^2 \mathbf{Y}_{ijst})$}

\end{itemize}
\end{frame}




\section{Empirical Findings}



\subsection{Empirical Findings}

\begin{frame}[label=Main3]{Empirical Findings}
\vspace{-0.2cm}
\begin{block}{\centering \Large Outline}
\begin{enumerate}
\item \normalsize Baseline Results
\item \normalsize Alternative Specifications
\item \normalsize Robustness Checks
\item \normalsize Placebo Tests
\end{enumerate}
\end{block}

\vspace{10pt}
\begin{block}{\centering \Large Empirical Models}

\vspace{-2pt}
\begin{itemize}
{\scriptsize \item To check the sensitivity, the results are based on different empirical models:}
\end{itemize}
\vspace{-2pt}
\begin{enumerate}
\item \scriptsize Ordinary Least Squares (OLS)
\\ \tiny\textcolor{red}{$\Longrightarrow$} \underline{No} panel structure, \underline{no} control for fixed effects, and \underline{no} lagged dependent variable.
\item \scriptsize Fixed Effects Model (FE)
\\ \tiny\textcolor{red}{$\Longrightarrow$} With panel structure, control for fixed effects, and lagged dependent variable.
%\item \scriptsize Two-Stage Least Squares (IV-2SLS)
%\\ \tiny\textcolor{red}{$\Longrightarrow$} With panel structure, control for fixed effects, and lagged dependent variable.
%\item \scriptsize General Method Moments (GMM)
%\\ \tiny\textcolor{red}{$\Longrightarrow$} With panel structure, control for fixed effects, and \underline{multiple} lagged dependent variables.
\end{enumerate}
\end{block} 
\end{frame}




%\subsection{\footnotesize Baseline Results}


\begin{frame}{Baseline Results}
             \begin{table}[h]             \label{table:Results}                        \centering            \textbf{Table 1. Impact of Teacher Bonus on Portuguese} \\            
\begin{adjustbox}{max width=\textwidth} 
\begin{tabular}{@{\extracolsep{4pt}}l*{6}{c}@{}}             \toprule                    & \multicolumn{3}{c}{No Control Variables} &            \multicolumn{3}{c}{With Control Variables} \\            \cline{2-4}              \cline{5-7}                    & \multicolumn{1}{c}{OLS} &                    \multicolumn{1}{c}{FE} &                    \multicolumn{1}{c}{FE} &            \multicolumn{1}{c}{OLS} &                    \multicolumn{1}{c}{FE} &            \multicolumn{1}{c}{FE} \\            \cline{2-2}                    \cline{3-3}                    \cline{4-4}            \cline{5-5}                    \cline{6-6}                    \cline{7-7}                    
                    &         (1)   &         (2)   &         (3)   &         (1)   &         (2)   &         (3)   \\
\hline
DiD                 &      -0.286***&       -0.009 &      -0.007   &      -0.070   &      -0.010   &      -0.016   \\
                    &     (0.101)   &     (0.065)   &     (0.033)   &     (0.079)   &     (0.062)   &     (0.038)   \\
$\mathbf{Y}_{t-1}$              &       -        &        -       &       0.756***&         -      &        -       &       0.703***\\
                    &               &               &     (0.010)   &               &               &     (0.014)   \\
\hline
No. Observations        &       19,611   &       19,611   &       12,244   &        6,489   &        6,489   &        4,511   \\
No. Clusters          &    746           &       746        &              691 &         363      &       363        &        359       \\
R-square            &       0.039   &       0.288   &       0.630   &       0.230   &       0.288   &       0.601   \\
\hline Control variables&          No   &          No   &          No   &          Yes   &          Yes   &          Yes   \\
Fixed effects       &          No   &         Yes   &          Yes   &          No   &         Yes   &          Yes   \\
Lagged values        &          No   &          No   &         Yes   &          No   &          No   &         Yes   \\           
\vspace{-18pt} \\
\noalign{\smallskip} \hline           
\end{tabular}            \medskip        
\end{adjustbox}
\begin{minipage}{1\textwidth}            \Tiny Notes: Dependent variable is student performance (test scores), which are normalized to mean 0 and standard deviation 1. Control variables include the full set of explanatory variables presented in table A1. Fixed effects control for the average differences across schools and years. Data are not nested within schools. Standard errors are robust to heteroskedasticity and clustered at class level. t statistics in parentheses. \( * p<0.1, ** p<0.05, *** p<0.01 \). \\                    
Source: GERES database (2005-2008), own estimates.                       
\end{minipage} 
\end{table}     
\end{frame}




\begin{frame}{Baseline Results}
             \begin{table}[h]             \label{table:Results}                        \centering            \textbf{Table 2. Impact of Teacher Bonus on Mathematics} \\            
\begin{adjustbox}{max width=\textwidth}  
\begin{tabular}{@{\extracolsep{4pt}}l*{6}{c}@{}}             \toprule                    & \multicolumn{3}{c}{No Control Variables} &            \multicolumn{3}{c}{With Control Variables} \\            \cline{2-4}              \cline{5-7}                    & \multicolumn{1}{c}{OLS} &                    \multicolumn{1}{c}{FE} &                    \multicolumn{1}{c}{FE} &            \multicolumn{1}{c}{OLS} &                    \multicolumn{1}{c}{FE} &            \multicolumn{1}{c}{FE} \\            \cline{2-2}                    \cline{3-3}                    \cline{4-4}            \cline{5-5}                    \cline{6-6}                    \cline{7-7}                    
                    &         (1)   &         (2)   &         (3)   &         (1)   &         (2)   &         (3)   \\
\hline
DiD                 &      -0.255***&      -0.006   &      -0.016   &      -0.046   &       0.017   &      -0.041   \\
                    &     (0.098)   &     (0.063)   &     (0.042)   &     (0.082)   &     (0.064)   &     (0.048)   \\
$\mathbf{Y}_{t-1}$              &      -         &      -         &       0.714***&      -         &     -          &       0.665***\\
                    &               &               &     (0.010)   &               &               &     (0.014)   \\
\hline
No. Observations        &       19,520   &       19,520   &       12,103   &        6,461   &        6,461   &        4,446   \\
No. Clusters          &       746        &     746          &              687 &       363        &        363       &       357        \\
R-square            &       0.034   &       0.253   &       0.598   &       0.199   &       0.256   &       0.585   \\
\hline Control variables&          No   &          No   &          No   &          Yes   &          Yes   &          Yes   \\
Fixed effects       &          No   &         Yes   &          Yes   &          No   &         Yes   &          Yes   \\
Lagged values        &          No   &          No   &         Yes   &          No   &          No   &         Yes   \vspace{-7pt} \\
             \noalign{\smallskip} \bottomrule            \end{tabular}            \medskip           
\end{adjustbox}
\begin{minipage}{1\textwidth}            \Tiny Notes: Dependent variable is student performance (test scores), which are normalized to mean 0 and standard deviation 1. Control variables include the full set of explanatory variables presented in table A1. Fixed effects control for the average differences across schools and years. Data are not nested within schools. Standard errors are robust to heteroskedasticity and clustered at class level. t statistics in parentheses. \( * p<0.1, ** p<0.05, *** p<0.01 \). \\                    
Source: GERES database (2005-2008), own estimates.                       
\end{minipage} 
\end{table}     
\end{frame}




\subsection{\footnotesize Alternative Specifications}
\setbeamertemplate{itemize item}[circle]
\newcommand{\sbt}{\,\begin{picture}(-1,1)(-1,-3)\circle*{4}\end{picture}\ }


\begin{frame} {Alternative Specifications}
\begin{block}{\centering \Large Specification A: State vs Other Schools}
\begin{itemize}
\item [\sbt] \scriptsize Same empirical model as in section before.
\item [\sbt] \scriptsize Limitation to Campinas.
\item [\sbt] \scriptsize But control group contains private schools.
\item [\sbt] \scriptsize \underline{State schools as treatment and private schools as control group.}
\end{itemize}
\end{block}

\vspace{0.7cm}
\begin{block}{\centering \Large Specification B: Campinas vs Other Cities}
\begin{itemize}
\item [\sbt] \scriptsize Same empirical model as in section before.
\item [\sbt] \scriptsize No limitation to Campinas. \textcolor{red}{$\Longrightarrow$} Inclusion of all GERES cities.
\item [\sbt] \scriptsize Limitation to state schools.
\item [\sbt] \scriptsize \underline{State schools in Campinas as treatment and state schools in other cities as control.} 
\end{itemize}
\end{block}  
\end{frame}




\begin{frame} {Alternative Specifications}
 \begin{table}[h]            \refstepcounter{table}                        \label{table:AlternativeSpecification}                        \centering            \textbf{Table 3. Alternative Specifications of the Research Design} \\             
\begin{adjustbox}{max width=\textwidth}          
\begin{tabular}{@{\extracolsep{4pt}}l*{8}{c}@{}}             \toprule             & \multicolumn{4}{c}{\textbf{(A) State vs. private Schools}} &            \multicolumn{4}{c}{\textbf{(B) Campinas vs. other Cities}} \\            \cline{2-5}              \cline{6-9}                    & \multicolumn{2}{c}{\textbf{Portuguese}} &                    \multicolumn{2}{c}{\textbf{Mathematics}} &                    \multicolumn{2}{c}{\textbf{Portuguese}} &            \multicolumn{2}{c}{\textbf{Mathematics}} \\            \cline{2-3}                    \cline{4-5}                    \cline{6-7}            \cline{8-9}          
                    &         (1)   &         (2)   &         (3)   &         (4)   &         (1)   &         (2)   &         (3)   &         (4)   \\
\hline
DiD                 &      -0.020   &      0.035   &      0.032   &      0.051   &      -0.018   &   0.099      &   0.079      &        0.158 \\
                    &    (0.035)   &    (0.045)   &    (0.043)   &    (0.059)   &   (0.047)    &  (0.072)     &   (0.056)    &  (0.092)     \\
$\mathbf{Y}_{t-1}$              &       0.719***&       0.686***&              0.695*** &      0.673***         &  0.740***     &   0.701***    &              0.678*** &      0.671***         \\
                    &    (0.011)   &    (0.016)   &       (0.012)        &    (0.015)           &   (0.011)    &   (0.018)    &              (0.012) &      (0.017)         \\
\hline
No. Observations    &       8,939   &        4,062   &       8,875   &        4,019   &   9,647       &     3,075      &         9,598 &    3,024       \\
No. Clusters        &      492         &    299           &              488 &      298         &    626           &     270          &              625 &     269          \\
R-square            &       0.687   &       0.683   &       0.669   &       0.681   &     0.628     &   0.612       &   0.591       &         0.601 \\
\hline Control variables&          No   &         Yes   &          No   &         Yes   &          No   &         Yes   &          No   &         Yes   \\
Fixed effects       &         Yes   &         Yes   &         Yes   &         Yes   &         Yes   &         Yes   &         Yes   &         Yes   \\
Lagged value        &         Yes   &         Yes   &         Yes   &         Yes   &         Yes   &         Yes   &         Yes   &         Yes \vspace{-5pt}  \\
            \noalign{\smallskip} \bottomrule             \end{tabular}
\end{adjustbox}      \medskip      
\begin{minipage}{1\textwidth}            \Tiny Notes: Specification A concentrates the investigation on Campinas estimating the DiD between state schools (treatment group) and private schools (control). Specification B expands the DiD analysis to all GERES students enrolled in state schools, in which 1 are students from Campinas and 0 otherwise. Dependent variable is student performance (test scores), which are normalized to mean 0 and standard deviation 1. Control variables include the full set of explanatory variables presented in Table A1. Fixed effects control for the average differences across schools and years. Data are not nested within schools. Standard errors are robust to heteroskedasticity and clustered at class level. t statistics in parentheses. \( * p<0.1, ** p<0.05, *** p<0.01 \). \\                    
Source: GERES database (2005\textendash 2008), own estimates.            \end{minipage}                \end{table} 
\end{frame}











\subsection{\footnotesize Placebo Tests}

\begin{frame} {Placebo Tests}
\begin{block}{\centering \Large Estimation 1: Fake Implementation Date}
\begin{itemize}
\item [\sbt] \scriptsize Same treatment and control groups as in baseline results. \\ \tiny \textcolor{red}{$\Longrightarrow$} State vs municipal school in Campinas.
\item [\sbt] \scriptsize But using a ``fake" date for the implementation of bonus: 2007 instead of 2008.
\\ \tiny\textcolor{red}{$\Longrightarrow$} Test scores from 2008 were missed.
\\ \tiny\textcolor{red}{$\Longrightarrow$} Post-treatment are the test scores achieved in November 2007 and pre-treatment the scores from 2005 to 2006. 

\end{itemize}
\end{block}

\vspace{0.7cm}
\begin{block}{\centering \Large Estimation 2: Fake Treatment Group}
\begin{itemize}
\item [\sbt] \scriptsize Same research design and implementation timetable as in baseline results.
\\ \tiny \textcolor{red}{$\Longrightarrow$} State vs municipal schools.
\\ \tiny \textcolor{red}{$\Longrightarrow$} Post-treatment are the test scores achieved in November 2008 and pre-treatment the scores from 2005 to 2007.
\item [\sbt] \scriptsize But using a ``fake" group for the bonus: Schools in which no bonus was implemented.
\\ \tiny\textcolor{red}{$\Longrightarrow$} Students enrolled in Campinas were missed.
\\ \tiny\textcolor{red}{$\Longrightarrow$} Investigating the other GERES cities (Campo Grande, Rio de Janeiro and Salvador).
\end{itemize}
\end{block}  
\end{frame}



\begin{frame} {Placebo Tests}
             \begin{table}[h]            \refstepcounter{table}                        \label{table:PlaceboTest}                        \centering            \textbf{Table 4. Placebo Tests} \\
\vspace{5pt}                    
\begin{adjustbox}{max width=\textwidth}             
\begin{tabular}{@{\extracolsep{4pt}}l*{8}{c}@{}}             \toprule                    & \multicolumn{4}{c}{\textbf{Fake Implementation Date}} &            \multicolumn{4}{c}{\textbf{Fake Treatment Group}} \\            \cline{2-5}              \cline{6-9}                    & \multicolumn{2}{c}{Portuguese} &                    \multicolumn{2}{c}{Mathematics} &                    \multicolumn{2}{c}{Portuguese} &            \multicolumn{2}{c}{Mathematics} \\            \cline{2-3}                    \cline{4-5}                    \cline{6-7}            \cline{8-9}                    
                    &         (1)   &         (2)   &         (3)   &         (4)   &         (1)   &         (2)   &         (3)   &         (4)   \\
\hline
DiD                 &      -0.037   &      -0.062   &       0.053   &      -0.028   &   -0.012      &   -0.073      &    0.031      &        -0.055 \\
                    &     (0.038)   &     (0.059)   &     (0.044)   &     (0.060)   &   (0.045)     &   (0.076)     &   (0.052)     &       (0.090) \\
$\mathbf{Y}_{t-1}$              &       0.721***&       0.664***&              0.675*** &       0.620***        &  0.719***     &  0.686***     &     0.670***          &    0.662***           \\
                    &     (0.012)   &     (0.019)   &              (0.012) &        (0.019)       &  (0.008)      &   (0.014)     &    (0.008)           &       (0.016)        \\
\hline
No. Observations        &       9,133   &        3,019   &       9,004   &        2,952   &    22,437      &    4,268       &         22,284 &    4,236       \\
No. Clusters          &     498          &    265           &              494 &      263         &    1,745           &       530        &              1,747 &      530         \\
R-square            &       0.608   &       0.565   &       0.583   &       0.553   &     0.539     &   0.573       &   0.483       &         0.510 \\
\hline Control variables&          No   &         Yes   &          No   &         Yes   &          No   &         Yes   &          No   &         Yes   \\
Fixed effects       &         Yes   &         Yes   &         Yes   &         Yes   &         Yes   &         Yes   &         Yes   &         Yes   \\
Lagged value        &         Yes   &         Yes   &         Yes   &         Yes   &         Yes   &         Yes   &         Yes   &         Yes  \vspace{-5pt} \\
            \noalign{\smallskip} \bottomrule            \end{tabular}            \medskip            
\end{adjustbox}            
\begin{minipage}{1\textwidth}            \Tiny Notes: The first placebo test performs the DiD analysis with a fake date for the implementation of bonus (2007 instead of 2008). The second one uses a fake treatment group expanding the investigation to all GERES students with the exception of those from Campinas and Belo Horizonte, being pupils enrolled in state (municipal) schools the treated (control) group. Dependent variable is student performance (test scores), which are normalized to mean 0 and standard deviation 1. Control variables include the full set of explanatory variables presented in Table A1. Fixed effects control for the average differences across schools and years. Data are not nested within schools. Standard errors are robust to heteroskedasticity and clustered at class level. t statistics in parentheses. \( * p<0.1, ** p<0.05, *** p<0.01 \). \\                    Source: GERES database (2005\textendash 2008), own estimates.                        \end{minipage}                \end{table}


\end{frame}




\subsection{\footnotesize Robustness Checks}

\begin{frame}[label=Main4]{Robustness Checks}
\begin{block}{\centering \Large Generalized Method of Moments (GMM)}
\vspace{-19pt} \flushright \hyperlink{GMM}{\beamerbutton{\textcolor{red}{Formal}}}
\begin{itemize}
\item [\sbt] \scriptsize Same DiD as in baseline results. \\ \tiny \textcolor{red}{$\Longrightarrow$} State vs municipal school in Campinas.
\item [\sbt] \scriptsize Integration of multiple lagged student achievements into the empirical model.
\item [\sbt] \scriptsize Two-phase approach:
\begin{enumerate}
\item \scriptsize First difference estimation to eliminate individual fixed effects.
\item \scriptsize Inclusion of lagged dependent variable as instrument.
\end{enumerate}
\item [\sbt] \scriptsize Working with two different estimators:
\begin{enumerate}
\item \scriptsize Arellano and Bond (1991) difference GMM estimator.
\item \scriptsize Blundell and Bond (1998) system GMM estimator.
\end{enumerate}
\item [\sbt] \scriptsize Estimation of two-step GMM. \\ \tiny \textcolor{red}{$\Longrightarrow$} Improving the robustness of the coefficients in relation to heteroscedasticity and autocorrelation.
\item [\sbt] \scriptsize Windmeijer (2005) finite-sample correction.
\\ \tiny \textcolor{red}{$\Longrightarrow$} Dynamic panel data contain large $N$, but small $T$.
\end{itemize}
\end{block}
\end{frame}




\begin{frame} {Robustness Checks}
\begin{table}[h]            \refstepcounter{table}                        \label{table:Robustness}                        \centering            \textbf{Table 5. Robustness Checks} \\             
\vspace{5pt}
\begin{adjustbox}{max width=\textwidth}           
\begin{tabular}{@{\extracolsep{4pt}}l*{8}{c}@{}}             \toprule             & \multicolumn{4}{c}{\textbf{Difference GMM}} &            \multicolumn{4}{c}{\textbf{System GMM}}  \\            \cline{2-5}              \cline{6-9}                   & \multicolumn{2}{c}{Portuguese} &                    \multicolumn{2}{c}{Mathematics} &                    \multicolumn{2}{c}{Portuguese} &            \multicolumn{2}{c}{Mathematics}  \\            \cline{2-3}                    \cline{4-5}                    \cline{6-7}            \cline{8-9}        
                    &         (1)   &         (2)   &         (3)   &         (4)   &         (1)   &         (2)   &         (3)   &         (4)   \\
\hline
DiD                 &    0.025     &   0.027      &   0.012     &       -0.008 &   0.018     &    -0.005     &   -0.030      &  -0.073       \\
                    &  (0.028)      &   (0.039)     &   (0.039)     &    (0.056)    & (0.035)   &   (0.042)    &   (0.056)     &      (0.070) \\
$\mathbf{Y}_{t-1}$              &  0.240***    &  0.295***     &              0.135*** &       0.203***        &  0.671***     & 0.713***      &              0.586*** &     0.603***       \\
                    &  (0.046)      &   (0.056)     &    (0.038)        & (0.060)  &   (0.041)     &   (0.050)     &    (0.045)         &            (0.068) \\
\hline 
AB for AR(1)  &        0.000       &      0.000         &              0.000 & 0.000    &             0.000 &        0.000       &             0.000 &             0.000           \\
AB for AR(2) &      0.151         &    0.204         &              0.058 & 0.177   &  0.002   &   0.084           &       0.002        &       0.096        \\
Sargan              &     0.152          &       0.050        &             0.423  &      0.571         &       0.590        &       0.442        &     0.597          &    0.789           \\                    
\hline
No. Observations        &    7,180      &   3,504        &         7,040 &     3,438      &    12,244      &    8,264       &   12,103 &   8,181        \\
No. Clusters          &    539           &    280           &              534 &      280        &      691         &     400          &              687 &      398       \\
%R-square            &     -     &      -    &     -     &         - &         - &   -       &    -      &    -      \\
\hline Control variables&          No   &         Yes   &          No   &         Yes   &          No   &         Yes   &          No   &         Yes   \\
Fixed effects       &         Yes   &         Yes   &         Yes   &         Yes   &         Yes   &         Yes   &         Yes   &         Yes   \\
Lagged value        &         Yes   &         Yes   &         Yes   &         Yes   &         Yes   &         Yes   &         Yes   &         Yes  \vspace{-5pt} \\
            \noalign{\smallskip} \bottomrule             \end{tabular}            \medskip           
\end{adjustbox}            
             \begin{minipage}{1\textwidth}            \Tiny Notes: Dependent variable is student performance (test scores), which are normalized to mean 0 and standard deviation 1. Control variables include the full set of explanatory variables presented in Table A1. Data are not nested within schools. Standard errors are robust to heteroskedasticity and clustered at class level. t statistics in parentheses. Estimations based on two-step GMM with Windmeijer (2005) corrected standard errors. The zero hypothesis of the Sargan test is H0: overidentifying restrictions are valid. For the Arellano-Bond test, AR(1) and AR(2) are respectively tests for first-order and second-order correlation in the first-differenced residuals under the null hypothesis of H0: no autocorrelation. For the Arellano-Bond and Sargan tests the $\text{\emph{p}}$-values are reported. \( * p<0.1, ** p<0.05, *** p<0.01 \).\\                    Source: GERES database (2005\textendash 2008), own estimates.            \end{minipage}                \end{table}
\end{frame}




\section{Conclusion}

\begin{frame}[label=Main5]{Conclusion}
\begin{block}{\centering \Large Main Findings}
\begin{enumerate}
\item \scriptsize Teacher Bonus has lead to an increase in the performance by Mathematics. \\ \tiny \textcolor{red}{$\Longrightarrow$} Student test scores in program (state) schools were between $0.087$ and $0.378$ standard deviation - dependent on the empirical model - higher than in comparison (municipal) schools.
\item \scriptsize No significant impact was found by Portuguese (reading).
\end{enumerate}
\end{block}

\vspace{0.3cm}
\begin{block}{\centering \Large Caveats and Limitations of the Study}
\vspace{-19pt} \flushright \hyperlink{Bibliography}{\beamerbutton{\textcolor{red}{Papers}}}
\begin{enumerate}
\item \scriptsize Natural experiment limited to a single city.
\\ \tiny \textcolor{red}{$\Longrightarrow$} Campinas is only one of the $645$ municipalities in the state of São Paulo, where the teacher incentive pay program was implemented.
\\ \tiny \textcolor{red}{$\Longrightarrow$} Possible extrapolations to other populations should be treated with caution.
\item \scriptsize DiD approach is based on a single year as post-treatment period (2008).
\\ \tiny \textcolor{red}{$\Longrightarrow$} Same approach used in other prestigious research projects. 
\\ \tiny \textcolor{red}{$\Longrightarrow$} The question to the effects of teacher bonus in the subsequent years remains open.
\item \scriptsize Transferability of the results to students of higher grades.
\\ \tiny \textcolor{red}{$\Longrightarrow$} The GERES database focused on pupils enrolled in the first four years of primary education.
\\ \tiny \textcolor{red}{$\Longrightarrow$} The older the students, the higher the importance of teachers for their academic performance.
\end{enumerate}
\end{block}  
\end{frame}


\begin{frame}{Conclusion}
\begin{block}{\centering Open Questions}
\begin{enumerate}
\item \footnotesize Link between teacher bonus and student performance.
\\ \tiny \textcolor{red}{$\Longrightarrow$} How the bonus could lead to an increase in student performance? Higher engagement or better qualification?
\\ \tiny \textcolor{red}{$\Longrightarrow$} Addition of the mechanisms to this paper or topic for other research project?
\item \footnotesize Difference by the results of Math and Portuguese
\\ \tiny\textcolor{red}{$\Longrightarrow$} Contradictory outcomes?
\\ \tiny\textcolor{red}{$\Longrightarrow$} How should I present/address that?
\item \footnotesize Literature Review.
\\ \tiny\textcolor{red}{$\Longrightarrow$} Is necessary a single section for literature review?
\\ \tiny\textcolor{red}{$\Longrightarrow$} No peer-reviewed publication for Brazil.
\\ \tiny\textcolor{red}{$\Longrightarrow$} Focus on papers of other countries?
\item \footnotesize Fixed Effect in GMM.
\\ \tiny\textcolor{red}{$\Longrightarrow$} Bug in $xtabond2$, $xtabond$ and $xtdpdsys$.
\\ \tiny\textcolor{red}{$\Longrightarrow$} By omitted coefficients, the degrees of freedom for the Sargan/Hansen test are incorrectly calculated.
\\ \tiny\textcolor{red}{$\Longrightarrow$} How to address this limitation?
\end{enumerate}
\end{block}
\end{frame}


\begin{frame}
\vspace{1cm}
\begin{center}
\Huge\textcolor{blue}{Thank you for your attention !} \\

\large {More Information? \\ Questions? \\ Suggestions? \\}
\vspace{0.5cm}
Please contact me: \\
Tharcisio Leone \\
\vspace{0.2cm}
\Email{\hspace{5pt} tharcisio.leone@giga-hamburg.de} \\
\ComputerMouse{\hspace{5pt} www.tharcisio-leone.com} \\
\phone{\hspace{5pt} +49 (0)40 42825-796}  
\end{center}
\end{frame}



\begin{frame}[label=Mirror]{Appendix Mirror}
\vspace{-27pt} \flushright \hyperlink{Outline}{\beamerbutton{\textcolor{red}{Back}}}

\setbeamertemplate{button}{\tikz
  \node[
  inner xsep=10pt,
  draw=structure!80,
  fill=structure!50,
  rounded corners=4pt]  {\Large\insertbuttontext};}
  
\vspace{15pt}
\begin{columns}
  \begin{column}{0.6\textwidth}
  \vspace{10pt}
  \hyperlink{IndPerf}{\beamerbutton {\textcolor{red}{Average Passing Rate}}}
           \vspace{10pt}
    \hyperlink{ICM}{\beamerbutton {\textcolor{red}{Realisation of Targets}}}
     \vspace{10pt}
     \hyperlink{IRT}{\beamerbutton
         {\textcolor{red}{Item Response Theory}}}
    \vspace{10pt}
    \hyperlink{INSE}{\beamerbutton {\textcolor{red}{Socio-economic Status}}}
        \vspace{10pt}
        \hyperlink{IndPerf}{\beamerbutton {\textcolor{red}{Indicator of Performance}}}
            \vspace{10pt}
    \hyperlink{IDESP}{\beamerbutton {\textcolor{red}{Education Development Index}}}
    \vspace{10pt}
    \hyperlink{GMM}{\beamerbutton {\textcolor{red}{Generalized Method Moments}}}
    \vspace{10pt}
    \hyperlink{EPF}{\beamerbutton {\textcolor{red}{Education Production Function}}}
  \end{column}
  
  
 
 \begin{column}{0.4\textwidth}
 \vspace{10pt}
    \hyperlink{Bibliography1}{\beamerbutton {\textcolor{red}{Bibliography}}}
    \vspace{10pt}
    \hyperlink{BonusCalculation}{\beamerbutton {\textcolor{red}{Bonus Formula}}}
            \vspace{10pt}
    \hyperlink{Stata}{\beamerbutton {\textcolor{red}{Stata Syntaxes}}}
     \vspace{10pt}
    \hyperlink{GERESSample}{\beamerbutton {\textcolor{red}{GERES Sample}}}
    \vspace{10pt}
    \hyperlink{SampleWave}{\beamerbutton {\textcolor{red}{Sample per Wave}}}
    \vspace{10pt}
    \hyperlink{ScoresWave}{\beamerbutton {\textcolor{red}{Scores per Wave}}}
       \vspace{10pt}
    \hyperlink{SARESP}{\beamerbutton {\textcolor{red}{Reference SARESP}}}
    \vspace{10pt}
    \hyperlink{SARESP}{\beamerbutton {\textcolor{red}{School Quality Gap}}}
  \end{column}
\end{columns}

\end{frame}



\appendix
\setcounter{equation}{0}
\renewcommand\theequation{A.\arabic{equation}}
\setnextsection{1}

\setbeamertemplate{section in toc}[sections numbered roman]


\setbeamertemplate{button}{\tikz
  \node[
  inner xsep=5pt,
  draw=structure!20,
  fill=structure!10,
  rounded corners=4pt]  {\scriptsize\insertbuttontext};}




\section{GERES}
\begin{frame}[label=GERESSample]{GERES Sample Size}
\vspace{-27pt} \flushright \hyperlink{Mirror}{\beamerbutton{\textcolor{red}{Mirror}}}
\vspace{-8pt}
\begin{table}[H]
\label{table:SampleSize}                         
  \begin{adjustbox}{max width=\textwidth}
    \begin{tabular}{llllllll}
    \toprule
          &       & \multicolumn{1}{c}{\textbf{Total}} &       & \multicolumn{4}{c}{\textbf{GERES}} \\
\cmidrule{3-3}\cmidrule{5-8}    \multicolumn{1}{c}{\textbf{City}} & \multicolumn{1}{c}{\textbf{Typ}} & \multicolumn{1}{p{4.39em}}{\textbf{ Schools }} &       & \multicolumn{1}{p{5.5em}}{\textbf{ Schools }} & \multicolumn{1}{p{5.5em}}{\textbf{ \% Sample }} & \multicolumn{1}{c}{\textbf{ Classes }} & \multicolumn{1}{c}{\textbf{ Students }} \\
    \midrule
          & State & \multicolumn{1}{c}{155 } &       & \multicolumn{1}{c}{20 } & \multicolumn{1}{c}{12.9} & \multicolumn{1}{c}{64 } & \multicolumn{1}{c}{1,682 } \\
    Belo Horizonte & Municipal & \multicolumn{1}{c}{135 } &       & \multicolumn{1}{c}{20 } & \multicolumn{1}{c}{14.8} & \multicolumn{1}{c}{88 } & \multicolumn{1}{c}{2,036 } \\
          & Private & \multicolumn{1}{c}{144 } &       & \multicolumn{1}{c}{20 } & \multicolumn{1}{c}{13.9} & \multicolumn{1}{c}{32 } & \multicolumn{1}{c}{669 } \vspace{-3pt} \\
    \midrule
          & State & \multicolumn{1}{c}{95 } &       & \multicolumn{1}{c}{20 } & \multicolumn{1}{c}{21.1} & \multicolumn{1}{c}{74 } & \multicolumn{1}{c}{2,158 } \\
    Campinas & Municipal & \multicolumn{1}{c}{39 } &       & \multicolumn{1}{c}{20 } & \multicolumn{1}{c}{51.3} & \multicolumn{1}{c}{73 } & \multicolumn{1}{c}{1,919 } \\
          & Private & \multicolumn{1}{c}{47 } &       & \multicolumn{1}{c}{20 } & \multicolumn{1}{c}{42.6} & \multicolumn{1}{c}{42 } & \multicolumn{1}{c}{804 } \vspace{-3pt} \\
    \midrule
          & State & \multicolumn{1}{c}{70 } &       & \multicolumn{1}{c}{20 } & \multicolumn{1}{c}{28.6} & \multicolumn{1}{c}{38 } & \multicolumn{1}{c}{845 } \\
    Campo Grande & Municipal & \multicolumn{1}{c}{76 } &       & \multicolumn{1}{c}{19 } & \multicolumn{1}{c}{25.0} & \multicolumn{1}{c}{97 } & \multicolumn{1}{c}{2,418 } \\
          & Private & \multicolumn{1}{c}{80 } &       & \multicolumn{1}{c}{20 } & \multicolumn{1}{c}{25.0} & \multicolumn{1}{c}{27 } & \multicolumn{1}{c}{342 } \vspace{-3pt} \\
    \midrule
          & State & \multicolumn{1}{c}{-} &       & \multicolumn{1}{c}{-} & \multicolumn{1}{c}{-} & \multicolumn{1}{c}{-} & \multicolumn{1}{c}{-} \\
    Rio de Janeiro & Municipal & \multicolumn{1}{c}{765 } &       & \multicolumn{1}{c}{30 } & \multicolumn{1}{c}{3.9} & \multicolumn{1}{c}{90 } & \multicolumn{1}{c}{2,527 } \\
          & Private & \multicolumn{1}{c}{805 } &       & \multicolumn{1}{c}{30 } & \multicolumn{1}{c}{3.7} & \multicolumn{1}{c}{55 } & \multicolumn{1}{c}{1,032 } \vspace{-3pt} \\
    \midrule
          & State & \multicolumn{1}{c}{67 } &       & \multicolumn{1}{c}{21 } & \multicolumn{1}{c}{31.3} & \multicolumn{1}{c}{25 } & \multicolumn{1}{c}{692 } \\
    Salvador & Municipal & \multicolumn{1}{c}{332 } &       & \multicolumn{1}{c}{20 } & \multicolumn{1}{c}{6.0} & \multicolumn{1}{c}{113 } & \multicolumn{1}{c}{3,032 } \\
          & Private & \multicolumn{1}{c}{278} &       & \multicolumn{1}{c}{20 } & \multicolumn{1}{c}{7.2} & \multicolumn{1}{c}{30 } & \multicolumn{1}{c}{544 } \vspace{-3pt} \\
    \midrule
    \multicolumn{2}{p{12.5em}}{Special schools} & \multicolumn{1}{c}{9 } &       & \multicolumn{1}{c}{9 } & \multicolumn{1}{c}{100.0} & \multicolumn{1}{c}{35 } & \multicolumn{1}{c}{869 } \vspace{-3pt} \\
    \midrule
    \textbf{Total} &       & \multicolumn{1}{c}{\textbf{3,097}} &       & \multicolumn{1}{c}{\textbf{309}} & \multicolumn{1}{c}{\textbf{10.0}} & \multicolumn{1}{c}{\textbf{883}} & \multicolumn{1}{c}{\textbf{21,569}} \vspace{-3pt} \\
    \bottomrule
   \end{tabular}%
  \end{adjustbox}
%\vspace{-13pt} \hspace{-20pt}
    \begin{minipage}{1\textwidth} 
{\tiny
Notes: Special schools refer to federal schools or classes in participating universities. Because of the municipalization of primary education in Rio de Janeiro, no state schools have been included in the GERES sample. \\ Source: GERES database (2005-2008), own compilation based on Brooke and Bomamino (2011). \par} 
\end{minipage}
\vspace{10pt}  
\end{table}%
\end{frame}



\begin{frame}[label=SampleWave]{GERES Sample per Wave}
\vspace{-27pt} \flushright \hyperlink{Main2}{\beamerbutton{\textcolor{red}{Back}}}
\vspace{30pt}
\begin{table}[H]
  \begin{adjustbox}{max width=\textwidth}
    \begin{tabular}{llllllllllllr}
    \toprule
          &       & \multicolumn{5}{c}{\textbf{Number of Schools / Wave}} &       & \multicolumn{5}{c}{\textbf{Number of Students / Wave}} \\
\cmidrule{3-7}\cmidrule{9-13}    \multicolumn{1}{c}{\textbf{City}} & \multicolumn{1}{c}{\textbf{Typ}} & \multicolumn{1}{p{3.445em}}{\textbf{ 1st }} & \multicolumn{1}{p{3.445em}}{\textbf{ 2nd  }} & \multicolumn{1}{p{3.445em}}{\textbf{ 3rd  }} & \multicolumn{1}{p{3.445em}}{\textbf{ 4th  }} & \multicolumn{1}{p{3.445em}}{\textbf{ 5th  }} &       & \multicolumn{1}{p{3.445em}}{\textbf{ 1st }} & \multicolumn{1}{p{3.445em}}{\textbf{ 2nd  }} & \multicolumn{1}{p{3.445em}}{\textbf{ 3rd  }} & \multicolumn{1}{p{3.445em}}{\textbf{ 4th  }} & \multicolumn{1}{p{3.445em}}{\textbf{ 5th  }} \\
\cmidrule{1-7}\cmidrule{9-13}          & Special & \multicolumn{1}{c}{1 } & \multicolumn{1}{c}{1 } & \multicolumn{1}{c}{1 } & \multicolumn{1}{c}{1 } & \multicolumn{1}{c}{1 } &       & \multicolumn{1}{c}{88 } & \multicolumn{1}{c}{85 } & \multicolumn{1}{c}{87 } & \multicolumn{1}{c}{89 } & \multicolumn{1}{c}{92 } \\
    Belo Horizonte & State & \multicolumn{1}{c}{20 } & \multicolumn{1}{c}{20 } & \multicolumn{1}{c}{20 } & \multicolumn{1}{c}{20 } & \multicolumn{1}{c}{20 } &       & \multicolumn{1}{c}{1,521 } & \multicolumn{1}{c}{1,554 } & \multicolumn{1}{c}{1,649 } & \multicolumn{1}{c}{1,670 } & \multicolumn{1}{c}{1,654 } \\
          & Municipal & \multicolumn{1}{c}{20 } & \multicolumn{1}{c}{20 } & \multicolumn{1}{c}{20 } & \multicolumn{1}{c}{20 } & \multicolumn{1}{c}{20 } &       & \multicolumn{1}{c}{1,840 } & \multicolumn{1}{c}{1,888} & \multicolumn{1}{c}{2,096 } & \multicolumn{1}{c}{2,044 } & \multicolumn{1}{c}{2,000 } \\
          & Private & \multicolumn{1}{c}{19 } & \multicolumn{1}{c}{19 } & \multicolumn{1}{c}{19 } & \multicolumn{1}{c}{19 } & \multicolumn{1}{c}{19 } &       & \multicolumn{1}{c}{641 } & \multicolumn{1}{c}{646 } & \multicolumn{1}{c}{655 } & \multicolumn{1}{c}{653 } & \multicolumn{1}{c}{653 } \\
\cmidrule{1-7}\cmidrule{9-13}          & State & \multicolumn{1}{c}{20 } & \multicolumn{1}{c}{20 } & \multicolumn{1}{c}{20 } & \multicolumn{1}{c}{20 } & \multicolumn{1}{c}{16 } &       & \multicolumn{1}{c}{1,939 } & \multicolumn{1}{c}{2,017} & \multicolumn{1}{c}{2,095 } & \multicolumn{1}{c}{2,201 } & \multicolumn{1}{c}{1,860 } \\
    Campinas & Municipal & \multicolumn{1}{c}{21 } & \multicolumn{1}{c}{21 } & \multicolumn{1}{c}{21 } & \multicolumn{1}{c}{21 } & \multicolumn{1}{c}{21 } &       & \multicolumn{1}{c}{1,691 } & \multicolumn{1}{c}{1,761} & \multicolumn{1}{c}{1,886 } & \multicolumn{1}{c}{2,084 } & \multicolumn{1}{c}{2,203 } \\
          & Private & \multicolumn{1}{c}{20 } & \multicolumn{1}{c}{20 } & \multicolumn{1}{c}{20 } & \multicolumn{1}{c}{19 } & \multicolumn{1}{c}{18 } &       & \multicolumn{1}{c}{768 } & \multicolumn{1}{c}{781} & \multicolumn{1}{c}{748 } & \multicolumn{1}{c}{766 } & \multicolumn{1}{c}{738 } \\
\cmidrule{1-7}\cmidrule{9-13}          & State & \multicolumn{1}{c}{19 } & \multicolumn{1}{c}{19 } & \multicolumn{1}{c}{19 } & \multicolumn{1}{c}{18 } & \multicolumn{1}{c}{18 } &       & \multicolumn{1}{c}{734 } & \multicolumn{1}{c}{825} & \multicolumn{1}{c}{864 } & \multicolumn{1}{c}{875 } & \multicolumn{1}{c}{833 } \\
    Campo Grande & Municipal & \multicolumn{1}{c}{20 } & \multicolumn{1}{c}{20 } & \multicolumn{1}{c}{20 } & \multicolumn{1}{c}{20 } & \multicolumn{1}{c}{20 } &       & \multicolumn{1}{c}{2,171 } & \multicolumn{1}{c}{2,247} & \multicolumn{1}{c}{2,197 } & \multicolumn{1}{c}{2,311 } & \multicolumn{1}{c}{2,115 } \\
          & Private & \multicolumn{1}{c}{20 } & \multicolumn{1}{c}{20 } & \multicolumn{1}{c}{20 } & \multicolumn{1}{c}{18 } & \multicolumn{1}{c}{14 } &       & \multicolumn{1}{c}{315 } & \multicolumn{1}{c}{330} & \multicolumn{1}{c}{260 } & \multicolumn{1}{c}{211 } & \multicolumn{1}{c}{188 } \\
\cmidrule{1-7}\cmidrule{9-13}          & Special & \multicolumn{1}{c}{8 } & \multicolumn{1}{c}{8 } & \multicolumn{1}{c}{8 } & \multicolumn{1}{c}{8 } & \multicolumn{1}{c}{8 } &       & \multicolumn{1}{c}{730 } & \multicolumn{1}{c}{720} & \multicolumn{1}{c}{740 } & \multicolumn{1}{c}{764 } & \multicolumn{1}{c}{799 } \\
    Rio de Janeiro & Municipal & \multicolumn{1}{c}{30 } & \multicolumn{1}{c}{30 } & \multicolumn{1}{c}{30 } & \multicolumn{1}{c}{36 } & \multicolumn{1}{c}{35 } &       & \multicolumn{1}{c}{2,271 } & \multicolumn{1}{c}{2,230} & \multicolumn{1}{c}{2,224 } & \multicolumn{1}{c}{3,575 } & \multicolumn{1}{c}{3,802 } \\
          & Private & \multicolumn{1}{c}{30 } & \multicolumn{1}{c}{30 } & \multicolumn{1}{c}{30 } & \multicolumn{1}{c}{30 } & \multicolumn{1}{c}{28 } &       & \multicolumn{1}{c}{971 } & \multicolumn{1}{c}{949} & \multicolumn{1}{c}{934 } & \multicolumn{1}{c}{935 } & \multicolumn{1}{c}{778 } \\
\cmidrule{1-7}\cmidrule{9-13}          & State & \multicolumn{1}{c}{11 } & \multicolumn{1}{c}{11 } & \multicolumn{1}{c}{11 } & \multicolumn{1}{c}{10 } & \multicolumn{1}{c}{-} &       & \multicolumn{1}{c}{571 } & \multicolumn{1}{c}{575} & \multicolumn{1}{c}{757 } & \multicolumn{1}{c}{845 } & \multicolumn{1}{c}{-} \\
    Salvador & Municipal & \multicolumn{1}{c}{26 } & \multicolumn{1}{c}{26 } & \multicolumn{1}{c}{26 } & \multicolumn{1}{c}{26 } & \multicolumn{1}{c}{-} &       & \multicolumn{1}{c}{2,278 } & \multicolumn{1}{c}{2,332} & \multicolumn{1}{c}{2,629 } & \multicolumn{1}{c}{2,657 } & \multicolumn{1}{c}{-} \\
          & Private & \multicolumn{1}{c}{18 } & \multicolumn{1}{c}{18 } & \multicolumn{1}{c}{18 } & \multicolumn{1}{c}{17 } & \multicolumn{1}{c}{-} &       & \multicolumn{1}{c}{495 } & \multicolumn{1}{c}{488} & \multicolumn{1}{c}{512 } & \multicolumn{1}{c}{477 } & \multicolumn{1}{c}{-} \\
    \midrule
    \textbf{Total} &       & \multicolumn{1}{c}{\textbf{303 }} & \multicolumn{1}{c}{\textbf{303 }} & \multicolumn{1}{c}{\textbf{303 }} & \multicolumn{1}{c}{\textbf{303 }} & \multicolumn{1}{c}{\textbf{238 }} &       & \multicolumn{1}{c}{\textbf{19,024 }} & \multicolumn{1}{c}{\textbf{19,428 }} & \multicolumn{1}{c}{\textbf{20,333 }} & \multicolumn{1}{c}{\textbf{22,157 }} & \multicolumn{1}{c}{\textbf{17,715 }} \\
    \bottomrule
    \end{tabular}%
\end{adjustbox}

%\vspace{-5pt} \hspace{-15pt}
    \begin{minipage}{1\textwidth} 
{\tiny
Notes: Special schools refer to federal schools or classes in participating universities. Because of the municipalization of primary education in Rio de Janeiro, no state schools have been included in the GERES sample. Due to administrative issues the students from the city of Salvador were removed from the GERES project in year 2008, and therefore they have no test scores for the wave 5. \\ Source: GERES database (2005-2008), own compilation based on Brooke and Bomamino (2011).\par}
\end{minipage}
\end{table}% 
\vspace{10pt}
\end{frame}



\begin{frame}[label=ScoresWave]{Test Scores per Wave}
\vspace{-67pt} \flushright \hyperlink{Mirror}{\beamerbutton{\textcolor{red}{Mirror}}}
\vspace{30pt}
\begin{table}[H]
\label{table:AverageGrade}                        
  \begin{adjustbox}{width=1\textwidth}
    \begin{tabular}{llllllllllllllll}
    \toprule
          &       & \multicolumn{2}{p{6.055em}}{\textbf{\hspace{12pt} Wave 1 }} &       & \multicolumn{2}{p{6.055em}}{\textbf{\hspace{12pt} Wave 2 }} &       & \multicolumn{2}{p{6.055em}}{\textbf{\hspace{12pt} Wave 3 }} &       & \multicolumn{2}{p{6.055em}}{\textbf{\hspace{12pt} Wave 4 }} &       & \multicolumn{2}{p{6.055em}}{\textbf{\hspace{12pt} Wave 5 }} \\
\cmidrule{3-4}\cmidrule{6-7}\cmidrule{9-10}\cmidrule{12-13}\cmidrule{15-16}    \multicolumn{1}{c}{\textbf{City}} & \multicolumn{1}{c}{\textbf{Typ}} & \multicolumn{1}{p{3.445em}}{\textbf{ Mean }} & \multicolumn{1}{p{2.61em}}{\textbf{ SD }} &       & \multicolumn{1}{p{3.445em}}{\textbf{ Mean }} & \multicolumn{1}{p{2.61em}}{\textbf{ SD }} &       & \multicolumn{1}{p{3.445em}}{\textbf{ Mean }} & \multicolumn{1}{p{2.61em}}{\textbf{ SD }} &       & \multicolumn{1}{p{3.445em}}{\textbf{ Mean }} & \multicolumn{1}{p{2.61em}}{\textbf{ SD }} &       & \multicolumn{1}{p{3.445em}}{\textbf{ Mean }} & \multicolumn{1}{p{2.61em}}{\textbf{ SD }} \\
    \midrule
          & Special & \multicolumn{1}{c}{142.29} & \multicolumn{1}{c}{20.84} &       & \multicolumn{1}{c}{170.11} & \multicolumn{1}{c}{24.46} &       & \multicolumn{1}{c}{217.67} & \multicolumn{1}{c}{40.10} &       & \multicolumn{1}{c}{259.67} & \multicolumn{1}{c}{43.43} &       & \multicolumn{1}{c}{301.47} & \multicolumn{1}{c}{48.13} \\
          & State & \multicolumn{1}{c}{102.33} & \multicolumn{1}{c}{28.65} &       & \multicolumn{1}{c}{131.48} & \multicolumn{1}{c}{31.62} &       & \multicolumn{1}{c}{146.21} & \multicolumn{1}{c}{52.46} &       & \multicolumn{1}{c}{188.29} & \multicolumn{1}{c}{59.20} &       & \multicolumn{1}{c}{229.87} & \multicolumn{1}{c}{63.05} \\
    Mathematics & Municipal & \multicolumn{1}{c}{101.27} & \multicolumn{1}{c}{24.98} &       & \multicolumn{1}{c}{128.74} & \multicolumn{1}{c}{28.59} &       & \multicolumn{1}{c}{135.31} & \multicolumn{1}{c}{49.18} &       & \multicolumn{1}{c}{174.48} & \multicolumn{1}{c}{56.32} &       & \multicolumn{1}{c}{217.79} & \multicolumn{1}{c}{57.67} \\
          & Private & \multicolumn{1}{c}{128.92} & \multicolumn{1}{c}{25.74} &       & \multicolumn{1}{c}{159.19} & \multicolumn{1}{c}{28.37} &       & \multicolumn{1}{c}{198.55} & \multicolumn{1}{c}{47.41} &       & \multicolumn{1}{c}{240.35} & \multicolumn{1}{c}{53.84} &       & \multicolumn{1}{c}{293.61} & \multicolumn{1}{c}{52.62} \\
\cmidrule{2-16}          & \textbf{Total} & \multicolumn{1}{c}{\textbf{107.95}} & \multicolumn{1}{c}{\textbf{28.78}} &       & \multicolumn{1}{c}{\textbf{136.22}} & \multicolumn{1}{c}{\textbf{32.00}} &       & \multicolumn{1}{c}{\textbf{151.25}} & \multicolumn{1}{c}{\textbf{55.81}} &       & \multicolumn{1}{c}{\textbf{190.39}} & \multicolumn{1}{c}{\textbf{62.01}} &       & \multicolumn{1}{c}{\textbf{235.19}} & \multicolumn{1}{c}{\textbf{64.96}} \\
    \midrule
          & Special & \multicolumn{1}{c}{139.64} & \multicolumn{1}{c}{14.49} &       & \multicolumn{1}{c}{151.84} & \multicolumn{1}{c}{15.17} &       & \multicolumn{1}{c}{169.81} & \multicolumn{1}{c}{19.70} &       & \multicolumn{1}{c}{179.53} & \multicolumn{1}{c}{18.42} &       & \multicolumn{1}{c}{188.50} & \multicolumn{1}{c}{16.74} \\
          & State & \multicolumn{1}{c}{101.24} & \multicolumn{1}{c}{15.01} &       & \multicolumn{1}{c}{120.84} & \multicolumn{1}{c}{21.93} &       & \multicolumn{1}{c}{136.00} & \multicolumn{1}{c}{25.41} &       & \multicolumn{1}{c}{151.50} & \multicolumn{1}{c}{25.46} &       & \multicolumn{1}{c}{162.79} & \multicolumn{1}{c}{24.76} \\
    Portuguese & Municipal & \multicolumn{1}{c}{101.15} & \multicolumn{1}{c}{22.60} &       & \multicolumn{1}{c}{118.54} & \multicolumn{1}{c}{19.88} &       & \multicolumn{1}{c}{131.03} & \multicolumn{1}{c}{23.81} &       & \multicolumn{1}{c}{145.39} & \multicolumn{1}{c}{24.94} &       & \multicolumn{1}{c}{158.68} & \multicolumn{1}{c}{23.92} \\
          & Private & \multicolumn{1}{c}{129.64} & \multicolumn{1}{c}{18.27} &       & \multicolumn{1}{c}{145.02} & \multicolumn{1}{c}{17.23} &       & \multicolumn{1}{c}{160.67} & \multicolumn{1}{c}{22.44} &       & \multicolumn{1}{c}{172.27} & \multicolumn{1}{c}{22.92} &       & \multicolumn{1}{c}{185.32} & \multicolumn{1}{c}{18.13} \\
\cmidrule{2-16}          & \textbf{Total} & \multicolumn{1}{c}{\textbf{107.61}} & \multicolumn{1}{c}{\textbf{25.60}} &       & \multicolumn{1}{c}{\textbf{124.88}} & \multicolumn{1}{c}{\textbf{22.73}} &       & \multicolumn{1}{c}{\textbf{138.47}} & \multicolumn{1}{c}{\textbf{26.80}} &       & \multicolumn{1}{c}{\textbf{151.95}} & \multicolumn{1}{c}{\textbf{26.75}} &       & \multicolumn{1}{c}{\textbf{164.75}} & \multicolumn{1}{c}{\textbf{25.33}} \\
    \bottomrule
     \end{tabular}%
 \end{adjustbox}
\begin{minipage}{1\textwidth} 
{\tiny
Notes: Special schools refer to federal schools or classes in participating universities. \\ Source: GERES database (2005-2008), own estimates.\par}
\end{minipage}
\end{table}% 
%\vspace{10pt}
\end{frame}










\section{IRT}

\begin{frame}[label=IRT]{Item Response Theory}
\vspace{-35pt} \flushright \hyperlink{Mirror}{\beamerbutton{\textcolor{red}{Mirror}}}
\vspace{10pt}
\begin{block}{\centering Proficiency Level by GERES}
 \begin{itemize}
\item [\sbt] \scriptsize The test scores by GERES were calculated using the Item Response Theory (IRT).
\item [\sbt] \scriptsize IRT is widely applied in studies of cognitive traits and educational outcomes.
\\ \tiny\textcolor{red}{$\Longrightarrow$} In Brazil, IRT is adopted by large-scale evaluations of educational achievements, such as SAEB and Prova Brasil.
\item [\sbt] \scriptsize Mathematical model to calculate the likelihood of each possible response to an item.
\\ \tiny\textcolor{red}{$\Longrightarrow$} Hit probability of the students is derived as a function of their proficiency and some item parameters.
\item [\sbt] \scriptsize Concurrent Calibration to link the parameters above to the base scale.
\\ \tiny\textcolor{red}{$\Longrightarrow$} Use of Maximum Marginal A Posteriori (MMAP) and Expected A Posteriori (EAP) method.
\end{itemize}
\end{block}


%\vspace{7pt}
\setbeamertemplate{itemize item}[triangle]
\begin{itemize} 
{\scriptsize\item The three-parameter logistic (3PL) model gives the probability of a correct answer:}
\end{itemize}

\vspace{-10pt}
\begin{equation} \label{eq:IRT}
P(Y_{jg} = 1 \vert a_j,b_j,c_j,\theta_i) = c_j + \frac{(1-c_j)}{1+e^{-Da_j(\theta-b_j)}}
\end{equation}

\setbeamertemplate{itemize item}[circle]
\begin{flushleft} \hspace{9pt}
{\scriptsize where:}
\end{flushleft}
\vspace{-15pt}
\begin{itemize}
{\tiny\item $P(Y_{jg} = 1 \vert)$ denotes the probability of the answer $Y_{ig}$ attributed to student $i$ in group $g$ is correct.}

{\tiny\item $a_j$, $b_j$ and $c_j$ denotes, respectively, the item's discrimination, the item's difficulty, and the pseudo-guessing parameter presenting the likelihood for a casual hit of the item.}
\vspace{-8pt}
{\tiny\item $\theta_i$ describes the ability of student $i$, and $D$ is a scaling constant.}
\end{itemize}
\end{frame}



\section{IDESP}

\begin{frame}[label=IDESP]{Education Development Index}
\vspace{-42pt} \flushright \hyperlink{Mirror}{\beamerbutton{\textcolor{red}{Mirror}}}
\vspace{10pt}
\begin{block}{\centering Education Development Index (IDESP)}
 \begin{itemize}
\item [\sbt] \scriptsize Indicator of education quality for the schools.
\item [\sbt] \scriptsize Short-term (annually) targets for learning improvement.
\item [\sbt] \scriptsize It enables the schools to monitor annually their progress toward the long-term goals.
\item [\sbt] \scriptsize It is calculated individually for each school and educational level:
\\ \tiny\textcolor{red}{$\Longrightarrow$} Lower primary education (1st-5th grades), upper primary education (6th-9th grades), and secondary education (10th-12th grades).
\end{itemize}
\end{block}



\vspace{7pt}
\setbeamertemplate{itemize item}[triangle]
\begin{itemize} 
{\scriptsize\item The IDESP is calculated individually for each school $s$ and educational level $l$, as follows:}
\end{itemize}

\begin{equation} \label{eq:IDESP}
IDESP_{sl} = IP_{sl} \cdot PR_{sl}
\end{equation}

\setbeamertemplate{itemize item}[circle]
\begin{flushleft} \hspace{9pt}
{\scriptsize where:}
\end{flushleft}
\vspace{-10pt}
\begin{itemize}
{\scriptsize\item $IP_{sl}$ is an indicator of performance, and $PR_{sl}$ the average passing rate.
%\vspace{-6pt}
\\ \tiny\textcolor{red}{$\Longrightarrow$} A good school is one where the students don't move on to the next grade with learning deficits, and at the same \vspace{-8pt} \\ time, they don' t need to repeat the same grade several times to learn the corresponding contents.}
\end{itemize}
\end{frame}


\begin{frame}[label=IndPerf]{Education Development Index}
\vspace{-28pt} \flushright \hyperlink{Main1}{\beamerbutton{\textcolor{red}{Back}}}
\vspace{10pt}
\begin{block}{\centering Indicator of Performance}
\begin{itemize}
\item [\sbt] \scriptsize Use of the compulsory standardized test SARESP. 
\\ \tiny\textcolor{red}{$\Longrightarrow$} Evaluation System of Learning Achievement in the State of São Paulo.
\\ \tiny\textcolor{red}{$\Longrightarrow$} IP is measured as the average score of students calculated individually for each of the three grades $g$ (5th, 9th and 12th) and two disciplines $j$ (math and portuguese).
\end{itemize}

\vspace{-5pt}
\setbeamertemplate{itemize item}[triangle]
\begin{itemize} 
{\scriptsize\item The indicator of performance is calculated using the school quality gap: }
\end{itemize}

\vspace{-8pt}
\begin{equation} \label{eq:IndPerformance}
IP_{sgj} = \left(1 - \frac{Sgap_{sgj}}{3}\right) \cdot 10
\end{equation}

\vspace{-8pt}
{\scriptsize with:}

\vspace{-25pt}
\begin{equation} \label{eq:Sgap}
Sgap_{sgj} = (3 \cdot BB_{sgj}) + (2 \cdot Ba_{sgj}) + (1 \cdot Ap_{sgj}) + (0 \cdot Ad_{sgj})
\end{equation}

\vspace{-10pt}
\begin{itemize}
\item [\sbt] \scriptsize Four categories of learning achievements: 
\\ \tiny\textcolor{red}{$\Longrightarrow$} Below the basic $(BB)$, basic $(Ba)$, appropriate $(Ap)$ and advanced $(Ad)$.
\item [\sbt] \scriptsize Indicator of performance ranges between 0 (maximal gap in quality) and 10 (no gap).
\end{itemize}
\end{block}

\begin{block}{\centering Average Passing Rate}
\begin{itemize}
\item [\sbt] \scriptsize It is calculated using the data from the Brazilian School Census.

\begin{equation} \label{eq:PassingRate}
PR_{sl} = \frac{A_{sl}}{T_{sl}} = \frac{\text{Number of approved students in level of schooling} \: l}{\text{Total number of students in level of schooling} \: l}
\end{equation}
\end{itemize}
\end{block}
\end{frame}



\begin{frame}[label=SARESP]{Education Development Index}
\vspace{-34pt} \flushright \hyperlink{Mirror}{\beamerbutton{\textcolor{red}{Mirror}}}
\vspace{20pt}
\begin{block}{\centering School Quality Gap}
\setbeamertemplate{itemize item}[triangle]
\begin{itemize} 
{\scriptsize\item The indicator of performance is calculated using the School Quality Gap ($Sgap_{sgj}$):}


\vspace{-20pt}
\begin{equation} \label{eq:Sgap}
Sgap_{sgj} = (3 \cdot BB_{sgj}) + (2 \cdot Ba_{sgj}) + (1 \cdot Ap_{sgj}) + (0 \cdot Ad_{sgj}) \nonumber
\end{equation}


\item [\sbt] \scriptsize Percentage of students in the respectively performance level listed in table A4. 
\\ \tiny\textcolor{red}{$\Longrightarrow$} Below the basic $(BB)$, basic $(Ba)$, appropriate $(Ap)$ and advanced $(Ad)$.
\item [\sbt] \scriptsize School Quality Gap ranges between $0$ and $3$.
\end{itemize}
\end{block}

%\vspace{10pt}
\begin{table}[H]
  \centering
 \refstepcounter{table}           \label{table:ReferenceSARESP}                        \centering            \textbf{Table A4. Reference values for SARESP} \vspace{2pt} \\ 
  \begin{adjustbox}{width=1\textwidth}
    \begin{tabular}{lrcccccccc}
    \toprule
     &       & \multicolumn{2}{c}{\textbf{5th Grade}} &       & \multicolumn{2}{c}{\textbf{9th Grade}} &       & \multicolumn{2}{c}{\textbf{12th Grade}} \\
\cmidrule{3-4}\cmidrule{6-7}\cmidrule{9-10}    \multicolumn{1}{c}{\textbf{Level}} &       & \textbf{Mathematics} & \textbf{Portuguese} &       & \textbf{Mathematics} & \textbf{Portuguese} &       & \textbf{Mathematics} & \textbf{Portuguese} \\
    \midrule
    \midrule
    Below Basic &       & $x < 175$ & $x < 150$ &       & $x < 225$ & $x < 200$ &       & $x < 275$ & $x < 250$ \\
    Basic &       & $175 \leq x < 225$ & $150 \leq x < 200$ &       & $225 \leq x < 300$ & $200 \leq x < 275$ &       & $275 \leq x < 350$ & $250 \leq x < 300$ \\
    Appropriate &       & $225 \leq x < 275$ & $200 \leq x < 250$ &       & $300 \leq x < 350$ & $275 \leq x < 325$ &       & $350 \leq x < 400$ & $300 \leq x < 375$ \\
    Advanced &       & $x \geq 275$ & $x \geq 250$ &       & $x \geq 350$ & $x \geq 325$ &       & $x \geq 400$ & $x \geq 375$ \\
     \bottomrule
      \end{tabular}%
  \label{tab:SARESP}%
\end{adjustbox}
\end{table}%

\vspace{-9pt} 
    \begin{minipage}{1\textwidth} 
{\tiny
Note: $x$ denotes the student score by SARESP. \vspace{-1pt}\\ Source: Author's own compilation based on SEE-SP (2018). \par}
\end{minipage} 
\vspace{5pt}
\end{frame}




\section{ICM}


\begin{frame}[label=ICM]{Indicator for Realisation of Targets (ICM)}
\vspace{-27pt} \flushright \hyperlink{Mirror}{\beamerbutton{\textcolor{red}{Mirror}}}
\vspace{10pt}
\setbeamertemplate{itemize item}[triangle]
\begin{itemize} 
{\scriptsize\item The ICM is measured individually for each school and level of schooling, as follows:}
\end{itemize}

\vspace{-15pt}
\begin{equation} \label{eq:ICM}
ICM = \left[\max{(IC;IQ)} \right] \cdot \left[1 +(NSE \cdot MOD) \right]
\end{equation}

\vspace{-15pt}
\begin{flushleft} \hspace{9pt}
{\scriptsize with}
\end{flushleft}

\vspace{-20pt}
\begin{equation} \label{eq:IC}
IC = \text{Compliance Index} = \left( \frac{IDESP_{EF} - IDESP_{BASE}}{IDESP_{META} - IDESP_{BASE}} \right)
\end{equation}

\vspace{-10pt}
\begin{flushleft} \hspace{9pt}
{\scriptsize and}
\end{flushleft}

\vspace{-20pt}
\begin{equation} \label{eq:AQ}
IQ = \text{Additional for Quality} = \left( \frac{IDESP_{EF} - IDESP_{AG}}{IDESP_{MF} - IDESP_{AG}} \right)
\end{equation}

\vspace{-5pt}
\setbeamertemplate{itemize item}[circle]
\begin{flushleft} \hspace{9pt}
{\scriptsize where:}
\end{flushleft}
\vspace{-5pt}
\begin{itemize}
{\tiny\item $IDESP_{EF}$ is the IDESP achieved in the evaluated period, $IDESP_{BASE}$ the value considered as the basis (previous year), $IDESP_{META}$ the target for the evaluation period, $IDESP_{MF}$ the final goal for 2030, $IDESP_{AG}$ the aggregate result for the evaluation period.}
\vspace{-5pt}
{\tiny\item $INSE$ is the indicator for socio-economic status of the school, and $MOD$ a modulator to the $INSE$.}
\end{itemize}

\vspace{-2pt}
\begin{block}{\centering Indicator for Realisation of Targets}
\begin{itemize}
%\item [\sbt] \scriptsize Measure of the ``achievement" of the short-time targets (IDESP).
%\\ \tiny\textcolor{red}{$\Longrightarrow$} After the end of the evaluated year, the SEE-SP calculates the Indicator for Realisation of Targets (ICM), and consequently the value of the teacher bonus paid to the educational staffs.
\item [\sbt] \scriptsize Compliance Index (IC).
\\ \tiny\textcolor{red}{$\Longrightarrow$} It presents the proportion of the target that the school has achieved in the evaluated year.
\item [\sbt] \scriptsize Additional for Quality (IQ).
\\ \tiny\textcolor{red}{$\Longrightarrow$} It indicates how close is the school to achieve the final goal (2030) compared to the other schools.
\item [\sbt] \scriptsize Payment of bonus in two different situations:
\\ \tiny\textcolor{red}{$\Longrightarrow$} If IC $>0$, independent of IQ.
\\ \tiny\textcolor{red}{$\Longrightarrow$} If IQ $>0$, independent of IC.
\end{itemize}
\vspace{-5pt}
\end{block}
\end{frame}





\begin{frame}[label=BonusCalculation]{Indicator for Realisation of Targets (ICM)}
\vspace{-31pt} \flushright \hyperlink{Main1}{\beamerbutton{\textcolor{red}{Back}}} 
\vspace{20pt}
\setbeamertemplate{itemize item}[triangle]
\begin{itemize} 
{\scriptsize\item The ICM is applied in the calculation of the teacher bonus, as follows:}
\end{itemize}

\vspace{-10pt}
\begin{equation} \label{eq:Bonus}
\text{Bonus} = \frac{1}{6} \cdot \text{ICM} \cdot \sum Y_{i,t} \cdot \frac{\sum WD_{i,t}}{\sum WD_{t}} \quad \iff \quad \frac{\sum WD_{i,t}}{\sum WD_{t}} \geq \frac{2}{3}
\end{equation}


\vspace{-5pt}
\setbeamertemplate{itemize item}[circle]
\begin{flushleft} \hspace{9pt}
{\scriptsize where:}
\end{flushleft}
\vspace{-8pt}
\begin{itemize}
{\tiny\item $\sum Y_{i,t}$ denotes the sum of the remunerations received by the official $i$ in the evaluated period $t$.}
%\vspace{-5pt}
{\tiny\item $\sum WD_{t}$ the amount of working days in $t$.}
\vspace{-3pt}
{\tiny\item $\sum WD_{i,t}$ the sum of days worked by the official $i$ in period $t$.}
\end{itemize}

\begin{block}{\centering Bonus Calculation}
\begin{itemize}
\item [\sbt] \scriptsize Proportional to the effective days worked in $t$. 
\\ \tiny\textcolor{red}{$\Longrightarrow$} But no bonus will be paid for employees with a proportion of working days lower than $2/3$.
\item [\sbt] \scriptsize ICM is limited to values between $0$ and $1.2$. 
\\ \tiny\textcolor{red}{$\Longrightarrow$} If ICM $=0$ (IC $\leq 0$ and IQ $\leq 0$), no bonus for the teachers will be paid.
\\ \tiny\textcolor{red}{$\Longrightarrow$} The ICM included in \eqref{eq:Bonus} will not exceed the value of $1.2$ even though the result from \eqref{eq:ICM} is higher than $1.2$.
\item [\sbt] \scriptsize Maximum Bonus Payment.
\\ \tiny\textcolor{red}{$\Longrightarrow$} $20$ percent of the gross salary in period $t$. 
\end{itemize}
\end{block}
\end{frame}




\section{INSE}

\begin{frame}[label=INSE]{Socio-economic Status}
\vspace{-45pt} \flushright \hyperlink{Mirror}{\beamerbutton{\textcolor{red}{Mirror}}}
\vspace{20pt}
\begin{block}{\centering Indicator for Socio-economic Status (INSE)}
 \begin{itemize}
\item [\sbt] \scriptsize INSE of students is taken in consideration in the calculation of the teacher bonus.
\\ \tiny\textcolor{red}{$\Longrightarrow$} Theory of cultural reproduction (Bourdieu, 1986).
\\ \tiny\textcolor{red}{$\Longrightarrow$} Economic and social capital have a positive impact on student educational achievements.
\item [\sbt] \scriptsize Data collection by means of questionnaires to the parents of the students.
\\ \tiny\textcolor{red}{$\Longrightarrow$} Questions to family income, education and occupation of parents, and the existence of consumer durables in the household (car, TV, DVD, washing machine, fridge, etc.).
\item [\sbt] \scriptsize Creation of a single index for socio-economic status.
\\ \tiny\textcolor{red}{$\Longrightarrow$} Use of Item Response Theory (IRT) to aggregate all the variables into a single index.
\\ \tiny\textcolor{red}{$\Longrightarrow$} INSE for the school is determined as the arithmetic average INSE of all students enrolled in this school.
\item [\sbt] \scriptsize INSE takes values between $0$ and $10$.
\\ \tiny\textcolor{red}{$\Longrightarrow$} A value of 0 (10) corresponds to the schools with the highest (lowest) social-economic status.
\end{itemize}
\end{block}

\vspace{10pt}
\begin{block}{\centering Modulator (MOD) }
 \begin{itemize}
\item [\sbt] \scriptsize MOD is taken in consideration in the calculation of the teacher bonus.
\\ \tiny\textcolor{red}{$\Longrightarrow$} It is a relative weight applied to the INSE in order to calibrate the progress of the school towards its targets.
\\ \tiny\textcolor{red}{$\Longrightarrow$} It indicates for each school the influence of the social-economic status on the performance of the students.
\end{itemize}
\end{block}
\end{frame}






\section{GMM}

\begin{frame}[label=GMM]{Generalized Method of Moments (GMM)}
\vspace{-32pt} \flushright \hyperlink{Bibliography2}{\beamerbutton{\textcolor{red}{Papers}}} \vspace{-32pt} \hyperlink{Mirror}{\beamerbutton{\textcolor{red}{Mirror}}}
\setbeamertemplate{itemize item}[triangle]
\vspace{50pt}
\begin{itemize} 
{\scriptsize\item Lets consider a (reduced) autoregressive model:}
\end{itemize}

\vspace{-10pt}
\begin{equation} \label{eq:GMMLinear}
y_{it} = \gamma y_{i,t-1} + u_{it} 
\end{equation}

%{\scriptsize for $i= 1 \dotsc, N$ and $t= 1, \dotsc, T$.}

\vspace{-15pt}
\setbeamertemplate{itemize item}[circle]
\begin{flushleft} \hspace{9pt}
{\scriptsize where:}
\end{flushleft}
\vspace{-10pt}
\begin{itemize}
{\scriptsize\item $y_{i,t-1}$ is the logged value of $y_{it}$.}
\vspace{-3pt}
{\scriptsize\item $u_{it}$ is a stochastic term with a two-way error component. \vspace{-6pt}
\\ \tiny\textcolor{red}{$\Longrightarrow$} $u_{it} =\mu_{i} + \nu_{it}$ with $\mu_{i} \sim \text{IDD}(0, \sigma^2_{\mu})$ and $\nu_{it} \sim \text{IDD}(0, \sigma^2_{\nu})$.}
\end{itemize}

\setbeamertemplate{itemize item}[triangle]
\begin{itemize} 
{\scriptsize\item By applying the first difference of \eqref{eq:GMMLinear} to swip out the individual effects:}

\vspace{-5pt}
\begin{equation} \label{eq:GMM-FD}
(y_{it} - y_{i,t-1}) = \gamma (y_{i,t-1} - y_{i,t-2}) + (\nu_{it} - \nu_{i,t-1})
\end{equation}


{\scriptsize\item Assuming, for example, $t=3$, we have:}

\vspace{-5pt}
\begin{equation} \label{eq:GMM-FDt3}
(y_{i3} - y_{i,2}) = \gamma (y_{i,2} - y_{i,1}) + (\nu_{i3} - \nu_{i,2})
\end{equation}

\vspace{5pt}
{\scriptsize\item Consequently, $y_{i,1}$ can be used as valid instrument in \eqref{eq:GMM-FDt3}. 
\\ \tiny\textcolor{red}{$\Longrightarrow$} $y_{i,1}$ is correlated with $(y_{i,2} - y_{i,1})$, but not with $(\nu_{i3} - \nu_{i,2})$ as long as $\nu_{it}$ is not serially correlated.
\vspace{-7pt}
\\ \tiny\textcolor{red}{$\Longrightarrow$} For $t>3$ this pattern is continuing with the addition of a extra valid instrument for each time period.}
\end{itemize}
\end{frame}


\begin{frame}{Generalized Method of Moments (GMM)}

\setbeamertemplate{itemize item}[triangle]
\begin{itemize} 
{\scriptsize\item The matrix of instruments $W_i$ for period $T$ can therefore be written as follows:}

\begin{tiny}
\begin{equation} \label{eq:GMM-Matrix}
W_i = \left( \begin{array}{cccc}
[y_{i1}] & 0 & \cdots & 0 \\
0 & [y_{i1}, y_{i2}] & 0 & \vdots \\
\vdots & 0 & \ddots & 0 \\
0 & \cdots & 0 & [y_{i1}, \dotsc, y_{i,T-2}] \\
\end{array}\right)
\end{equation}
\end{tiny}

{\scriptsize\item By pre-multiplying the differenced equation \eqref{eq:GMM-FD} in vector form by $W^{\prime}$, we have:}

\begin{scriptsize}
\begin{equation} \label{eq:GMM-Vector}
W^{\prime} \Delta y = W^{\prime} (\Delta y_{-1})\gamma + W^{\prime} \Delta \nu
\end{equation}
\end{scriptsize}

\vspace{-10pt} 
{\scriptsize\item $\Delta \nu_i$ has mean zero and variance $E(\Delta \nu_i \Delta \nu_i^{\prime})= \sigma^2_{\nu} G$.
\vspace{-7pt} 
\\ \tiny\textcolor{red}{$\Longrightarrow$} Where $\Delta \nu_i^{\prime} = (\nu_{i3} - \nu_{i2}, \dotsc, \nu_{iT} - \nu_{i,T-1})$ and}

\begin{tiny}
\begin{equation} \label{eq:GMM-Matrix}
G = \left( \begin{array}{ccccccc}
2 & -1 & 0 & \cdots & 0 & 0 & 0 \\
-1 & 2 & -1 & \cdots & 0 & 0 & 0 \\
0 & -1 & 2 & \cdots & 0 & 0 & 0 \\
\vdots & \vdots  & \vdots & \ddots & \vdots & \vdots & \vdots \\
0 & 0  & 0  & \cdots & 2 & -1 & 0 \\
0 & 0  & 0  & \cdots & -1 & 2 & -1 \\
0 & 0  & 0  & \cdots & 0 & -1 & 2 \\
\end{array}\right)
\end{equation}
\end{tiny}
\end{itemize}
\end{frame}


\begin{frame}{Generalized Method of Moments (GMM)}
\vspace{-66pt} \flushright \hyperlink{Main4}{\beamerbutton{\textcolor{red}{Back}}}
\setbeamertemplate{itemize item}[triangle]

\vspace{40pt}
\begin{itemize} 
{\scriptsize\item Therefore, the Arellano and Bond (1991) one-step estimator can be found by:}

\vspace{-10pt}
\begin{scriptsize}
\begin{multline} \label{eq:GMM-AB-OneStep}
\hat{\gamma}_1 = [(\Delta y_{-1})^{\prime} W(W^{\prime}(I_N \varotimes G)W)^{-1}W^{\prime}(\Delta y_{-1})]^{-1} \\ \times [(\Delta y_{-1})^{\prime} W(W^{\prime}(I_N \varotimes G)W)^{-1}W^{\prime}(\Delta y)]
\end{multline}
\end{scriptsize}
\end{itemize}


\vspace{-5pt}
\setbeamertemplate{itemize item}[circle]
\begin{itemize}
{\scriptsize\item Limitation: \vspace{-8pt}
\\ \tiny\textcolor{red}{$\Longrightarrow$} One-step GMM does not provide efficient estimate of $\gamma$ by ${N\to\infty}$ and $T$ fixed (Hansen, 1982).}
\end{itemize}


\vspace{20pt}
\setbeamertemplate{itemize item}[triangle]
\begin{itemize} 
{\scriptsize\item The two-step Arellano and Bond (1991) estimator is given by:}

\begin{scriptsize}
\begin{equation} \label{eq:GMM-AB-TwoStep}
\hat{\gamma}_2 = [(\Delta y_{-1})^{\prime} W\hat{V}_N^{-1}W^{\prime}(\Delta y_{-1})]^{-1} \times [(\Delta y_{-1})^{\prime} W\hat{V}_N^{-1}W^{\prime}(\Delta y_{-1})]
\end{equation}
\end{scriptsize}
\end{itemize}
\end{frame}




\section{Stata} 
\begin{frame}[fragile, label=Stata]{Stata Syntaxes}
\vspace{-28pt} \flushright \hyperlink{Main3}{\beamerbutton{\textcolor{red}{Back}}}
\vspace{5pt}
\begin{lstlisting}[language=R]
## Estimating OLS Pooled
regress Zprofic_mat DiD time treated $controlvar, vce(cluster IDturma)

## Estimating FE
reghdfe Zprofic_mat ZMat_L1 DiD time treated $controlvar, absorb(IDescola wave) vce(cluster IDturma)

## Estimating GMM-DIFFERENCE 
xi: xtabond2 L(0/1).Zprofic_mat DiD time treated $controlvar i.wave i.IDescola, gmmstyle(L1.Zprofic_mat) ivstyle(DiD time treated $controlvar i.wave i.IDescola) robust twostep cluster(IDturma) noleveleq

## Estimating GMM-SYSTEM 
xi: xtabond2 L(0/1).Zprofic_mat DiD time treated $controlvar i.wave i.IDescola, gmmstyle(L1.Zprofic_mat) ivstyle(DiD time treated $controlvar i.wave i.IDescola) robust twostep cluster(IDturma)
\end{lstlisting}
\end{frame}




\section{Reference}

\begin{frame}[label=Bibliography1]{Bibliography}
\vspace{-105pt} \flushright \hyperlink{Mirror}{\beamerbutton{\textcolor{red}{Mirror}}}
\vspace{50pt}
\begin{block}{\centering Fixed effect estimates by Lagged value-added models}
\vspace{-17pt} \flushright \hyperlink{Main6}{\beamerbutton{\textcolor{red}{Back}}}
\vspace{5pt}
\begin{thebibliography}{9}
\setbeamertemplate{bibliography item}[article]
\tiny \bibitem{A} Britton, J. and Propper, C. (2016). Teacher pay and school productivity: Exploiting wage
regulation. \textit{Journal of Public Economics}, 133:75-89.
\setbeamertemplate{bibliography item}[article]
\tiny \bibitem{B} Imberman, S. A., and Lovenheim, M. F. (2015). Incentive strength and teacher productivity: Evidence from a group-based teacher incentive pay system. \textit{Review of Economics and Statistics}, 97(2):364-386.
\end{thebibliography}
\end{block}
\end{frame}


\begin{frame}[label=Bibliography2]{Bibliography}
\vspace{-75pt} \flushright \hyperlink{Mirror}{\beamerbutton{\textcolor{red}{Mirror}}}
\vspace{50pt}
\begin{block}{\centering GMM in top-tier Peer Review Journals}
\vspace{-17pt} \flushright \hyperlink{GMM}{\beamerbutton{\textcolor{red}{Back}}}
\vspace{5pt}
\begin{thebibliography}{9}
\setbeamertemplate{bibliography item}[article]
\tiny \bibitem{A} Blundell, R., Pistaferri, L., and Saporta-Eksten, I. (2018). Children, time allocation, and consumption insurance. \textit{Journal of Political Economy}, 126(S1), 73-115.

\setbeamertemplate{bibliography item}[article]
\tiny \bibitem{B} Jessen, R., Rostam-Afschar, D., and Schmitz, S. (2018). How important is precautionary labour supply?. \textit{Oxford Economic Papers}, 70(3), 868-891.

\tiny \bibitem{C} Fosu, A. K., and Abass, A. F. (2019). Domestic credit and export Diversification: Africa from a global perspective. \textit{Journal of African Business}, 20(2), 160-179.

\setbeamertemplate{bibliography item}[article]
\tiny \bibitem{D} Zengin, H., and Korkmaz, A. (2019). Determinants of Turkey’s foreign aid behavior. New Perspectives on Turkey. \textit{Cambridge University Press}, 60(1), 109-135.

\setbeamertemplate{bibliography item}[article] 
\tiny \bibitem{E} Brito, R. D., and Bystedt, B. (2010). Inflation targeting in emerging economies: Panel evidence. \textit{Journal of Development Economics}, 91(2), 198-210.
\end{thebibliography}
\end{block}
\end{frame}


\begin{frame}[label=Bibliography]{Bibliography}
\vspace{-57pt} \flushright \hyperlink{Mirror}{\beamerbutton{\textcolor{red}{Mirror}}}
\vspace{50pt}
\begin{block}{\centering DiD using a single year as post-treatment period}
\vspace{-17pt} \flushright \hyperlink{Main5}{\beamerbutton{\textcolor{red}{Back}}}
\vspace{5pt}
\begin{thebibliography}{9}
\setbeamertemplate{bibliography item}[article]
\tiny \bibitem{A} Loyalka, P., Sylvia, S., Liu, C., Chu, J., and Shi, Y. (2019). Pay by design: Teacher
performance pay design and the distribution of student achievement. \textit{Journal of Labor
Economics}, 37(3):621-662.
\setbeamertemplate{bibliography item}[article]
\tiny \bibitem{B} Imberman, S. A., and Lovenheim, M. F. (2015). Incentive strength and teacher productivity: Evidence from a group-based teacher incentive pay system. \textit{Review of Economics and Statistics}, 97(2):364-386.
\setbeamertemplate{bibliography item}[article]
\tiny \bibitem{C} Springer, M. G., Pane, J. F., Le, V. N., McCaffrey, D. F., Burns, S. F., Hamilton, L. S., and Stecher, B. (2012). Team pay for performance: Experimental evidence from the round rock pilot project on team incentives. \textit{Educational Evaluation and Policy Analysis}, 34(4):367-390.
\setbeamertemplate{bibliography item}[article]
\tiny \bibitem{D} Muralidharan, K. and Sundararaman, V. (2011). Teacher performance pay: Experimental
evidence from India. \textit{Journal of Political Economy}, 119(1):39-77.
\setbeamertemplate{bibliography item}[article]
\tiny \bibitem{E} Lavy, V. (2009). Performance pay and teachers' effort, productivity, and grading ethics. \textit{American Economic Review}, 99(5):1979-2011.
\end{thebibliography}
\end{block}
\end{frame}



\section{EPF}

\begin{frame}[label=EPF]{Educational Production Function}
\vspace{-27pt} \flushright \hyperlink{Mirror}{\beamerbutton{\textcolor{red}{Mirror}}}
\vspace{-7pt}
\begin{table}
\begin{adjustbox}{max width=\textwidth} 
\begin{tabular}{l*{6}{cc}}
\toprule
& \multicolumn{3}{c}{\textbf{Mathematics}} &                      \multicolumn{3}{c}{\textbf{Portuguese}} \\            \cline{2-4}              \cline{5-7}                    
                    &\multicolumn{1}{c}{(1)}   &\multicolumn{1}{c}{(2)}   &\multicolumn{1}{c}{(3)}   &\multicolumn{1}{c}{(4)}   &\multicolumn{1}{c}{(5)}   &\multicolumn{1}{c}{(6)}   \\
                    &         OLS   &     FE-LSDV   &     IV-2SLS   &         OLS   &     FE-LSDV   &     IV-2SLS   \\
\midrule
\textbf{\emph{Student Characteristics}}&               &               &               &               &               &               \\
Male                &       0.011   &       0.000   &       0.000   &      -0.171***&       0.000   &       0.000   \\
Enjoy Learning      &       0.007   &       0.000   &       0.000   &      -0.024   &       0.000   &       0.000   \\
Make Homework       &       0.107***&       0.000   &       0.000   &       0.104***&       0.000   &       0.000   \\
Kindergarten Attendance&       0.114***&       0.000   &       0.000   &       0.074** &       0.000   &       0.000   \\
Change of School    &       0.048   &       0.000   &       0.000   &      -0.002   &       0.000   &       0.000   \\
\emph{Race}         &               &               &               &               &               &               \\
White               &       0.048   &       0.000   &       0.000   &       0.063   &       0.000   &       0.000   \\
Mixed               &      -0.004   &       0.000   &       0.000   &       0.017   &       0.000   &       0.000   \\
Black               &      -0.264***&       0.000   &       0.000   &      -0.195***&       0.000   &       0.000   \\
Asian               &       0.000   &       0.000   &       0.000   &       0.000   &       0.000   &       0.000   \\
Indigenous          &       0.101   &       0.000   &       0.000   &       0.045   &       0.000   &       0.000   \\
\emph{Education Mother}&               &               &               &               &               &               \\
Less then 4 years   &      -0.389***&       0.000   &       0.000   &      -0.413***&       0.000   &       0.000   \\
4 years             &      -0.308***&       0.000   &       0.000   &      -0.345***&       0.000   &       0.000   \\
8 years             &      -0.189***&       0.000   &       0.000   &      -0.237***&       0.000   &       0.000   \\
Secondary           &      -0.102   &       0.000   &       0.000   &      -0.132** &       0.000   &       0.000   \\
Tertiary            &       0.000   &       0.000   &       0.000   &       0.000   &       0.000   &       0.000   \vspace{-3pt}\\
\bottomrule
\vspace{-18pt} \\
\multicolumn{7}{l}{\tiny Note: Continued on next slide.}\\
\end{tabular}
\end{adjustbox}
\end{table}
\end{frame} 


\begin{frame}{Educational Production Function}
\vspace{-27pt} \flushright \hyperlink{Mirror}{\beamerbutton{\textcolor{red}{Mirror}}}
\vspace{15pt}
\begin{table}
\begin{adjustbox}{max width=\textwidth} 
\begin{tabular}{l*{6}{cc}}
\toprule
& \multicolumn{3}{c}{\textbf{Mathematics}} &                      \multicolumn{3}{c}{\textbf{Portuguese}} \\            \cline{2-4}              \cline{5-7}                    
                    &\multicolumn{1}{c}{(1)}   &\multicolumn{1}{c}{(2)}   &\multicolumn{1}{c}{(3)}   &\multicolumn{1}{c}{(4)}   &\multicolumn{1}{c}{(5)}   &\multicolumn{1}{c}{(6)}   \\
                    &         OLS   &     FE-LSDV   &     IV-2SLS   &         OLS   &     FE-LSDV   &     IV-2SLS   \\
\midrule
\textbf{\emph{Student Characteristics}}&               &               &               &               &               &               \\
\emph{Education Father}&               &               &               &               &               &               \\
Less then 4 years   &       0.000   &       0.000   &       0.000   &       0.000   &       0.000   &       0.000   \\
4 years             &       0.095** &       0.000   &       0.000   &       0.133***&       0.000   &       0.000   \\
8 years             &       0.176***&       0.000   &       0.000   &       0.217***&       0.000   &       0.000   \\
Secondary           &       0.260***&       0.000   &       0.000   &       0.350***&       0.000   &       0.000   \\
Tertiary            &       0.241***&       0.000   &       0.000   &       0.365***&       0.000   &       0.000   \\
\emph{Household Income}&               &               &               &               &               &               \\
Very Low            &      -0.441***&       0.000   &       0.000   &      -0.346***&       0.000   &       0.000   \\
Low                 &      -0.377***&       0.000   &       0.000   &      -0.336***&       0.000   &       0.000   \\
Medium              &      -0.259***&       0.000   &       0.000   &      -0.219***&       0.000   &       0.000   \\
High                &      -0.150** &       0.000   &       0.000   &      -0.122** &       0.000   &       0.000   \\
Very High           &       0.000   &       0.000   &       0.000   &       0.000   &       0.000   &       0.000   \\
\textbf{\emph{Lagged Scores}}&               &               &               &               &               &               \\
Mathematics         &     -          &              - &      -0.354***         &      -         &       -        &         -      \\
Portuguese          &      -         &       -        &     -          &      -         &        -       &              -0.355*** \\
\bottomrule
\vspace{-18pt} \\
\multicolumn{7}{l}{\tiny Note: Continued on next slide.}\\
\end{tabular}
\end{adjustbox}
\end{table}
\end{frame}


\begin{frame}{Educational Production Function}
\vspace{-27pt} \flushright \hyperlink{Mirror}{\beamerbutton{\textcolor{red}{Mirror}}}
\vspace{15pt}
\begin{table}
\begin{adjustbox}{max width=\textwidth} 
\begin{tabular}{l*{6}{cc}}
\toprule
& \multicolumn{3}{c}{\textbf{Mathematics}} &                      \multicolumn{3}{c}{\textbf{Portuguese}} \\            \cline{2-4}              \cline{5-7}                    
                    &\multicolumn{1}{c}{(1)}   &\multicolumn{1}{c}{(2)}   &\multicolumn{1}{c}{(3)}   &\multicolumn{1}{c}{(4)}   &\multicolumn{1}{c}{(5)}   &\multicolumn{1}{c}{(6)}   \\
                    &         OLS   &     FE-LSDV   &     IV-2SLS   &         OLS   &     FE-LSDV   &     IV-2SLS   \vspace{-3pt}\\
\midrule 
\textbf{\emph{Teacher Characteristics}}&               &               &               &               &               &               \\
Male                &       0.219***&       0.031   &       0.038   &       0.261***&      -0.067   &       0.048    \\
\emph{Education Level}&               &               &               &               &               &               \vspace{-2pt} \\
Less than secondary &       0.000   &      -0.118   &       0.299   &       0.000   &      -0.216   &      -0.121   \vspace{-2pt} \\
Secondary           &      -0.064   &      -0.227*  &       0.235   &       0.160   &      -0.130   &       0.126   \vspace{-2pt} \\
Vocational          &      -0.322   &       0.036   &       0.021   &      -0.165   &       0.036   &       0.156   \vspace{-2pt} \\
Tertiary            &      -0.235   &       0.044   &       0.150*  &      -0.064   &       0.060   &       0.158   \vspace{-2pt} \\
Master              &      -0.005   &       0.000   &       0.000   &       0.146   &       0.000   &       0.000   \vspace{-2pt} \\
Doctorate           &       0.000   &       0.000   &       0.000   &       0.000   &       0.000   &       0.000    \\
\emph{Years of Experience}&               &               &               &               &               &               \vspace{-2pt} \\
Less than 1         &       0.000   &       0.010   &      -0.421***&       0.000   &       0.181** &       0.185*** \vspace{-2pt}\\
1 - 2               &      -0.110   &      -0.182*  &      -0.222*  &      -0.323   &      -0.083   &      -0.389*** \vspace{-2pt}\\
3 - 4               &      -0.161   &      -0.033   &      -0.092*  &      -0.189   &       0.063   &       0.077   \vspace{-2pt} \\
5 - 10              &      -0.173   &      -0.023   &      -0.079*  &      -0.245   &       0.017   &       0.014   \vspace{-2pt}\\
11 - 15             &      -0.151   &       0.003   &      -0.019   &      -0.289   &      -0.047*  &      -0.010   \vspace{-2pt}\\
More than 15        &      -0.131   &       0.000   &       0.000   &      -0.252   &       0.000   &       0.000   \\
\emph{Schools that work}&               &               &               &               &               &               \vspace{-2pt} \\
Only One            &      -0.147   &       0.060   &      -0.063   &       0.038   &       0.076   &      -0.119   \vspace{-2pt}\\
Two                 &      -0.187   &       0.043   &      -0.043   &       0.000   &       0.070   &      -0.115   \vspace{-2pt}\\
Three or more       &       0.000   &       0.000   &       0.000   &       0.134   &       0.000   &       0.000   \vspace{-2pt}\\
\bottomrule
\vspace{-18pt} \\
\multicolumn{7}{l}{\tiny Note: Continued on next slide.}\\
\end{tabular}
\end{adjustbox}
\end{table}
\end{frame}




\begin{frame}{Educational Production Function}
\vspace{-27pt} \flushright \hyperlink{Mirror}{\beamerbutton{\textcolor{red}{Mirror}}}
\vspace{-7pt}
\begin{table}
\begin{adjustbox}{max width=\textwidth} 
\begin{tabular}{l*{6}{cc}}
\toprule
& \multicolumn{3}{c}{\textbf{Mathematics}} &                      \multicolumn{3}{c}{\textbf{Portuguese}} \\            \cline{2-4}              \cline{5-7}                    
                    &\multicolumn{1}{c}{(1)}   &\multicolumn{1}{c}{(2)}   &\multicolumn{1}{c}{(3)}   &\multicolumn{1}{c}{(4)}   &\multicolumn{1}{c}{(5)}   &\multicolumn{1}{c}{(6)}   \\
                    &         OLS   &     FE-LSDV   &     IV-2SLS   &         OLS   &     FE-LSDV   &     IV-2SLS   \vspace{-2pt}\\
\midrule
\textbf{\emph{Teacher Characteristics}}&               &               &               &               &               &               \\ 
\emph{Age}          &               &               &               &               &               &               \\
Up to 24            &       0.000   &       0.011   &       0.095   &       0.000   &      -0.142** &      -0.115   \\
25 - 29             &      -0.033   &      -0.045   &       0.090   &       0.063   &      -0.172***&      -0.072   \\
30 - 39             &      -0.044   &      -0.042   &       0.008   &       0.093   &      -0.113***&      -0.060   \\
40 - 49             &       0.045   &      -0.017   &       0.046   &       0.226***&      -0.090***&      -0.059   \\
50 - 54             &      -0.145   &      -0.143***&      -0.044   &       0.035   &      -0.179***&      -0.131** \\
More than 55        &       0.006   &       0.000   &       0.000   &       0.220***&       0.000   &       0.000   \\
\emph{Weekly teaching hours}&               &               &               &               &               &               \\
Up to 20            &       0.000   &      -0.023   &      -0.029   &      -0.137   &      -0.081   &      -0.108   \\
21 - 25             &       0.116   &      -0.036   &      -0.010   &       0.014   &      -0.073*  &      -0.091   \\
26 - 30             &       0.168** &      -0.011   &       0.033   &       0.036   &      -0.111***&      -0.098   \\
31 - 40             &       0.064   &      -0.092** &      -0.038   &       0.002   &      -0.122***&      -0.117   \\
More than 41        &       0.118   &       0.000   &       0.000   &       0.000   &       0.000   &       0.000   \\
\emph{Other Job}    &               &               &               &               &               &               \\
No                  &       0.134   &       0.012   &      -0.098   &       0.156*  &       0.023   &       0.026   \\
Yes, in education&       0.076   &       0.005   &      -0.143** &       0.103   &       0.081*  &       0.032   \\
Yes, outside education&       0.000   &       0.000   &       0.000   &       0.000   &       0.000   &       0.000   \vspace{-3pt}\\
\bottomrule
\vspace{-18pt} \\
\multicolumn{7}{l}{\tiny Note: Continued on next slide.}\\
\end{tabular}
\end{adjustbox}
\end{table}
\end{frame}


\begin{frame}{Educational Production Function}
\vspace{-27pt} \flushright \hyperlink{Mirror}{\beamerbutton{\textcolor{red}{Mirror}}}
\vspace{-7pt}
\begin{table}
\begin{adjustbox}{max width=\textwidth} 
\begin{tabular}{l*{6}{cc}}
\toprule
& \multicolumn{3}{c}{\textbf{Mathematics}} &                      \multicolumn{3}{c}{\textbf{Portuguese}} \\            \cline{2-4}              \cline{5-7}                    
                    &\multicolumn{1}{c}{(1)}   &\multicolumn{1}{c}{(2)}   &\multicolumn{1}{c}{(3)}   &\multicolumn{1}{c}{(4)}   &\multicolumn{1}{c}{(5)}   &\multicolumn{1}{c}{(6)}   \\
                    &         OLS   &     FE-LSDV   &     IV-2SLS   &         OLS   &     FE-LSDV   &     IV-2SLS   \\
\midrule 
\textbf{\emph{School Characteristics}}&               &               &               &               &               &               \\
Library             &      -0.029   &       0.000   &       0.000   &      -0.094   &       0.000   &       0.000   \\
Computer lab        &      -0.142** &       0.000   &       0.000   &      -0.118** &       0.000   &       0.000   \\
Science lab         &       0.729***&       0.000   &       0.000   &       0.748***&       0.000   &       0.000   \\
Sports court        &      -0.038   &       0.000   &       0.000   &       0.027   &       0.000   &       0.000   \\
Art room            &       0.218***&       0.000   &       0.000   &       0.182***&       0.000   &       0.000   \\
Intimidation of students&      -0.061   &      -0.035   &      -0.040   &       0.010   &       0.015   &       0.004   \\
Intimidation of employees&      -0.029   &       0.010   &      -0.016   &      -0.036   &      -0.043*  &      -0.029   \\
Violence against students&      -0.109** &      -0.034   &      -0.096** &      -0.096** &      -0.030   &      -0.000   \\
Violence against employees &       0.003   &       0.021   &       0.094** &      -0.032   &       0.039   &       0.011   \\
Depredation         &       0.007   &       0.027   &       0.050   &      -0.063   &      -0.027   &       0.031   \\
Drug use            &      -0.041   &      -0.026   &      -0.016   &      -0.004   &       0.064   &       0.025   \\
Interference of drug gangs&       0.051   &       0.027   &       0.018   &       0.011   &      -0.048   &      -0.017 \vspace{-2pt}  \\
 \hline
Observations        &        6,461   &        7,985   &        3,127   &        6,489   &        8,010   &        3,211   \\
R-square            &       0.199   &       0.819   &      -   &       0.230   &       0.818   &      -   \vspace{-2pt} \\
\hline Individual FE&          No   &         Yes   &         Yes   &          No   &         Yes   &         Yes   \\
Time FE             &          No   &         Yes   &         Yes   &          No   &         Yes   &         Yes   \\
School FE           &          No   &         Yes   &         Yes   &          No   &         Yes   &         Yes \\
\bottomrule
\vspace{-18pt} \\
\multicolumn{7}{l}{\tiny Notes: Dependent variable is student performance (test scores). \( * p<0.1, ** p<0.05, *** p<0.01 \).} \vspace{-8pt}\\
\multicolumn{7}{l}{\tiny Source: GERES database (2005-2008), own estimates.}\\
\end{tabular}
\end{adjustbox}
\end{table}
\end{frame}



\section{Tests}

\begin{frame}{Example of Question for Portuguese}
\vspace{-0.5cm}
\begin{figure}[htb]
\vspace{-0.5cm}
\centering
\includegraphics[width=1.02\textwidth,height=0.8\textheight]{Figure/TestPor1}
\end{figure}
     \vspace{-1.72cm}  \hspace{-1pt}
\begin{minipage}{1\textwidth} 
{{\fontsize{4}{4}\selectfont
Note: Question number 1 of wave 1 for children in the first year of primary education. \\  
Source: GERES database (2005-2008).\par}}
\end{minipage} 
\end{frame}

\begin{frame}{Example of Question for Portuguese}
\vspace{-0.5cm}
\begin{figure}[htb]
\vspace{-0.5cm}
\centering
\includegraphics[width=1.02\textwidth,height=0.8\textheight]{Figure/TestPor2}
\end{figure}
     \vspace{-1.72cm}  \hspace{-1pt}
\begin{minipage}{1\textwidth} 
{{\fontsize{4}{4}\selectfont
Note: Question number 2 of wave 1 for children in the first year of primary education. \\  
Source: GERES database (2005-2008).\par}}
\end{minipage} 
\end{frame}

\begin{frame}{Example of Question for Mathematics}
\vspace{-0.5cm}
\begin{figure}[htb]
\vspace{-0.5cm}
\centering
\includegraphics[width=1.02\textwidth,height=0.8\textheight]{Figure/TestMat1}
\end{figure}
     \vspace{-1.72cm}  \hspace{-1pt}
\begin{minipage}{1\textwidth} 
{{\fontsize{4}{4}\selectfont
Note: Question number 4 of wave 1 for children in the first year of primary education. \\  
Source: GERES database (2005-2008).\par}}
\end{minipage} 
\end{frame}

\begin{frame}{Example of Question for Mathematics}
\vspace{-0.5cm}
\begin{figure}[htb]
\vspace{-0.5cm}
\centering
\includegraphics[width=1.02\textwidth,height=0.8\textheight]{Figure/TestMat2}
\end{figure}
     \vspace{-1.72cm}  \hspace{-1pt}
\begin{minipage}{1\textwidth} 
{{\fontsize{4}{4}\selectfont
Note: Question number 5 of wave 1 for children in the first year of primary education. \\  
Source: GERES database (2005-2008).\par}}
\end{minipage} 
\end{frame}


\end{document}