\documentclass{beamer} 
\usetheme{Madrid}
\usecolortheme{beaver} % or try albatross, beaver, crane, ...
\usepackage{amsmath, amsfonts, amssymb}
\usepackage{amsmath, amsfonts, amssymb}
\usepackage{float}
\usepackage{xcolor}
\usepackage{graphicx}
\usepackage{adjustbox}
\usepackage{indentfirst}
\usepackage{booktabs}
\usepackage{tabu}
\usepackage{scalefnt}
\usepackage{pdflscape}
\usepackage{draftwatermark}
\usepackage{rotating}
\usepackage[belowskip=-15pt,aboveskip=0pt]{caption}
\usepackage{subcaption}
\usepackage{adjustbox}
\usepackage{natbib} [options]
\usepackage{fixmath}
\usepackage[utf8]{inputenc}
\usepackage{xparse}
\usepackage{varwidth}
\usepackage{breqn}
\usepackage{mathtools}
\usepackage{adjustbox}
\usepackage{pdflscape,array,booktabs}
\usepackage{wrapfig}
\usepackage{attachfile}
\usepackage{booktabs,multirow}
\usepackage{color}
\usepackage{appendixnumberbeamer}
\usepackage{hyperref}
\usepackage[printwatermark]{xwatermark}
\usepackage{tikz}
\usepackage{lipsum}
\usepackage{appendixnumberbeamer}
\usepackage{mathtools}
\usepackage{verbatim}
\usepackage{dcolumn}





\makeatletter
\setbeamertemplate{footline}
{
  \leavevmode%
  \hbox{%
  \begin{beamercolorbox}[wd=.30\paperwidth,ht=2.25ex,dp=1ex,center]{author in head/foot}%
    \usebeamerfont{author in head/foot}\insertshortauthor~~\beamer@ifempty{\insertshortinstitute}{}{(\insertshortinstitute)}
  \end{beamercolorbox}%
  
  \begin{beamercolorbox}[wd=.43\paperwidth,ht=2.25ex,dp=1ex,center]{title in head/foot}%
    \usebeamerfont{title in head/foot}\insertshorttitle
  \end{beamercolorbox}%
  
  \begin{beamercolorbox}[wd=.27\paperwidth,ht=2.25ex,dp=1ex,right]{date in head/foot}%
    \usebeamerfont{date in head/foot}\insertshortdate{}\hspace*{2em}
%    \insertframenumber{} / \inserttotalframenumber\hspace*{2ex} % DELETED
  \end{beamercolorbox}}%
  \vskip0pt%
}
\makeatother




\newsavebox\mybox
%\savebox\mybox{\tikz[color=red,opacity=0.5]\node{Do not circulate};}
\newwatermark*[
  allpages,
  angle=45,
  scale=4.5,
  xpos=-20,
  ypos=15
]{\usebox\mybox}


\expandafter\def\expandafter\insertshortdate\expandafter{%
  \insertshortdate\hfill\insertframenumber\,/\,\inserttotalframenumber}





\usepackage{tikz}
\usetikzlibrary{shapes}
\newcommand*{\circled}[2][red]{
  \tikz[baseline=(char.base)]{
              \node[shape=ellipse,inner sep=2pt,
                draw=#1,
             ] (char) {#2};}
}
\newcommand{\markcells}[3][green]{
  \tikz[baseline=(char.mid)]{\node[shape=ellipse,overlay,draw,#1]{\phantom{\rule{#2}{#3}}};}%
}


\newcommand{\Rho}{\mathrm{P}}
\newcommand{\Chi}{\mathrm{X}}


\captionsetup[table]{position=top,skip=7pt} 







\title[\tiny The Geography of Intergenerational Mobility in Brazil]{The Geography of Intergenerational Mobility}
\subtitle{\footnotesize Evidence of educational persistence and the ``Great Gatsby Curve" in Brazil}



\author [Tharcisio Leone]{ \\ \textbf {Tharcisio Leone} \\ \footnotesize PhD Candidate in Economics}


\institute[GIGA/FU-Berlin] 
{
  German Institute of Global and Area Studies (GIGA)\\
  $\&$ Free University of Berlin\\
    } 
  
 
\vspace{5.5cm}  
\date[DIAL Mid-term Conference 2019]{\tiny DIAL Mid-term Conference 2019 \\ Turku, Finland \\ June 6, 2019}



\begin{document}


\begin{frame}
  \titlepage
\end{frame}



\AtBeginSection[]
{
  \begin{frame}<beamer>
    \frametitle{Outline}
    \tableofcontents[currentsection,currentsubsection]
  \end{frame}
}

% Uncomment these lines for an automatically generated outline.
\begin{frame}{Outline}
  \tableofcontents
\end{frame}








\section{Motivation} 
\begin{frame}{Motivation}
\begin{figure}[htb]
\centering
\includegraphics[scale=0.37]{Figure/GGC-Corak2}
\end{figure}
\end{frame}



\begin{frame}{Income inequality across states}
\begin{figure}[htb]
\centering
%\captionsetup{justification=centering,margin=2cm}
%caption{\textbf{Income inequality across states}}
      \includegraphics[scale=0.7]{Figure/OutcomeShare}
     \label{fig:OutcomeShare}
     \begin{minipage}{0.83\textwidth} % choose width suitably
{\fontsize{5}{1}\selectfont
\vspace*{-3mm}
Note: Estimations based on self declared per capita domiciliary income. \\
Source: PNAD-2014, own estimates.\par}
\end{minipage}
\end{figure}     
\end{frame}



\begin{frame}{Research questions}
\begin{enumerate}
\item Is there a variation in intergenerational educational mobility across Brazilian states?
\hfill \break 
\item Does the “Great Gatsby Curve” also hold true within a single country?		
\hfill \break 
\item How higher income inequality leads to lower rate of mobility?
  		\begin{itemize}
			\item \footnotesize Investigation of one specific mechanism behind the correlation between inequality and mobility: School dropout rate.
			\end{itemize}
\end{enumerate}
\end{frame}



\section{Data}
\begin{frame}[label=main4]{Data}
\begin{enumerate}
  		\item Brazilian National Household Sample Survey (PNAD)
  		\begin{itemize}
			\item \footnotesize Mobility supplement 2014 ($46,051$ individuals)
			\end{itemize}
			    \item Main variables for the investigation
  		\begin{itemize}
  		\item \footnotesize Number of years of schooling
  		 \item \footnotesize Level of education
  		%\item \footnotesize School Dropout rate
  		 \item \footnotesize Income inequality
  		 \end{itemize}
  		 	 
  		\item Variable construction
  		\begin{itemize}
  		\item \footnotesize Only individuals born between 1940 and 1989
  		\item \footnotesize The most educated parent
  		\item \footnotesize 75/10 Ratio for income inequality 
  		%\vspace{-0.1cm}
  		%\\ \tiny\textcolor{red}{$\Longrightarrow$} Income Inequality as a Time-Variant Covariate \hyperlink{Harmonsation}{\beamerbutton {\textcolor{red}{Harmonsation of PNAD-Data}}}
  		\item \footnotesize Economic Marginalization \\ \tiny\textcolor{red}{$\Longrightarrow$} Children from parents with no primary education
  		  		\end{itemize}
  		\item Further variables
  		\begin{itemize}
  		\item \footnotesize Individual characteristics \\ \tiny\textcolor{red}{$\Longrightarrow$} Gender, birth cohort, race, locality of residence and living with both parents at age 15.
  		\end{itemize}
  		\end{enumerate}
\end{frame}





\section{Estimating intergenerational mobility}

\begin{frame} {Estimating intergenerational mobility}
\begin{block}{1. Transition matrix}
\begin{itemize}
\item \footnotesize Probability of children from parents with the educational attainment $j$ to achieve the education level $i$.
\item \footnotesize Provide a overview about the direction of the mobility.
\end{itemize}
\end{block}
\hfill \break 
\begin{block}{2. Linear regression model}
\begin{itemize}
\item \footnotesize Summarize the grade of persistence between parents' and children's educational attainment.
\item \footnotesize Take into account the changes over time in the education inequality.
\end{itemize}
\end{block}
\end{frame}


\newcounter{saveenumi}
\newcommand{\seti}{\setcounter{saveenumi}{\value{enumi}}}
\newcommand{\conti}{\setcounter{enumi}{\value{saveenumi}}}
\resetcounteronoverlays{saveenumi}




\begin{frame} {Transition matrices}
\begin{enumerate}
\item{Classification of educational outcomes}
{\\ \footnotesize \textcolor{red}{$\Longrightarrow$} Education of children (generation $t$) and parents (generation $t+1$) into four categories: no school certificate, primary, secondary and tertiary education.}

\vspace{0.3cm}
%\hfill \break 
\item{Estimation of matrices}

{\footnotesize \textcolor{red}{$\Longrightarrow$} A trasition matrix is a process $\{X_0, X_1, X_2, ...\}$ with number of states $S$, where $S$ has size $\mathbb{R}$ (possibly infinite) such that:
\begin{equation} \label{eq:Notation}
p_{ij} = \mathbb{P} (X_{t+1} = j \mid X_t = i) \quad \textrm{for} \quad i, j \in S, \quad t = 0,1,2,...
\end{equation}}



\vspace{-0.2cm}
\item Measure of intergenerational mobility 
\vspace{0cm}
{\\ \footnotesize\textcolor{red}{$\Longrightarrow$} Upward and Downward Mobility
\vspace{0cm}
\begin{equation} \label{eq:UpwardMobility}
UpM = Pr \left(X_t > l \mid X_{t+1} = l \right) \quad \textrm{and} \quad DoM = Pr \left(X_t < l \mid X_{t+1} = l \right)
\end{equation}}

\end{enumerate}
\end{frame}




\begin{frame} {Descendants predicted propabilities of education}
\begin{figure}[htb]
\centering
\includegraphics[scale=0.72]{Figure/Evolution-Educ}
\begin{minipage}{0.84\textwidth} % choose width suitably
{\fontsize{5}{1}\selectfont
\vspace*{-4mm}
Notes: Estimations for boys and girls. Parents's schooling refers to the educational attainment of the better educated parent.\\
Source: PNAD-2014, own estimates.\par}
\end{minipage}
\end{figure}
\end{frame}



\begin{frame} {Immobility ratio, Upward and Downward mobility}
\begin{figure}[htb]
\centering
\includegraphics[scale=0.7]{Figure/IMob-Up-Down}
\begin{minipage}{0.82\textwidth} % choose width suitably
{\fontsize{5}{1}\selectfont 
\vspace*{-2mm}
Note: Downward (upward) mobility presents the share of children which have achieved a lower (higher) level of education than their most educated parent.\\
Source: PNAD-2014, own estimates.\par}
\end{minipage}
\end{figure}
\end{frame}




\begin{frame} {Linear regression model}
\begin{itemize}
{\footnotesize\item The association between the education of children and parents is given by:}

\begin{equation} \label{eq:OLS}
educ^c_{is}=\alpha+\beta \: educ^p_{is}+\epsilon_i \quad \textrm{for} \quad i=1,2,...N 
\end{equation}

\hfill \break
{\footnotesize\item Normalisation of coefficient $\beta$ by the corresponding standard deviation:}

\begin{equation} \label{eq:Beta}
\hat{\beta} = \rho_s^{cp} \: \frac{\sigma^p_s}{\sigma^c_s}, \qquad \textrm{with} \qquad \sigma=\sqrt{\frac{1}{N}\sum_{i=1}^{N}(x_1-\mu)^2}
\end{equation}
\end{itemize}
\end{frame}


\begin{frame} {Linear regression model}
\begin{itemize} 
{\footnotesize\item From equations \eqref{eq:OLS} and \eqref{eq:Beta} the resulting model can be summarized as:}


\begin{equation} \label{eq:OLSRho}
\scriptsize\frac{educ^c_{is}}{\sigma^c_s}=\delta \:+\: \rho \left(\frac{educ^p_{is}}{\sigma^p_s}\right) \:+\: \epsilon_i \qquad \textrm{with} \qquad \rho \in [0,1] 
\end{equation}

\hfill \break
{\footnotesize\item Assuming a educational production function:}

\begin{equation} \label{eq:OLSFixed}
\scriptsize\frac{educ^c_{is}}{\sigma^c_s} \:=\: \delta \:+\: \rho \: \frac{educ^p_{is}}{\sigma^p_s} \:+\: \eta \left(\frac{educ^p_{is}}{\sigma^p_s} \times UF_i \right) \:+\: \lambda \: UF_i \:+\: \gamma \left(X_i \times UF_i \right) \:+\: \epsilon_{is}
\end{equation}


{\scriptsize with:}
\begin{itemize}
{\scriptsize\item Dummy variables $UF$ present the state of residence}
{\scriptsize\item Control variables: Gender, year of birth, locality of residence (rural/urban), race and living with both parents at age of 15.} 
\end{itemize}
\end{itemize}
\end{frame}






\begin{frame} [label=main2]{Intergenerational persistence in education \hyperlink{Evolution}{\beamerbutton {\textcolor{red}{Evolution over Time}}}}
\begin{figure}[htb]
\centering
\includegraphics[scale=0.7]{Figure/RhoS-1940-1989-All}
\begin{minipage}{0.84\textwidth} % choose width suitably
{\fontsize{5}{1}\selectfont
\vspace*{-1mm} 
Note: The closer the estimated value is to one, the stronger the association between parents' and children's educational attainment and consequently, the lower the intergenerational mobility. \\
Source: PNAD-2014, own estimates.\par}
\end{minipage}
\end{figure}
\end{frame}




\section{The Great Gatsby Curve}



\begin{frame} {The Great Gatsby Curve}
\begin{figure}[htb]
\centering
\includegraphics[scale=0.7]{Figure/GGG-1970-1979}
\begin{minipage}{0.82\textwidth} % choose width suitably
{\fontsize{5}{1}\selectfont
\vspace*{-4mm}  
Notes: r = Pearson's correlation. Asterisk be printed for a correlation coefficient with p-values of $.1$ or lower. \\ Source: PNAD-2014, own estimates.\par}
\end{minipage}
\end{figure}
\end{frame}










\section{Linking Inequality and school dropouts}

\begin{frame} [label=main3]{Linking Inequality and school dropouts}
\begin{block}{Interim results}
\begin{itemize}
\item \footnotesize The level of intergenerational mobility ranges across Brazilian states.
\item \footnotesize The ``Great Gatsby Curve" holds true within a single country. \\ \tiny \textcolor{red}{$\Longrightarrow$} States with greater income disparity tend to have lower levels of education mobility between generations.
\end{itemize}
\end{block}

\hfill \break
\begin{block}{Next steps}
\begin{itemize}
\item \footnotesize Change from an analysis of intergenerational mobility via correlation hypothesis to an investigation of the determinats.
\item \footnotesize The link between inequality and school dropout rate.
\item \footnotesize Effects of Economic Marginalization:
\\ \tiny \textcolor{red}{$\Longrightarrow$} Children from parents with no primary education.
\\ \tiny \textcolor{red}{$\Longrightarrow$} Individuals with very low expected earnings premium.

\item \footnotesize Marginalization arises by increasing income inequality. \hyperlink{KearneyLevine}{\beamerbutton {\textcolor{red}{Ex-ante Model}}}
\end{itemize}
\end{block}
\end{frame}




\begin{frame}[label=main5]{Linking Inequality and school dropouts}
\begin{block}{Research question}
\begin{itemize}
\item \footnotesize Have children from low socioeconomic backgrounds who live in states with high income inequality levels a lower chance to conclude secondary education?
\end{itemize}
\end{block}
\begin{itemize}
\item \footnotesize The empirical probit model can be write as:
\end{itemize}
\begin{footnotesize}



\begin{equation} \label{eq:EduOutcome}
\begin{split}
ComSec_{i,t}=\pi_0 \:+\: \pi_1 \left(MSB_{i} \times Ineq_{s,t+14} \right) \:+\: \pi_2 \: MSB_{i}  \:+\: \pi_3 \: Ineq_{s,t+14} \\ \:+\: \gamma_1 \: male_{i} \:+\: \gamma_2 \: rural_{i} \:+\: \gamma_3 \: bothP_{i} \:+\: \gamma_4 \: race_{i} \:+\: \gamma_5 \: birth_{i} \:+\: \epsilon_{i} 
\end{split}
\end{equation}
\end{footnotesize}

{\scriptsize where:}
\begin{itemize}
{\scriptsize\item \textit{ComSec}:
Concluded secondary education.}
{\scriptsize\item \textit{MSB:} Dummy variable indicating children from parents with no primary education.}
\vspace{-0.1cm}
\\ \tiny \textcolor{red}{$\Longrightarrow$} Proxy for socioeconomic marginalization
{\scriptsize\item \textit{Ineq}: Ratio 75/10 of income distribution. \hyperlink{Harmonsation}{\beamerbutton {\textcolor{red}{Harmonsation of PNAD-Data}}}}
\\ \tiny \textcolor{red}{$\Longrightarrow$} Harmonisation of PNAD-Data (1981-2003)
\\ \tiny \textcolor{red}{$\Longrightarrow$} Inequality from the year in which the individuals concluded the compulsory schooling $(t+14)$.
\end{itemize}
\end{frame}





\begin{frame}[label=main6]{\scriptsize Table 1: The impact of Inequality on Educational Attainment \hyperlink{Robustness}{\beamerbutton {\textcolor{red}{Robustness Checks}}}}
% Table generated by Excel2LaTeX from sheet 'Latex'
\begin{table}[htbp]
  \centering
%  \captionsetup{justification=centering,margin=1cm}
%\caption{\textbf{The impact of Inequality on Educational Attainment.}}
\vspace*{-2mm}
  \begin{adjustbox}{max width=\textwidth}
    \begin{tabular}{lllllllllllll}
    \toprule
    \multicolumn{1}{c}{\textbf{Birth Cohort}} &       & \multicolumn{1}{c}{\textbf{All}} &       & \multicolumn{1}{c}{\textbf{1965}} &       & \multicolumn{1}{c}{\textbf{1970}} &       & \multicolumn{1}{c}{\textbf{1975}} &       & \multicolumn{1}{c}{\textbf{1980}} &       & \multicolumn{1}{c}{\textbf{1985}} \\
    \multicolumn{1}{c}{Birth years} &       & \multicolumn{1}{c}{(1965-1989)} &       & \multicolumn{1}{c}{(1965-1969)} &       & \multicolumn{1}{c}{(1970-1974)} &       & \multicolumn{1}{c}{(1975-1979)} &       & \multicolumn{1}{c}{(1980-1984)} &       & \multicolumn{1}{c}{(1985-1989)} \\
\cmidrule{3-3}\cmidrule{5-5}\cmidrule{7-7}\cmidrule{9-9}\cmidrule{11-11}\cmidrule{13-13}    Socio-economic Marginalization \# Inequality &       & \multicolumn{1}{c}{-0.0542***} &       & \multicolumn{1}{c}{-0.0486} &       & \multicolumn{1}{c}{-0.0353} &       & \multicolumn{1}{c}{-0.00346} &       & \multicolumn{1}{c}{-0.0982**} &       & \multicolumn{1}{c}{-0.0519} \\
          &       & \multicolumn{1}{c}{(0.0186)} &       & \multicolumn{1}{c}{(0.0644)} &       & \multicolumn{1}{c}{(0.0473)} &       & \multicolumn{1}{c}{(0.0335)} &       & \multicolumn{1}{c}{(0.0486)} &       & \multicolumn{1}{c}{(0.0462)} \\
    Socio-economic Marginalization &       & \multicolumn{1}{c}{-0.405***} &       & \multicolumn{1}{c}{-0.535} &       & \multicolumn{1}{c}{-0.492*} &       & \multicolumn{1}{c}{-0.705***} &       & \multicolumn{1}{c}{-0.261} &       & \multicolumn{1}{c}{-0.270} \\
          &       & \multicolumn{1}{c}{(0.104)} &       & \multicolumn{1}{c}{(0.350)} &       & \multicolumn{1}{c}{(0.287)} &       & \multicolumn{1}{c}{(0.208)} &       & \multicolumn{1}{c}{(0.251)} &       & \multicolumn{1}{c}{(0.237)} \\
    Inequality &       & \multicolumn{1}{c}{0.0129} &       & \multicolumn{1}{c}{-0.0256} &       & \multicolumn{1}{c}{0.0387} &       & \multicolumn{1}{c}{-0.0164} &       & \multicolumn{1}{c}{0.0929***} &       & \multicolumn{1}{c}{-0.0170} \\
          &       & \multicolumn{1}{c}{(0.0138)} &       & \multicolumn{1}{c}{(0.0503)} &       & \multicolumn{1}{c}{(0.0377)} &       & \multicolumn{1}{c}{(0.0242)} &       & \multicolumn{1}{c}{(0.0327)} &       & \multicolumn{1}{c}{(0.0320)} \\
    Male  &       & \multicolumn{1}{c}{-0.203***} &       & \multicolumn{1}{c}{-0.0988*} &       & \multicolumn{1}{c}{-0.114**} &       & \multicolumn{1}{c}{-0.217***} &       & \multicolumn{1}{c}{-0.291***} &       & \multicolumn{1}{c}{-0.257***} \\
          &       & \multicolumn{1}{c}{(0.0208)} &       & \multicolumn{1}{c}{(0.0531)} &       & \multicolumn{1}{c}{(0.0487)} &       & \multicolumn{1}{c}{(0.0455)} &       & \multicolumn{1}{c}{(0.0425)} &       & \multicolumn{1}{c}{(0.0445)} \\
    Rural &       & \multicolumn{1}{c}{-0.650***} &       & \multicolumn{1}{c}{-0.575***} &       & \multicolumn{1}{c}{-0.812***} &       & \multicolumn{1}{c}{-0.784***} &       & \multicolumn{1}{c}{-0.687***} &       & \multicolumn{1}{c}{-0.490***} \\
          &       & \multicolumn{1}{c}{(0.0333)} &       & \multicolumn{1}{c}{(0.0854)} &       & \multicolumn{1}{c}{(0.0831)} &       & \multicolumn{1}{c}{(0.0723)} &       & \multicolumn{1}{c}{(0.0674)} &       & \multicolumn{1}{c}{(0.0692)} \\
    Living with both parent &       & \multicolumn{1}{c}{0.0830***} &       & \multicolumn{1}{c}{-0.0687} &       & \multicolumn{1}{c}{0.0312} &       & \multicolumn{1}{c}{0.0614} &       & \multicolumn{1}{c}{0.0505} &       & \multicolumn{1}{c}{0.256***} \\
          &       & \multicolumn{1}{c}{(0.0248)} &       & \multicolumn{1}{c}{(0.0683)} &       & \multicolumn{1}{c}{(0.0601)} &       & \multicolumn{1}{c}{(0.0537)} &       & \multicolumn{1}{c}{(0.0497)} &       & \multicolumn{1}{c}{(0.0494)} \\
    Birth year &       & \multicolumn{1}{c}{0.0159***} &       & \multicolumn{1}{c}{0.00722} &       & \multicolumn{1}{c}{-0.00521} &       & \multicolumn{1}{c}{0.0157} &       & \multicolumn{1}{c}{0.0116} &       & \multicolumn{1}{c}{-0.0373**} \\
          &       & \multicolumn{1}{c}{(0.00156)} &       & \multicolumn{1}{c}{(0.0184)} &       & \multicolumn{1}{c}{(0.0171)} &       & \multicolumn{1}{c}{(0.0161)} &       & \multicolumn{1}{c}{(0.0155)} &       & \multicolumn{1}{c}{(0.0158)} \\
    White (reference) &       & \multicolumn{1}{c}{-} &       & \multicolumn{1}{c}{-} &       & \multicolumn{1}{c}{-} &       & \multicolumn{1}{c}{-} &       & \multicolumn{1}{c}{-} &       & \multicolumn{1}{c}{-} \\
    Black &       & \multicolumn{1}{c}{-0.160***} &       & \multicolumn{1}{c}{-0.189**} &       & \multicolumn{1}{c}{-0.217**} &       & \multicolumn{1}{c}{-0.0887} &       & \multicolumn{1}{c}{-0.184**} &       & \multicolumn{1}{c}{-0.152*} \\
          &       & \multicolumn{1}{c}{(0.0366)} &       & \multicolumn{1}{c}{(0.0927)} &       & \multicolumn{1}{c}{(0.0872)} &       & \multicolumn{1}{c}{(0.0801)} &       & \multicolumn{1}{c}{(0.0730)} &       & \multicolumn{1}{c}{(0.0787)} \\
    Mixed (white/black) &       & \multicolumn{1}{c}{-0.271***} &       & \multicolumn{1}{c}{-0.374***} &       & \multicolumn{1}{c}{-0.305***} &       & \multicolumn{1}{c}{-0.302***} &       & \multicolumn{1}{c}{-0.256***} &       & \multicolumn{1}{c}{-0.166***} \\
          &       & \multicolumn{1}{c}{(0.0222)} &       & \multicolumn{1}{c}{(0.0568)} &       & \multicolumn{1}{c}{(0.0521)} &       & \multicolumn{1}{c}{(0.0489)} &       & \multicolumn{1}{c}{(0.0455)} &       & \multicolumn{1}{c}{(0.0476)} \\
    Asian &       & \multicolumn{1}{c}{0.296*} &       & \multicolumn{1}{c}{0.635} &       & \multicolumn{1}{c}{0.938***} &       & \multicolumn{1}{c}{0.356} &       & \multicolumn{1}{c}{-0.413} &       & \multicolumn{1}{c}{0.130} \\
          &       & \multicolumn{1}{c}{(0.159)} &       & \multicolumn{1}{c}{(0.387)} &       & \multicolumn{1}{c}{(0.353)} &       & \multicolumn{1}{c}{(0.373)} &       & \multicolumn{1}{c}{(0.282)} &       & \multicolumn{1}{c}{(0.355)} \\
    Indigenous &       & \multicolumn{1}{c}{-0.346*} &       & \multicolumn{1}{c}{-0.653} &       & \multicolumn{1}{c}{-0.555} &       & \multicolumn{1}{c}{-0.209} &       & \multicolumn{1}{c}{-0.147} &       & \multicolumn{1}{c}{-0.381} \\
          &       & \multicolumn{1}{c}{(0.191)} &       & \multicolumn{1}{c}{(0.619)} &       & \multicolumn{1}{c}{(0.457)} &       & \multicolumn{1}{c}{(0.396)} &       & \multicolumn{1}{c}{(0.357)} &       & \multicolumn{1}{c}{(0.362)} \\
    Constant &       & \multicolumn{1}{c}{-30.86***} &       & \multicolumn{1}{c}{-13.43} &       & \multicolumn{1}{c}{10.54} &       & \multicolumn{1}{c}{-30.38} &       & \multicolumn{1}{c}{-22.57} &       & \multicolumn{1}{c}{74.74**} \\
          &       & \multicolumn{1}{c}{(3.109)} &       & \multicolumn{1}{c}{(36.29)} &       & \multicolumn{1}{c}{(33.80)} &       & \multicolumn{1}{c}{(31.78)} &       & \multicolumn{1}{c}{(30.73)} &       & \multicolumn{1}{c}{(31.47)} \\
          &       & \multicolumn{1}{c}{} &       & \multicolumn{1}{c}{} &       & \multicolumn{1}{c}{} &       & \multicolumn{1}{c}{} &       & \multicolumn{1}{c}{} &       & \multicolumn{1}{c}{} \\
    Observations &       & \multicolumn{1}{c}{23,008} &       & \multicolumn{1}{c}{3,699} &       & \multicolumn{1}{c}{4,223} &       & \multicolumn{1}{c}{4,724} &       & \multicolumn{1}{c}{5,387} &       & \multicolumn{1}{c}{4,975} \\
    \midrule
    \multicolumn{13}{p{63.685em}}{Notes: $^{*}p<0.05$, $^{**}p<0.01$, $^{***}p<0.001$. Standard errors in parentheses. $dy/dx$ for factor levels is the discrete change from the base level. All predictors at their mean value.} \\
    \multicolumn{6}{l}{Source: PNAD-2014, own estimates.} \\
    \end{tabular}%
  \label{tab:Dropout}%
  \end{adjustbox}
\end{table}%
\end{frame}









\begin{frame}{Adjusted Predictions in Secondary Education}
\vspace{-0.3cm}
\begin{block}{\centering Focus on the marginal effects at a point $\tilde{\boldsymbol{x}}$:}
\begin{itemize}
\item \footnotesize How changes in secondary education are related to changes in the ratio 75/10 ?
\end{itemize}

\vspace{-0.6cm}
\begin{equation} \label{eq:MaeginalEffect}
\footnotesize \frac{\partial E (ComSec \vert \boldsymbol{x})}{\partial \boldsymbol{x}} \Bigg\vert_{\boldsymbol{x}=\tilde{\boldsymbol{x}}} = \quad \frac{\partial F (\boldsymbol{x} \boldsymbol{\beta})}{\partial \boldsymbol{x}} \Bigg \vert_{\boldsymbol{x}=\tilde{\boldsymbol{x}}}= f(\tilde{\boldsymbol{x}} \boldsymbol{{\beta}}) \boldsymbol{{\beta}} \nonumber 
\end{equation}
\footnotesize \textcolor{red}{$\Longrightarrow$} Main finding from the chart below: \underline{Education gap increases by rising inequality.}

\end{block}


\vspace{-0.2cm}
\begin{figure}[htb]
\centering
\includegraphics[scale=0.46]{Figure/NoEduc_Ratio7510R2-Peryear}
\begin{minipage}{0.8\textwidth} % choose width suitably
{\fontsize{5}{1}\selectfont 
Notes: Ratio 75/10 represents the relation between the income earned by individuals at the 75th percentile compared to the earnings of individuals at the 10th percentile. Estimations of income inequality based on ratio 75/10 of total income for the economically active population with earnings greater than zero and aged 15 or over. \\ Source: PNAD-2014, own estimates.\par} 
\end{minipage}
\end{figure}
\end{frame}






\section{Conclusion}

\begin{frame} {Conclusion}
\begin{itemize}
\item Intergenerational persistence in education varies substantially across Brazilian states.
\vspace{0.5cm}
\item Statistically signifficant association between intergenerational mobility and income inequality %and expected earnings return to human capital.
\vspace{0.5cm}
\item Confirmation of the ``Great Gatsby curve" at a national level.
\vspace{0.5cm}
\item Children born in families with no education are more likely to leave the school early if they are living in states where the gap between the bottom and middle of income distribution is wider.
\end{itemize}
\end{frame}



\begin{frame}
\vspace{2cm}
\begin{center}
\Huge\textcolor{red}{Thank you for your attention !} \\

\large {More Information? \\ Questions? \\ Suggestions? \\}
\vspace{1cm}
Please contact me: \\
Tharcisio Leone \\
tharcisio.leone@giga-hamburg.de 
\end{center}
\end{frame}





\appendix
\section{Model of Solon (2004)} 

\setcounter{equation}{0}
\begin{frame}[label=SolonModell]{Model of Solon (2004)}
\begin{block}{Intergenerational transmission of inequality}
\begin{itemize}
\item \footnotesize Family (parents and children) is a intergenerational decision maker.
\item \footnotesize Assumption (simplification): No taxation, government investment in children, borrow and bequeath financial assets.
\item \footnotesize Family $i$ contains one parent of generation $t–1$ and one child of generation $t$.
\item \footnotesize Allocation of parent’s lifetime earnings $y_{i,t-1}$ between the parent’s own consumption $C_{i,t-1}$ and investment $I_{i,t-1}$ in the child’s human capital.
\end{itemize}
\end{block}

\begin{itemize}
\item \footnotesize Budget constraint of parents: 

\begin{equation} \label{eq:Budget}
y_{i,t-1}=C_{i,t-1}+I_{i,t-1}
\end{equation}


\item \footnotesize Technology transforms the investment $I_{i,t-1}$ into the child’s human capital $h_{it}$:

\begin{equation} \label{eq: Technology}
h_{it}=\theta \ \log \ I_{i,t-1}+e_{it} 
\end{equation}

\item \tiny $\theta > 0$ represents a positive marginal product for human capital investment.
\item \tiny The semi-log functional form imposes decreasing marginal product.
\item \tiny $e_{it}$ denotes the human capital endowment the child receives regardless of $I_{i,t-1}$.
\end{itemize}
\end{frame}


\begin{frame}{Model of Solon (2004)}

\begin{itemize}

\item \footnotesize Human capital endowment $e_{it}$ follows the first-order autoregressive process: 
\begin{equation} \label{eq: Human}
e_{it}=\delta + \lambda e_{i,t-1}+\upsilon_{i,t}
\end{equation}

\item \tiny $\upsilon_{i,t}$ is a white-noise error term.
\item \tiny $\lambda$ is a heritability coefficient with $\lambda \in [0,1]$.

\item \footnotesize Child’s lifetime income $y_{it}$ is determined by the semi-log earnings function:
\begin{equation} \label{eq: Lifetime}
log \ y_{it}=\mu + ph_{it} 
\end{equation}

\item \tiny $p$ is the earnings return to human capital.

\item \footnotesize Substituting equation \eqref{eq: Technology} into equation \eqref{eq: Lifetime} yields: 
\begin{equation} \label{eq: Substituting}
\log \ y_{it}=\mu + \gamma \ log \ I_{i,t-1} + pe_{it}
\end{equation}

\item \tiny $\gamma$ is the “earnings return to human capital investment".
\item \tiny $ \gamma = \theta p $: The elasticity of the child’s income with respect to investment in the child’s human capital.

\end{itemize}
\end{frame}


\begin{frame}{Model of Solon (2004)}
\begin{block}{\small How does the family decide how much to invest in the child’s human capital?}
\end{block}
\begin{itemize}
\item \footnotesize Maximisation of the Cobb-Douglas utility function:
\begin{equation} \label{eq: Cobb-Douglas} 
U_i=(1-\alpha) \log \ C_{i,t-1}+ \alpha \log \ y_{it}
\end{equation}

\item \tiny $\alpha \in [0,1]$ and measures the parent’s taste for $y_{it}$ relative to $C_{i,t-1}$ (altruism parameter).

\item \footnotesize Substituting equation \eqref{eq: Substituting} into equation \eqref{eq: Cobb-Douglas}:
\begin{equation} \label{eq: rewritten} 
U_i = (1 - \alpha) \log \ (y_{i,t-1}-I_{i,t-1})+ \alpha \mu + \alpha \gamma \log \ I_{i,t-1} + \alpha pe_{it}
\end{equation}


\item \footnotesize The first-order condition for maximizing this utility function is:
\begin{equation} \label{eq: First-order} 
\frac{\partial U_i}{\partial I_{i,t-1}} = \frac{-(1-\alpha)}{y_{i,t-1}-I_{i,t-1}} + \frac{\alpha \gamma}{I_{i,t-1}}=0
\end{equation}

\item \footnotesize Solving for the optimal choice of $I_{i,t-1}$ yields:
\begin{equation} \label{eq: Optimal-choice} 
I_{i,t-1} = \left \{ \frac{\alpha \gamma}{1 - \alpha (1 - \gamma)} \right \} y_{i,t-1}
\end{equation}
\end{itemize}

\flushright \hyperlink{main}{\beamerbutton{\textcolor{red}{Back to Presentation}}}
\end{frame}





\section{Model of Kearney and Levine (2014)}
\setcounter{equation}{0}
\begin{frame}[label=KearneyLevine]{Model of Kearney and Levine (2014)}
\begin{block}{A Stylized Model of the Decision to Dropout Education System}
\begin{itemize}
\item \footnotesize Causal relationship between higher income inequality and school dropout.
\item \footnotesize Economic marginalization for children from socially disadvantaged families.
\item \footnotesize Children maximize a intra-generational utility function between utility in the current $(t)$ and future period $(t+1)$.
\item \footnotesize By Dropout of education: Current-period utility $u^d$ and the present discounted sum of future period $V^d$.
\item \footnotesize By Remain enrolled in the school: $u^e$ and $V^e$.
\end{itemize}
\end{block}

\begin{itemize}
\item \footnotesize Intra-generational utility function:

\begin{equation} \label{eq:Condition1} 
u^d + E(V^d)>u^e + E(V^e)
\end{equation}


\item \footnotesize Assumption: Positive returns to education, such that $E(V^e)>E(V^d)$.

\item \footnotesize Then the decision to dropout education will be never optimal as long as $u^d \leq u^e$.

\item \footnotesize However, psychic costs in school can lead to $u^d > u^e$.
\end{itemize}
\end{frame}





\begin{frame}{Model of Kearney and Levine (2014)}
\begin{itemize}
\item \footnotesize The child's utility in the future can achieve $U^{high}$ or $U^{low}$, with  $p \in [0,1]$.
\item \footnotesize The condition to drop out the school can be rewritten as:

\begin{equation} \label{eq:Condition2} 
u^d + V^{low} > u^e + pV^{high} + (1-p)V^{low}
\end{equation}

\item \footnotesize By rearranging the terms, the condition to remain in the school yields:


\begin{equation} \label{eq:Condition3} 
\left[pV^{high} + (1-p)V^{low} \right] - V^{low} > u^d - u^e  
\end{equation}

\item \footnotesize \underline{Veil of Ignorance}: Child cannot determinate $p$ in the period $t$, working in this way with the individual subjective perception of sucess $q$.
\item \footnotesize $q$ is a function of the likelohood of success $(p)$ and the external factors that affect the individual perception of sucess $(x)$, such that:

\begin{equation}
q=q(p,x)
\end{equation}

\textcolor{red}{$\Longrightarrow$} Effects of Economic Marginalization on $x$: \underline{``It is not for people like me".}
\end{itemize}
\end{frame}



\begin{frame}{Model of Kearney and Levine (2014)}
\begin{itemize}
\item \footnotesize The condition for the student to continue studying follows:

\begin{equation} \label{eq:Condition4} 
\left[qV^{high} + (1-q)V^{low} \right] > V^{low} + (u^d - u^e) 
\end{equation}


\item \footnotesize The reservation subjective probability $q^r$ required for the continuity of students in the school is given as:

\begin{equation} \label{eq:Condition5} 
q \geq q^r = \left \{ \frac{u^d - u^e}{V^{high} - V^{low}} \right \}
\end{equation}


\item \footnotesize The perceived probability of sucess is a function of the interaction of being low SES and inequality, such that:


\begin{equation} \label{eq:Condition7}
\frac{\partial q}{\partial (SocIneq)} \mid (SES=low) < 0
\end{equation}

\textcolor{red}{$\Longrightarrow$} For children from socially weaker families, the increase in the gap between the bottom and middle of the income distribution might lead to a reduction of the subjective perception of $p$.
\end{itemize}
\flushright \hyperlink{main3}{\beamerbutton{\textcolor{red}{Back to Presentation}}}
\end{frame}




\section{Evolution of Mobility over time}

\setcounter{page}{1}
\setcounter{equation}{0}
\begin{frame} [label=Evolution]{Evolution of Mobility over time}
\begin{block}{Questions}
\begin{enumerate}
\item \footnotesize There is only one mobility for all the sample?
\item \footnotesize Did mobility change over time?
\end{enumerate}
\end{block}


\vspace{1.5cm}
\begin{block}{Estimation of the variation over time}
\begin{itemize}
{\footnotesize\item Marginal effect of birth cohorts on schooling of children, from equation (5):}
\end{itemize}

\begin{scriptsize}
\begin{equation} \label{eq:OLSFixed}
\frac{educ^c_{is}}{\sigma^c_s} \:=\: \delta \:+\: \rho \: \frac{educ^p_{is}}{\sigma^p_s} \:+\: \eta \left(\frac{educ^p_{is}}{\sigma^p_s} \times UF_i \right) \:+\: \lambda \: UF_i \:+\: \gamma \left(X_i \times UF_i \right) \:+\: \epsilon_{is} \nonumber
\end{equation}
\end{scriptsize}
\end{block}
\end{frame}



\begin{frame} {Evolution of Mobility over time}
\begin{itemize}
{\footnotesize\item Marginal effect of birth cohorts on schooling of children, from equation (5):}
\end{itemize}

\begin{scriptsize}
\begin{equation} \label{eq:OLSFixed}
\frac{educ^c_{is}}{\sigma^c_s} \:=\: \delta \:+\: \rho \: \frac{educ^p_{is}}{\sigma^p_s} \:+\: \eta \left(\frac{educ^p_{is}}{\sigma^p_s} \times UF_i \right) \:+\: \lambda \: UF_i \:+\: \gamma \left(X_i \times UF_i \right) \:+\: \epsilon_{is} \nonumber
\end{equation}
\end{scriptsize}

%\pause
\begin{figure}[htb]
\centering
\includegraphics[scale=0.5]{Figure/Marginal-Effect-birthc}
\flushright \hyperlink{main2}{\beamerbutton{\textcolor{red}{Back to Presentation}}}
\end{figure}
\end{frame}




\section{Harmonsation of PNAD-Data}
\begin{frame}[label=Harmonsation]{Income Inequality as a Time-Variant Covariate}
\vspace{-0.3cm}		
\begin{block}{\centering Harmonsation of PNAD-Data}
\begin{itemize}
\item \scriptsize Availability of microdata: Sample surveys from 1976 to 2014.
\item \scriptsize Use of sample surveys from 1981 to 2004 in the paper. \\ \tiny \textcolor{red}{$\Longrightarrow$} Exceptions: In 1980, 1991 and 2000 there was no PNAD because Census; and in 1984 due to the recasting of sample. Microdata from 1983 could not be used.
\\ \tiny \textcolor{red}{$\Longrightarrow$} No total coverage of the territory: No Data for the rural population of the North region of Brazil (refers to only $2\%$ of the total population).
\item \scriptsize Proxy for economically active population aged 15 and over.  \\ \tiny \textcolor{red}{$\Longrightarrow$} People that have worked + could work, but were off work + have looked for a work.
\item \scriptsize Use of the variable ``total (individual) income".
\\ \tiny \textcolor{red}{$\Longrightarrow$} Only observations with income grater than 0 (no subsistence work).
\item \scriptsize Calculation of 75/10-Ratio of income inequality per year.
\\ \tiny \textcolor{red}{$\Longrightarrow$} Use of average values of 75/10-Ratio for the estimations in the model.
\end{itemize}
\end{block}

\vspace{-0.2cm}	
\begin{block}{\centering Model Integration}
\begin{itemize}
\item \scriptsize Important is the level of inequality when the individuals concluded the primary (compulsory) education.
 \\ \tiny \textcolor{red}{$\Longrightarrow$} Example: Children born in 1970 were registered in schools in 1976 and finished the primary education in 1984.
 \\ \tiny \textcolor{red}{$\Longrightarrow$} Dropout for cohort 1970-1974 is correlated with the (average) inequality between 1984-1988.
 \item \scriptsize Creation of 5-years bith cohorts. 
\\ \tiny \textcolor{red}{$\Longrightarrow$} Limitation to individuas born between 1965-1989 \textcolor{red}{$\Longrightarrow$} No inequality level for years before 1979.
\end{itemize}
\end{block}

\vspace{-0.5cm}	
\flushright \hyperlink{main5}{\beamerbutton{\textcolor{red}{Back to Presentation}}}
\end{frame}





\section{Robustness Checks}



\begin{frame}[label=Robustness]{Robustness Checks}
\begin{block}{\centering 1. Alternative Econometric Approaches}
\begin{itemize}
		\item \footnotesize NoEduc is proxy for socio-economic marginalisation $(MSB)$.
		
		\item \footnotesize I suspect that $MSB$ ($NoEducP$) is endogenous. \\ \tiny \textcolor{red}{$\Longrightarrow$} The ``feeling" of marginalisation (deprivation) can change according to the economic situation of the family.
		
		\item \footnotesize Instrumental variable ``Lived with both parents". \\ \tiny\textcolor{red}{$\Longrightarrow$} Having both parents in the household can shift the family`s budget constraint, giving a higher socio-economic status for the familiy (this is similar to a higher education level of parents).
		
		\item \footnotesize Three different approaches
		\begin{enumerate}
		\setcounter{enumi}{1}
		\item \footnotesize Linear Probability Model (LPM)
		\item \footnotesize Two-Stage Least Squares (2SLS)
		\item \footnotesize Bivariate Probit
		\end{enumerate}
		
\end{itemize}
\end{block}


\begin{block}{\centering 2. Alternative Model Specifications}
\begin{enumerate}
\setcounter{enumi}{4}
\item \footnotesize Exclusion of Interstate Migrants
\item \footnotesize Ratio 90/10 of Income Inequality
\item \footnotesize Change of variable for socio-economic marginalization ($IlliteP$ instead $NoEducP$)
\item \footnotesize Combination of two variables for MSB ($IlliteP$ and $NoEducP$)
\end{enumerate}
		\end{block}
\end{frame}






\begin{frame}{Table 2: Robustness Checks}

\begin{block}{\centering Focus of attention}
\begin{itemize}
\item \footnotesize Coefficients of the interaction term between $MSB$ and $Inequality$.
\item \footnotesize Predictive margins for every 10th value of Ratio 75/10 (from 4.5 to 8.0).
\end{itemize}
\end{block}

\vspace{0.3cm}
%Table generated by Excel2LaTeX from sheet 'Rob.Checks'
\begin{table}[htbp]
  \centering
%\captionsetup{justification=centering,margin=1cm}
%\caption{\textbf{Robustness Checks}}
\vspace*{-2mm}
 \begin{adjustbox}{max width=\textwidth}
    \begin{tabular}{llllllllllllllll}
    \toprule
          & Main Model &       & \multicolumn{5}{c}{Alternative Econometric Approaches} &       & \multicolumn{7}{c}{Alternative Model Specifications} \\
\cmidrule{2-2}\cmidrule{4-8}\cmidrule{10-16}          & \multicolumn{1}{c}{\textbf{(1)}} &       & \multicolumn{1}{c}{\textbf{(2)}} &       & \multicolumn{1}{c}{\textbf{(3)}} &       & \multicolumn{1}{c}{\textbf{(4)}} &       & \multicolumn{1}{c}{\textbf{(5)}} &       & \multicolumn{1}{c}{\textbf{(6)}} &       & \multicolumn{1}{c}{\textbf{(7)}} &       & \multicolumn{1}{c}{\textbf{(8)}} \\
    \textbf{Model} & \multicolumn{1}{c}{\textbf{Probit}} &       & \multicolumn{1}{c}{\textbf{LPM}} &       & \multicolumn{1}{c}{\textbf{LPM}} &       & \multicolumn{1}{c}{\textbf{Bivariate Probit}} &       & \multicolumn{1}{c}{\textbf{Probit}} &       & \multicolumn{1}{c}{\textbf{Probit}} &       & \multicolumn{1}{c}{\textbf{Probit}} &       & \multicolumn{1}{c}{\textbf{Probit}} \\
    \textbf{Estimation method} & \multicolumn{1}{c}{\textbf{MLE}} &       & \multicolumn{1}{c}{\textbf{OLS}} &       & \multicolumn{1}{c}{\textbf{2SLS}} &       & \multicolumn{1}{c}{\textbf{MLE}} &       & \multicolumn{1}{c}{\textbf{MLE}} &       & \multicolumn{1}{c}{\textbf{MLE}} &       & \multicolumn{1}{c}{\textbf{MLE}} &       & \multicolumn{1}{c}{\textbf{MLE}} \\
    \textbf{Changes to Specidication (1)} & \multicolumn{1}{c}{\textbf{-}} &       & \multicolumn{1}{c}{\textbf{No}} &       & \multicolumn{1}{c}{\textbf{No}} &       & \multicolumn{1}{c}{\textbf{No}} &       & \multicolumn{1}{c}{\textbf{No Migrants}} &       & \multicolumn{1}{c}{\textbf{Ratio 90/10}} &       & \multicolumn{1}{c}{\textbf{Illiterate Parents}} &       & \multicolumn{1}{c}{\textbf{Illiterate and no Educated Parents}} \\
\cmidrule{2-2}\cmidrule{4-4}\cmidrule{6-6}\cmidrule{8-8}\cmidrule{10-10}\cmidrule{12-12}\cmidrule{14-14}\cmidrule{16-16}    Coefficient of MSB  \# Inequality & \multicolumn{1}{c}{-0.0540***} &       & \multicolumn{1}{c}{-0.0179***} &       & \multicolumn{1}{c}{-0.0175***} &       & \multicolumn{1}{c}{-0.0487***} &       & \multicolumn{1}{c}{-0.0695*} &       & \multicolumn{1}{c}{-0.0199**} &       & \multicolumn{1}{c}{-0.0318} &       & \multicolumn{1}{c}{-0.0307} \\
          & \multicolumn{1}{c}{(0.0186)} &       & \multicolumn{1}{c}{(0.00636)} &       & \multicolumn{1}{c}{(0.00636)} &       & \multicolumn{1}{c}{(0.0178)} &       & \multicolumn{1}{c}{(0.0418)} &       & \multicolumn{1}{c}{(0.00796)} &       & \multicolumn{1}{c}{(0.0227)} &       & \multicolumn{1}{c}{(0.0216)} \\
    APRs for MSB and Inequality &       &       &       &       &       &       &       &       &       &       &       &       &       &       &  \\
    1bn.\_at & \multicolumn{1}{c}{-0.223***} &       & \multicolumn{1}{c}{-0.220***} &       & \multicolumn{1}{c}{-0.223***} &       & \multicolumn{1}{c}{-0.175***} &       & \multicolumn{1}{c}{-0.210***} &       & \multicolumn{1}{c}{-0.209***} &       & \multicolumn{1}{c}{-0.236***} &       & \multicolumn{1}{c}{-0.201***} \\
          & \multicolumn{1}{c}{(0.0196)} &       & \multicolumn{1}{c}{(0.0179)} &       & \multicolumn{1}{c}{(0.0179)} &       & \multicolumn{1}{c}{(0.0212)} &       & \multicolumn{1}{c}{(0.0430)} &       & \multicolumn{1}{c}{(0.0271)} &       & \multicolumn{1}{c}{(0.0257)} &       & \multicolumn{1}{c}{(0.0227)} \\
    2.\_at & \multicolumn{1}{c}{-0.244***} &       & \multicolumn{1}{c}{-0.238***} &       & \multicolumn{1}{c}{-0.241***} &       & \multicolumn{1}{c}{-0.190***} &       & \multicolumn{1}{c}{-0.237***} &       & \multicolumn{1}{c}{-0.217***} &       & \multicolumn{1}{c}{-0.247***} &       & \multicolumn{1}{c}{-0.213***} \\
          & \multicolumn{1}{c}{(0.0134)} &       & \multicolumn{1}{c}{(0.0125)} &       & \multicolumn{1}{c}{(0.0125)} &       & \multicolumn{1}{c}{(0.0210)} &       & \multicolumn{1}{c}{(0.0288)} &       & \multicolumn{1}{c}{(0.0242)} &       & \multicolumn{1}{c}{(0.0184)} &       & \multicolumn{1}{c}{(0.0154)} \\
    3.\_at & \multicolumn{1}{c}{-0.264***} &       & \multicolumn{1}{c}{-0.256***} &       & \multicolumn{1}{c}{-0.258***} &       & \multicolumn{1}{c}{-0.205***} &       & \multicolumn{1}{c}{-0.263***} &       & \multicolumn{1}{c}{-0.224***} &       & \multicolumn{1}{c}{-0.258***} &       & \multicolumn{1}{c}{-0.225***} \\
          & \multicolumn{1}{c}{(0.00884)} &       & \multicolumn{1}{c}{(0.00863)} &       & \multicolumn{1}{c}{(0.00861)} &       & \multicolumn{1}{c}{(0.0210)} &       & \multicolumn{1}{c}{(0.0181)} &       & \multicolumn{1}{c}{(0.0213)} &       & \multicolumn{1}{c}{(0.0130)} &       & \multicolumn{1}{c}{(0.0101)} \\
    4.\_at & \multicolumn{1}{c}{-0.284***} &       & \multicolumn{1}{c}{-0.274***} &       & \multicolumn{1}{c}{-0.276***} &       & \multicolumn{1}{c}{-0.220***} &       & \multicolumn{1}{c}{-0.289***} &       & \multicolumn{1}{c}{-0.232***} &       & \multicolumn{1}{c}{-0.268***} &       & \multicolumn{1}{c}{-0.237***} \\
          & \multicolumn{1}{c}{(0.00876)} &       & \multicolumn{1}{c}{(0.00853)} &       & \multicolumn{1}{c}{(0.00851)} &       & \multicolumn{1}{c}{(0.0215)} &       & \multicolumn{1}{c}{(0.0183)} &       & \multicolumn{1}{c}{(0.0185)} &       & \multicolumn{1}{c}{(0.0115)} &       & \multicolumn{1}{c}{(0.0102)} \\
    5.\_at & \multicolumn{1}{c}{-0.305***} &       & \multicolumn{1}{c}{-0.291***} &       & \multicolumn{1}{c}{-0.293***} &       & \multicolumn{1}{c}{-0.236***} &       & \multicolumn{1}{c}{-0.314***} &       & \multicolumn{1}{c}{-0.240***} &       & \multicolumn{1}{c}{-0.278***} &       & \multicolumn{1}{c}{-0.248***} \\
          & \multicolumn{1}{c}{(0.0130)} &       & \multicolumn{1}{c}{(0.0123)} &       & \multicolumn{1}{c}{(0.0123)} &       & \multicolumn{1}{c}{(0.0223)} &       & \multicolumn{1}{c}{(0.0286)} &       & \multicolumn{1}{c}{(0.0159)} &       & \multicolumn{1}{c}{(0.0149)} &       & \multicolumn{1}{c}{(0.0155)} \\
    6.\_at & \multicolumn{1}{c}{-0.324***} &       & \multicolumn{1}{c}{-0.309***} &       & \multicolumn{1}{c}{-0.311***} &       & \multicolumn{1}{c}{-0.253***} &       & \multicolumn{1}{c}{-0.338***} &       & \multicolumn{1}{c}{-0.248***} &       & \multicolumn{1}{c}{-0.288***} &       & \multicolumn{1}{c}{-0.260***} \\
          & \multicolumn{1}{c}{(0.0187)} &       & \multicolumn{1}{c}{(0.0177)} &       & \multicolumn{1}{c}{(0.0177)} &       & \multicolumn{1}{c}{(0.0235)} &       & \multicolumn{1}{c}{(0.0417)} &       & \multicolumn{1}{c}{(0.0133)} &       & \multicolumn{1}{c}{(0.0205)} &       & \multicolumn{1}{c}{(0.0225)} \\
    7.\_at & \multicolumn{1}{c}{-0.344***} &       & \multicolumn{1}{c}{-0.327***} &       & \multicolumn{1}{c}{-0.328***} &       & \multicolumn{1}{c}{-0.270***} &       & \multicolumn{1}{c}{-0.361***} &       & \multicolumn{1}{c}{-0.255***} &       & \multicolumn{1}{c}{-0.298***} &       & \multicolumn{1}{c}{-0.272***} \\
          & \multicolumn{1}{c}{(0.0248)} &       & \multicolumn{1}{c}{(0.0235)} &       & \multicolumn{1}{c}{(0.0235)} &       & \multicolumn{1}{c}{(0.0251)} &       & \multicolumn{1}{c}{(0.0553)} &       & \multicolumn{1}{c}{(0.0111)} &       & \multicolumn{1}{c}{(0.0267)} &       & \multicolumn{1}{c}{(0.0300)} \\
    8.\_at & \multicolumn{1}{c}{-0.363***} &       & \multicolumn{1}{c}{-0.345***} &       & \multicolumn{1}{c}{-0.346***} &       & \multicolumn{1}{c}{-0.287***} &       & \multicolumn{1}{c}{-0.383***} &       & \multicolumn{1}{c}{-0.263***} &       & \multicolumn{1}{c}{-0.307***} &       & \multicolumn{1}{c}{-0.283***} \\
          & \multicolumn{1}{c}{(0.0309)} &       & \multicolumn{1}{c}{(0.0296)} &       & \multicolumn{1}{c}{(0.0296)} &       & \multicolumn{1}{c}{(0.0271)} &       & \multicolumn{1}{c}{(0.0687)} &       & \multicolumn{1}{c}{(0.00927)} &       & \multicolumn{1}{c}{(0.0330)} &       & \multicolumn{1}{c}{(0.0376)} \\
    9.\_at & \multicolumn{1}{c}{-0.382***} &       & \multicolumn{1}{c}{-0.363***} &       & \multicolumn{1}{c}{-0.363***} &       & \multicolumn{1}{c}{-0.305***} &       & \multicolumn{1}{c}{-0.404***} &       & \multicolumn{1}{c}{-0.271***} &       & \multicolumn{1}{c}{-0.315***} &       & \multicolumn{1}{c}{-0.295***} \\
          & \multicolumn{1}{c}{(0.0369)} &       & \multicolumn{1}{c}{(0.0358)} &       & \multicolumn{1}{c}{(0.0358)} &       & \multicolumn{1}{c}{(0.0294)} &       & \multicolumn{1}{c}{(0.0819)} &       & \multicolumn{1}{c}{(0.00822)} &       & \multicolumn{1}{c}{(0.0392)} &       & \multicolumn{1}{c}{(0.0452)} \\
   10.\_at & \multicolumn{1}{c}{-0.401***} &       & \multicolumn{1}{c}{-0.381***} &       & \multicolumn{1}{c}{-0.381***} &       & \multicolumn{1}{c}{-0.323***} &       & \multicolumn{1}{c}{-0.423***} &       & \multicolumn{1}{c}{-0.278***} &       & \multicolumn{1}{c}{-0.324***} &       & \multicolumn{1}{c}{-0.306***} \\
          & \multicolumn{1}{c}{(0.0427)} &       & \multicolumn{1}{c}{(0.0420)} &       & \multicolumn{1}{c}{(0.0420)} &       & \multicolumn{1}{c}{(0.0320)} &       & \multicolumn{1}{c}{(0.0947)} &       & \multicolumn{1}{c}{(0.00819)} &       & \multicolumn{1}{c}{(0.0452)} &       & \multicolumn{1}{c}{(0.0528)} \\
          & \multicolumn{1}{c}{} &       & \multicolumn{1}{c}{} &       & \multicolumn{1}{c}{} &       & \multicolumn{1}{c}{} &       &       &       &       &       &       &       &  \\
    Observations & \multicolumn{1}{c}{23,008} &       & \multicolumn{1}{c}{23,008} &       & \multicolumn{1}{c}{23,008} &       & \multicolumn{1}{c}{23,008} &       & \multicolumn{1}{c}{5,340} &       & \multicolumn{1}{c}{23,008} &       & \multicolumn{1}{c}{24,842} &       & \multicolumn{1}{c}{21,668} \\
     \midrule
    \multicolumn{16}{p{91.685em}}{Notes: The coefficients of the interaction between the socio-economic marginalization (MSB) and the inequality level show how the effects of having (un-)educated parents on the children's chance of schooling change by different values of inequality. The adjusted predictions at representative values (APRs) fixed the covariate "ratio 75/10" to each of the $10$ deciles of the inequality distribution, showing respectively the gap in the chances to achieve a secondary school certificate for the two investigated populations - children from parents with and without (primary) education. For the LPMs, the standard errors are robust to arbitrary heteroskedasticy. $^{*}p<0.05$, $^{**}p<0.01$, $^{***}p<0.001$. Standard errors in parentheses. All predictors at their mean value.} \\
    \multicolumn{16}{l}{Source: PNAD-2014, own estimates.} \\
    \end{tabular}%
  \label{tab:Rob.Checks}%
   \end{adjustbox}
\end{table}%

\vspace{-0.5cm}	
\flushright \hyperlink{main6}{\beamerbutton{\textcolor{red}{Back to Presentation}}}		
\end{frame}

\end{document}